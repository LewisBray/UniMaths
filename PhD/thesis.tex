\documentclass[a4paper, 12pt]{report}

\usepackage{amsmath}
\usepackage{amssymb}
\usepackage{amsthm}
\usepackage{upgreek}
\usepackage{quotes}
\usepackage[nottoc, notlot, notlof]{tocbibind}

\newtheorem{theorem}{Theorem}[section]
\newtheorem{lemma}[theorem]{Lemma}
\newtheorem{corollary}[theorem]{Corollary}
\newtheorem{proposition}[theorem]{Proposition}
\newtheorem{conjecture}[theorem]{Conjecture}
\theoremstyle{remark}
\newtheorem{remark}[theorem]{Remark}
\theoremstyle{definition}
\newtheorem{definition}[theorem]{Definition}
\newtheorem{assumptions}[theorem]{Assumptions}
\newtheorem{example}[theorem]{Example}

\setlength{\unitlength}{10mm}
\renewcommand{\baselinestretch}{1.5}

\newcommand{\overbar}[1]{\mkern 1.5mu\overline{\mkern-1.5mu#1\mkern-1.5mu}\mkern 1.5mu}

\begin{document}

\title{Investigations on Transition Densities of Certain Classes of Stochastic Processes}
\author{Lewis J. Bray}

\maketitle

\thispagestyle{empty}\null\newpage

\pagenumbering{roman}

\chapter*{Acknowledgements}

I would like to take this opportunity to show my appreciation to the many that have helped me in some way in the course of completing my thesis.  Firstly my sincerest thanks go to my supervisor Prof. Niels Jacob.  His supervision and guidance has in no small part gotten me this far and his open door policy has definitely alleviated many headaches of mine and others while almost surely bolstering some of his own.  Special thanks also go to Dr. Neate for his seemingly omniscient insight into Mathematica as well Dr. Evans for general support and advice.

For having diligently taught me mathematics and being nothing but kind to me I would like to gratefully thank the mathematics department, it has been an unforgettable seven years that has prepared me greatly for the challenges ahead.  My thanks also go to my internal and external examiners for taking the time to read my thesis and to Prof. Schilling for allowing me to participate in the 2015 summer school at TUD.

I would be remiss to not mention my office roommates whom were always a pleasure to converse with on any topic; always willing to offer insights and I am sure would agree that “hard work and mathematics” was the only item on the office agenda.  I also wish to thank my girlfriend for her support and putting up with the frequent grunts of dissatisfaction from my desk; in spite of having her own MSc to finish she would always have time for me.

Finally, I wish to thank my family.  Due to the innumerable contributions towards my upbringing, education, funding, support, etc. from each of them it is no overstatement to say that I am here because of them.  Whilst I am doubtful my dependence on them will end anytime soon it does not mean I am any less grateful.

\newpage

\chapter*{Abstract}

In the first part of this thesis we study the geometry of the transition densities of additive processes generated by operators $q(t, D)$ with symbols $q(t, \xi)$ where $q(t, \cdot)$ is negative definite for all $t \ge 0$.  We show that the geometric properties are very analogous to the transition densities of L\'evy process with negative definite characteristic exponent $\psi$.

In the second part we allow our symbols $q(t, x, \xi)$ to also depend on state space and move into the realm of additive-type processes.  We use the geometry developed in the first half to estimate fundamental solutions of additive-type processes and find the symbol of their fundamental solutions.

\null\newpage

\hspace{0.5cm}

\newpage

\pagenumbering{arabic}

\tableofcontents

\chapter*{Introduction}
\addcontentsline{toc}{chapter}{Introduction}

In this thesis we aim to study the transition densities of Markov processes via metrics.  Specifically we consider symmetric L\'evy processes on $\mathbb{R}^n$ and in our exploration we are naturally lead to consider the geometry of additive processes on $\mathbb{R}^n$ which we develop a framework for.  Letting our additive processes depend on the state space we then go on to consider additive-type processes and use our previously developed framework to aid in estimating.

Previously geometry has been used for estimating transition densities of certain stochastic processes and recently has been extended to the case of L\'evy process transition densities as well as to diffusions on fractals, see \cite{m-stableEstimate} and \cite{45in8}.  We begin by giving a brief overview of the subject and some contributions to the field.  The most familiar Markov process is Brownian motion on $\mathbb{R}^n$ which we denote by $(B_t)_{t \ge 0}$.  If this process starts at some $x \in \mathbb{R}^n$ then,
$$
P^x\{B_t \in A\} = \int_Ap_t^G(x, y)\,\mathrm{d}y,
$$
where $A \subset \mathbb{R}^n$ is a Borel set and,
$$
p_t^G(x, y) := (2\pi t)^{-n/2}e^{-\frac{|x - y|^2}{2t}}.
$$
Clearly $d(x, y) := |x - y|$ is a metric (however trivial) and so we see a first example of representing transition densities in terms of metrics.

In terms of analysis, transition densities of Markov processes are fundamental solutions to diffusion equations.  In fact, $p_t^G(x, \cdot)$ is a fundamental solution to the heat equation or,
$$
T_t^Gu_0(x) := \big(p_t^G \ast u_0\big)(x) = \int_{\mathbb{R}^n}p_t^G(x, y)u_0(y)\,\mathrm{d}y,
$$
is the solution to the heat equation with initial value $u_0$.  Here we call $(T_t^G)_{t \ge 0}$ the associated semigroup with the Laplacian operator being its generator.

When dealing with symmetric L\'evy processes with associated characteristic exponent $\psi : \mathbb{R}^n \to \mathbb{R}$, we know that their transition densities $(p_t)_{t > 0}$ are fundamental solutions to the Cauchy problem,
$$
\begin{cases}
\frac{\partial u}{\partial t}(t, x) = -\psi(D)u(t, x), & t \in [0, T],\\
u(0, x) = u_0(x), & x \in \mathbb{R}^n,
\end{cases}
$$
where,
$$
\psi(D)u(x) = (2\pi)^{-n/2}\int_{\mathbb{R}^n}e^{ix\cdot\xi}\psi(\xi)\hat{u}(\xi)\,\mathrm{d}\xi,
$$
and are of the form,
$$
p_t(x) = (2\pi)^{-n}\int_{\mathbb{R}^n}e^{ix\cdot\xi}e^{-t\psi(\xi)}\,\mathrm{d}\xi.
$$
Here $(T_t)_{t \ge 0}$, where $T_tu(x) := \big(p_t \ast u\big)(x)$ for all $t \ge 0$, is the associated semigroup and $-\psi(D)$ its generator.  A more interesting example of geometry being used is for the transition density $p_t^{(\alpha)}(x, y)$ associated with the fractional Laplacian operator $(-\Delta)^{\alpha/2}$, $0 < \alpha < 2$.  Estimates are well-known for this case and for our purposes the last result taken from \cite{m-stableEstimate} is,
$$
p_t^{(\alpha)}(x, y) \simeq \frac{1}{t^{n/\alpha}}\bigg(1 + \frac{|x - y|}{t^{1/\alpha}}\bigg)^{-(n + \alpha)},
$$
which we prefer to modify to,
$$
p_t^{(\alpha)}(x, y) \simeq \frac{1}{t^{n/\alpha}}\bigg(\frac{t^2}{(t^2 + |x - y|^2)}\bigg)^{\frac{n + \alpha}{2}}.
$$
For $\alpha = 1$, i.e. the Cauchy process we find,
$$
p_t^{(1)}(x, y) \simeq \frac{1}{t^n}\bigg(\frac{t^2}{(t^2 + |x - y|^2)}\bigg)^{\frac{n + 1}{2}} = \frac{1}{t^n}e^{-\delta_{C, t}^2(x, y)},
$$
where $\delta_{C, t}(x, y) = \sqrt{\frac{n + 1}{2}\ln\big(\frac{t^2 + |x - y|^2}{t^2}\big)}$ is, for each $t > 0$, a translation invariant metric.  This is of a similar structure to the Brownian motion case.

Currently, in the case of symmetric L\'evy processes, the diagonal term is completely understood in terms of a metric.  For example if we return to the Brownian motion example, we see that we can write,
$$
p_t^G(x, y) = (2\pi)^{-n}\Gamma\Big(\frac{n}{2} + 1\Big)\lambda^{(n)}\Big(B\big(0, \sqrt{2/t}\big)\Big)e^{-\frac{|x - y|^2}{2t}},
$$
where $\Gamma$ is the gamma function and $\lambda^{(n)}\big(B(x, r)\big)$ is the volume of the $n$-dimensional ball in $\mathbb{R}^n$ with centre $x$ and radius $r$.  Hence $p_t^G$ may be written in the form of a volume term multiplied by an exponential decay.  Similarly we see for the Cauchy process in $\mathbb{R}$ that,
$$
p_t^{(1)}(x, y) = \frac{1}{\pi}\frac{t}{t^2 + |x - y|^2} = \frac{1}{2\pi}\lambda^{(1)}\big(B^{d_{C, t}}(0, 1)\big)e^{-\delta_{C, t}^2(x, y)},
$$
where $d_{C, t}(x, y) = \sqrt{t|x - y|}$ is a metric for all $t > 0$ and $B^{d_{C, t}}(0, 1) := \{x \in \mathbb{R}^n : d(x, 0) < 1\}$.  Hence we have another representation which is a volume term multiplied by an exponential decay.  In both of these examples it is clear that the diagonal term is determined by a volume term related to a metric and it has been shown that this holds in general for all L\'evy processes with a real-valued characteristic exponent.  The off-diagonal term, which may also be understood in terms of a different metric, is still being investigated.  

While studying the geometry of L\'evy processes, another family of probability measures naturally arises and when they define a process we call it the dual process of our original L\'evy process.  This process is not necessarily L\'evy but in some cases can be considered as having a time dependent characteristic exponent $q(t, \xi)$ associated to it, we study such functions and the kinds of processes they generate and see if we can also understand the transition densities of this new class of processes using geometric terms.  We then move on to considering functions $q(t, x, \xi)$ as symbols of pseudo-differential operators and look to studying the processes they generate.  There are many problems in this area and we try to apply some of the techniques used in studying L\'evy-type processes to derive results.

We begin by introducing concepts and results used throughout starting with the Fourier transform.  We see how the definition of the Fourier transform can be defined on Schwartz space $\mathcal{S}(\mathbb{R}^n)$ and the tempered distributions $\mathcal{S}'(\mathbb{R}^n)$ (the dual space of Schwartz space) and go through the important convolution theorems.  We then introduce positive definite functions and Bochner's Theorem which is of great importance throughout, it allows us to characterise the Fourier transform of bounded Borel measures as positive definite functions.

A natural follow on is to define negative definite functions.  We state some general properties that we use throughout and state how we can define a convolution semigroup given a negative definite functions.  Here we also define some Sobolev-type spaces related to negative definite functions which are used for estimates later on.  Next we go through some operator semigroup theory, defining important concepts such as generators and Feller semigroups.  It is here we see that for a negative definite function $\psi : \mathbb{R}^n \to \mathbb{C}$ we can define an operator,
$$
-\psi(D)u(x) = -(2\pi)^{-n/2}\int_{\mathbb{R}^n}e^{ix\cdot\xi}\psi(\xi)\hat{u}(\xi)\,\mathrm{d}\xi, \,\,\,\,\, u \in \mathcal{S}(\mathbb{R}^n),
$$
which is shown to be a restriction to the generator of a Feller semigroup.  To finish up the section on operator semigroups we discuss subordination in the sense of Bochner using Bernstein functions which we mainly use to generate examples.

Before moving on from introducing concepts we go through some stochastic process theory with the main result being that starting with a negative definite function, we can construct an associated process which is shown to be a L\'evy process.  Finally we discuss a paper by N. Jacob, et al. \cite{Paper} which is the aforementioned paper that looks to describe the transition densities of L\'evy processes in geometric terms, we walk through the main ideas of defining a metric as the square root of a negative defintie function; their representation of the diagonal term of a transition density in terms of a metric and the stated conjecture which aims to describe the off-diagonal behaviour in terms of another metric.

In the next section we start developing new results.  For motivational purposes, let $\psi : \mathbb{R}^n \to \mathbb{R}$ be a continuous negative definite function with suitable growth conditions, then the transition densities $(p_t)_{t > 0}$ of the associated L\'evy process are given by,
$$
p_t(x) = (2\pi)^{-n}\int_{\mathbb{R}^n}e^{ix\cdot\xi}e^{-t\psi(\xi)}\,\mathrm{d}\xi.
$$
It was pointed out in \cite{Paper} that another family of probability measures can be defined by $\rho_t(\mathrm{d}x) := \frac{e^{-t\psi(x)}}{(2\pi)^np_t(0)}\,\mathrm{d}x$ for all $t > 0$, where this defined a process it was denoted the ``dual'' process of the L\'evy process associated to $\psi$.  However, in the case of the Gaussian we have,
$$
\rho_t^G(\mathrm{d}x) = \frac{e^{-t\frac{|\xi|^2}{2}}}{(2\pi)^n(2\pi t)^{-n/2}}\,\lambda^{(n)}(\mathrm{d}x) = \bigg(\frac{t}{2\pi}\bigg)^{n/2}e^{-t\frac{|x|^2}{2}}\,\lambda^{(n)}(\mathrm{d}x),
$$
which unfortunately vanishes at $t = 0$.  Since we would like our measures to be normalised we require $\rho_t^G \to \varepsilon_0$ (the Dirac measure at $0$) as $t \to 0$ and thus an obvious fix is to replace $t$ with $1/t$.  We show that this holds for all probability measures $\nu_t := \rho_{1/t}$.  Since we have a probability measure that may define a dual process, a natural curiosity is to see if we can find ``generators'' of these dual processes, through some calculation we arrive at,
$$
q(t, \xi) = -\frac{\partial}{\partial t}\ln\frac{p_\frac{1}{t}(\xi)}{p_\frac{1}{t}(0)}, \,\,\,\,\, t > 0.
$$
An open question is whether $q(t, \xi)$ is negative definite or not for all $t > 0$ and we see some examples where it is, however it drives us to consider constructing processes where we begin with time dependent negative definite functions.

We now look to functions $q : [0, \infty) \times \mathbb{R}^n \to \mathbb{C}$ where $q(t, \cdot) : \mathbb{R}^n \to \mathbb{C}$ is negative definite for all $t \ge 0$.  For $u \in \mathcal{S}(\mathbb{R}^n)$ (Schwartz space) we can define the operator,
$$
q(t, D)u(x) = (2\pi)^{-n/2}\int_{\mathbb{R}^n}e^{ix\cdot\xi}q(t, \xi)\hat{u}(\xi)\,\mathrm{d}\xi,
$$
and look to the Cauchy problem,
$$
\begin{cases}
\frac{\partial u}{\partial t}(t, x) + q(t, D)u(t, x) = f(t, x), & t \in [0, T],\\
u(0, x) = u_0(x), & x \in \mathbb{R}^n.
\end{cases}
$$
If our operator was time independent then we would look to semigroups for the solution, this is not the case and so we need an analogous concept, hence we use the notion of fundamental solutions.  A fundamental solution $U(t, s)$ to our Cauchy problem as defined in this thesis has all the properties you would expect from a semigroup except the two parameters means the semigroup property is replaced with the evolution property,
$$
U(t, r)U(r, s) = U(t, s), \,\,\,\,\, 0 \le s \le r \le t \le T.
$$
Following the semigroups analogy we define a family of measures $(\mu_{t, s})_{t \ge s \ge 0}$ and an operator $V(t, s)$ by,
$$
\hat{\mu}_{t, s}(\xi) = (2\pi)^{-n/2}e^{-\int_s^tq(\tau, \xi)\,\mathrm{d}\tau},
$$
and for $u \in \mathcal{S}(\mathbb{R}^n)$,
$$
\begin{aligned}
V(t, s)u(x) & = \int_{\mathbb{R}^n}u(x - y)\,\mu_{t, s}(\mathrm{d}y)\\
& = (2\pi)^{-n/2}\int_{\mathbb{R}^n}e^{ix\cdot\xi}e^{-\int_s^tq(\tau, \xi)\,\mathrm{d}\tau}\hat{u}(\xi)\,\mathrm{d}\xi,
\end{aligned}
$$
and we show that $V(t, s)$ is a fundamental solution to our Cauchy problem.

We continue by imposing that $q$ is real-valued and $q(t, 0) = 0$ for all $t \ge 0$ and we get that $(\mu_{t, s})_{t \ge s \ge 0}$ is a family of probability measures, which we show to be the distributions for the increments of an additive process.  Additive processes can be thought of as L\'evy process with no time homogeneity, this follows from the evolution property of the related fundamental solutions.  From here it is no surprise that $V(t, s)$ can be written as a convolution operator, i.e. $V(t, s)u = p_{t, s} \ast u$ where,
$$
p_{t, s}(x) = (2\pi)^{-n}\int_{\mathbb{R}^n}e^{ix\cdot\xi}e^{-\int_s^tq(\tau, \xi)\,\mathrm{d}\tau}\,\mathrm{d}\xi,
$$
which obviously defines a family of probability densities $(p_{t, s})_{t > s \ge 0}$.  Our intention now is to generalise a result from \cite{Paper} in which they achieve a representation and estimate for the diagonal terms of the probability densities of a L\'evy process in terms of geometry induced by $\psi$.  For a locally bounded negative definite function $\psi : \mathbb{R}^n \to \mathbb{R}$ we have that $\psi(-\xi) = \overbar{\psi(\xi)}$ and $\psi^{1/2}$ is sub-additive, from these properties it is tempting to define a metric $d_\psi(\xi, \eta) := \sqrt{\psi(\xi - \eta)}$ which can be done if we impose $\psi(\xi) = 0$ if and only if $\xi = 0$.  We follow their idea and define a metric $d_{Q_{t, s}}(\xi, \eta) := \sqrt{Q_{t, s}(\xi - \eta)}$ where $Q_{t, s}(\xi) := \int_s^tq(\tau, \xi)\,\mathrm{d}\xi$ and obtain a similar representation of $p_{t, s}(0)$ in terms of $d_{Q_{t, s}}$ along with the estimate,
$$
p_{t, s}(0) \asymp \lambda^{(n)}\Big(B^{d_{Q_{t, s}}}\big(0, \sqrt{\beta_1/\beta_0}\big)\Big),
$$
where $\beta_0q(t_0, \xi) \le q(t, \xi) \le \beta_1q(t_0, \xi)$ for some $t_0 \ge 0$ and $\beta_1, \beta_0 > 0$.  We note that this reduces to the L\'evy process case when $q$ does not depened on time.

In the final section we consider functions $q : [0, \infty) \times \mathbb{R}^n \times \mathbb{R}^n \to \mathbb{C}$ where $q(t, x, \cdot) : \mathbb{R}^n \to \mathbb{C}$ is negative definite for all $(t, x) \in [0, \infty) \times \mathbb{R}^n$.  We adopt the idea of decomposing $q$ in the following manner,
$$
q(t, x, \xi) = q_1(t, \xi) + q_2(t, x, \xi),
$$
and borrowing a result from R. Zhang \cite{RZ}, under some assumptions a fundamental solution $U(t, s)$ to,
$$
\begin{cases}
\frac{\partial u}{\partial t}(t, x) + q(t, x, D)u(t, x) = 0, & t \in [0, T],\\
u(0, x) = u_0(x), & x \in \mathbb{R}^n,
\end{cases}
$$
exists.  Our first intention is to mimic an idea in \cite{EstimatesPaper} where they estimate a semigroup generated by a negative definite symbol against a semigroup generated by a space independent symbol.  In our case we get an estimate of the form,
$$
\begin{aligned}
\|U(t, s)u - V(t, s)u\|_0 & \le c\frac{(t - s)^{1 - \rho}}{1 - \rho}\big\|\big(1 + \psi(\xi)\big)^{1 - \rho}u\big\|_0\\
& = c\frac{(t - s)^{1 - \rho}}{1 - \rho}\|u\|_{\psi, 2(1 - \rho)},
\end{aligned}
$$
for $0 \le \rho < 1$ where $V(t, s)$ is a fundamental solution to the Cauchy problem,
$$
\begin{cases}
\frac{\partial u}{\partial t}(t, x) + q_1(t, D)u(t, x) = 0, & t \in [0, T],\\
u(0, x) = u_0(x), & x \in \mathbb{R}^n.
\end{cases}
$$

The final focus of the thesis is to find the symbol of $U(t, s)$ i.e. represent it as a pseudo-differential operator, the idea is taken from \cite{SymbolPaper} where the symbol $\lambda_t(x, \xi)$ for a Feller semigroup $(T_t)_{t \ge 0}$ generated by a negative definite symbol is found.  As one would expect the symbol for $T_t$ is given by,
$$
\lambda_t(x, \xi) := e_{-\xi}(x)T_te_\xi(x),
$$
where $e_\xi(x) = e^{ix\cdot\xi}$.  However, the author makes the very strong assumption that $e_\xi$ is in the domain of the generator and so we take the approach of \cite{Conservative} which does not require such a strong assumption but means that we need a kernel representation for our fundamental solution $U(t, s)$.  Adapting a paper by B. B\"ottcher \cite{KernelPaper} we can get a kernel $p_{t, s}$ for $U(t, s)$ using the Riesz Representation Theorem by showing that $U(t, s)$ is positivity preserving and acts on $C_\infty(\mathbb{R}^n)$.  Showing also that $\|U(t, s)u\|_\infty \le \|u\|_\infty$ means that we can extend the domain of $U(t, s)$ to the bounded Borel measurable functions and get bounded results, hence we define,
$$
\lambda_{t, s}(x, \xi) = e_{-\xi}(x)U(t, s)e_\xi(x),
$$
which we show to be the symbol of $U(t, s)$.

\newpage\null

\part{Auxiliary Results}

\newpage\null

\chapter{The Fourier Transform and Convolution Semigroups}\label{Ch.oTFTaCS}

In this first part, we introduce concepts used throughout this thesis such as results from Fourier analysis and about negative definite functions in Chapter \ref{Ch.oTFTaCS}; the theory of one-parameter operator semigroups and subordination (in the sense of Bochner) in Chapter \ref{Ch.oOPS}; finishing with some results in stochastic processes with the focus on constructing L\'evy processes in Chapter \ref{Ch.oSP}.  For this our standard references are \cite{Vol1}, \cite{Vol2}, \cite{Vol3} and \cite{Bernstein}, but for sections \ref{Se.PDF} and \ref{Se.CSaNDF} we also refer to \cite{I.25}, for section \ref{Se.OPOS} to \cite{I.235} and \cite{Tanabe} and for sections \ref{Se.KEThm} and \ref{Se.NDFtLP} to \cite{SPRef} as well as \cite{Sato}.  Eventually in \ref{MRtLP} we discuss metrics and metric measure spaces associated to certain L\'evy processes.

\section{The Fourier Transform in $\mathcal{S}(\mathbb{R}^n)$}\label{TFTiSS}

We begin with some topological definitions.

\begin{definition}
Let $X$ be a topological vector space.  A set $G \subset X$ is called:
\begin{enumerate}
\item \emph{convex} if for all $\lambda \in [0, 1]$ and $x, y \in G$ it follows that $\lambda x + (1 - \lambda)y \in G$;

\item \emph{absorbing} if for any $x \in X$ there exists $\lambda > 0$ such that $x \in \lambda G := \{y \in X : y = \lambda z \text{ and } z \in G\}$;

\item \emph{balanced} whenever $x \in G$ and $|\lambda| \le 1$ implies $\lambda x \in G$.
\end{enumerate}
\end{definition}

\begin{definition}
We call a topological vector space $X$ a \emph{locally convex topological vector space} if there exists a base of neighbourhoods of $0 \in X$ consisting of convex, balanced and absorbing sets.
\end{definition}

\begin{definition}
Let $X$ be a vector space over a scalar field $\mathbb{K}$.  We call a function $\nu : X \to \mathbb{R}$ a \emph{seminorm} on $X$ if and only if:
\begin{enumerate}
\item $\nu(x) \ge 0$ for all $x \in X$;

\item $\nu(\lambda x) = |\lambda|\nu(x)$ for all $\lambda \in \mathbb{K}$ and $x \in X$;

\item $\nu(x + y) \le \nu(x) + \nu(y)$ for all $x, y \in X$.
\end{enumerate}
We call a family of seminorms $(\nu_i)_{i \in I}$, where $I$ is an index set, a \emph{family of separating seminnorms} if for some $0 \ne x \in X$ there exists some $\nu_i$ such that $\nu_i(x) \ne 0$.
\end{definition}

If $\mathcal{B}_0$ is a base (of neighbourhoods) of $0 \in X$, then $\mathcal{B}_x := \{x + B_0 : B_0 \in \mathcal{B}_0\}$ is a base of $x \in X$ and the topology of a topological vector space is completely determined by $\mathcal{B}_0$.  This leads to,
\begin{proposition}
Let $(\nu_i)_{i \in I}$, where $I$ is an index set, be a family of separating seminorms on the vector space $X$ and $\mathcal{B}_0'$ the family of subsets $U_{i, \varepsilon} \subset X$ defined for $\varepsilon > 0$ and $i \in I$ by,
$$
U_{i \varepsilon} := \{x \in X : \nu_i(x) < \varepsilon\}.
$$
Then the set $\mathcal{B}_0$ of all finite intersections of elements of $\mathcal{B}_0'$ form a base of $0 \in X$ which turns $X$ into a locally convex topological vector space.
\end{proposition}

We can now introduce the Schwartz space.

\begin{definition}
The \emph{Schwartz space} $\mathcal{S}(\mathbb{R}^n)$ consists of all functions $u \in C^\infty(\mathbb{R}^n)$ such that for all $m_1, m_2 \in \mathbb{N}_0$,
\begin{equation}
p_{m_1, m_2}(u) := \sup_{x \in \mathbb{R}^n}\bigg(\big(1 + |x|^2\big)^{m_1/2}\sum_{|\alpha| \le m_2}|\partial_\alpha u(x)|\bigg) < \infty.
\end{equation}
The family $(p_{m_1, m_2})_{m_1, m_2 \in \mathbb{N}_0}$ forms a family of separating seminorms and hence induce a locally convex topology on $\mathcal{S}(\mathbb{R}^n)$ which is metrizable.
\end{definition}

\begin{definition}
Let $u \in \mathcal{S}(\mathbb{R}^n)$, then the \emph{Fourier transform} of $u$ is defined by,
\begin{equation}
\hat{u}(\xi) := (2\pi)^{-n/2}\int_{\mathbb{R}^n}e^{-ix\cdot\xi}u(x)\,\mathrm{d}x.
\end{equation}
Sometimes we will write $\mathrm{F}u(\xi)$.  We also define the \emph{inverse Fourier transform} of $u$ by,
\begin{equation}
\mathrm{F}^{-1}u(\eta) := (2\pi)^{-n/2}\int_{\mathbb{R}^n}e^{i\eta\cdot y}u(y)\,\mathrm{d}y.\label{IFT}
\end{equation}
\end{definition}
The following theorem summarises some useful properties of the Fourier transform on $\mathcal{S}(\mathbb{R}^n)$.

\begin{theorem}
The Fourier transform $\mathrm{F}$ is a linear bijective and continuous operator from $\mathcal{S}(\mathbb{R}^n)$ into itself which has a continuous inverse given by \eqref{IFT}.
\end{theorem}
\begin{remark}
Note that on $\mathcal{S}(\mathbb{R}^n)$ we have $\mathrm{F}^4 = \mathrm{id}$.
\end{remark}
Of importance will be,
\begin{theorem}
For all $u \in \mathcal{S}(\mathbb{R}^n)$ we have that the Riemann-Lebesgue Lemma,
\begin{equation}
\|\hat{u}\|_\infty \le (2\pi)^{-n/2}\|u\|_{\mathcal{L}^1},\label{R-LL}
\end{equation}
and a first version of the Theorem of Plancherel,
\begin{equation}
\|u\|_0 = \|\hat{u}\|_0,\label{PT}
\end{equation}
holds.
\end{theorem}
\begin{remark}
By polarisation, we get from \eqref{PT} for all $u, v \in \mathcal{S}(\mathbb{R}^n)$,
\begin{equation}
\langle u, v\rangle_0 = \langle\hat{u}, \hat{v}\rangle_0.
\end{equation}
This allows us to extend the Fourier transform from $\mathcal{S}(\mathbb{R}^n)$ to an isometry on $\mathcal{L}^2(\mathbb{R}^n)$, whereas \eqref{R-LL} entails that we can extend the Fourier transform from $\mathcal{S}(\mathbb{R}^n)$ to a continuous linear mapping from $\mathcal{L}^1(\mathbb{R}^n)$ to $C_\infty(\mathbb{R}^n)$.
\end{remark}

There is an important relationship between Fourier transforms and the convolution of functions which we now state.
\begin{definition}
Let $u, v \in \mathcal{S}(\mathbb{R}^n)$, their \emph{convolution} $u\ast v \in \mathcal{S}(\mathbb{R}^n)$ is given by,
\begin{equation}
(u\ast v)(x) := \int_{\mathbb{R}^n}u(x - y)v(y)\,\mathrm{d}y.
\end{equation}
\end{definition}

\begin{theorem}
Let $u, v \in \mathcal{S}(\mathbb{R}^n)$.  Then we have,
\begin{equation}
(u\cdot v)^\wedge(\xi) = (2\pi)^{-n/2}(\hat{u}\ast\hat{v})(\xi),
\end{equation}
and,
\begin{equation}
(u\ast v)^\wedge(\xi) = (2\pi)^{n/2}(\hat{u}\cdot\hat{v})(\xi).
\end{equation}
\end{theorem}


\section{The Fourier Transform in $\mathcal{S}'(\mathbb{R}^n)$}

We now look to extend the Fourier transform from $\mathcal{S}(\mathbb{R}^n)$ to $\mathcal{S}'(\mathbb{R}^n)$ by duality.  We will always consider the weak-$\ast$-topology on $\mathcal{S}'(\mathbb{R}^n)$.
\begin{definition}
$\mathcal{S}'(\mathbb{R}^n)$ is called \emph{the space of tempered distributions}.  It consists of all distributions $u \in \mathcal{D}'(\mathbb{R}^n)$ having a continuous extension to $\mathcal{S}(\mathbb{R}^n)$.
\end{definition}
\begin{remark}
Let $X$ be a topological vector space and $X^\ast$ its dual (i.e. the space of all continuous linear functionals $u : X \to \mathbb{C}$).  The weakest topology on $X^\ast$ which makes all elements of $X$ continuous, i.e. for every $u \in X$ a continuous linear functional on $X^\ast$ is given by $u(x^\ast) = x^\ast(u)$, is called the weak-$\ast$-topology on $X^\ast$.  A sequence $(u_\nu)_{\nu \in \mathbb{N}}$, $u_\nu \in \mathcal{S}'(\mathbb{R}^n)$, converges in the weak-$\ast$-topology to $u \in \mathcal{S}'(\mathbb{R}^n)$ if and only if,
$$
\langle u_\nu, \phi\rangle \to \langle u, \phi\rangle, \,\,\,\, \text{for all } \phi \in \mathcal{S}(\mathbb{R}^n).
$$
\end{remark}

\begin{definition}
Let $u \in \mathcal{S}'(\mathbb{R}^n)$.  Then the \emph{Fourier transform} of $u$ is defined by,
\begin{equation}
\langle\hat{u}, \phi\rangle := \langle u, \hat{\phi}\rangle,
\end{equation}
for all $\phi \in \mathcal{S}(\mathbb{R}^n)$.
\end{definition}
\begin{remark}
For $g \in \mathcal{L}^1(\mathbb{R}^n) \subset \mathcal{S}'(\mathbb{R}^n)$ and $\phi \in \mathcal{S}(\mathbb{R}^n)$ we find,
$$
\langle\hat{g}, \phi\rangle = \langle g, \hat{\phi}\rangle = \int_{\mathbb{R}^n}g(x)\hat{\phi}(x)\,\mathrm{d}x = \int_{\mathbb{R}^n}\hat{g}(x)\phi(x)\,\mathrm{d}x,
$$
which shows that $\hat{g}$ in the sense of $\mathcal{S}'(\mathbb{R}^n)$ coincides with $\hat{g}$ as it is defined on $\mathcal{L}^1(\mathbb{R}^n)$, where $g$ is identified with the functional $\langle g, \cdot \rangle$.  Moreover, since convergence in $\mathcal{L}^2(\mathbb{R}^n)$ implies weak-$\ast$-convergence in $\mathcal{S}'(\mathbb{R}^n)$, we may deduce that the Fourier transform as defined on $\mathcal{S}'(\mathbb{R}^n)$ also extends to the Fourier transform defined on $\mathcal{L}^2(\mathbb{R}^n)$.
\end{remark}

Similar to the Fourier transform on $\mathcal{S}(\mathbb{R}^n)$ we find,
\begin{theorem}
The Fourier transform is a continuous linear operator from $\mathcal{S}'(\mathbb{R}^n)$ into itself which is bijective and has a continuous inverse $\mathrm{F}^{-1} = \mathrm{F}^3$.
\end{theorem}

\begin{definition}
Let $u \in \mathcal{S}'(\mathbb{R}^n)$ and $\varphi \in \mathcal{S}(\mathbb{R}^n)$, their \emph{convolution} $u \ast \varphi$ is defined as,
\begin{equation}
\big(u \ast \varphi\big)(x) := u\big(\varphi(x - \cdot)\big).
\end{equation}
\end{definition}
\begin{remark}
We find that $u \ast \varphi$ is in $\mathcal{S}'(\mathbb{R}^n)$.
\end{remark}

\begin{theorem}
For $u \in \mathcal{S}'(\mathbb{R}^n)$ and $\phi \in \mathcal{S}(\mathbb{R}^n)$ we have,
\begin{equation}
(u\ast\phi)^\wedge(\xi) = (2\pi)^{n/2}(\hat{\phi}\cdot\hat{u})(\xi),
\end{equation}
and,
\begin{equation}
(u\cdot\phi)^\wedge(\xi) = (2\pi)^{-n/2}(\hat{\phi}\ast\hat{u})(\xi).
\end{equation}
\end{theorem}


\section{Positive Definite Functions}\label{Se.PDF}

In this section we study the Fourier transforms of bounded Borel measures $\mu \in \mathcal{M}_b^+(\mathbb{R}^n)$, the focus being on Bochner's Theorem.  We have that $\mathcal{M}_b^+(\mathbb{R}^n) \subset \mathcal{S}'(\mathbb{R}^n)$ and so the Fourier transform of $\mu$ is well-defined.  For $\phi \in \mathcal{S}(\mathbb{R}^n)$ we have,
$$
\begin{aligned}
\langle\hat{\mu}, \phi\rangle & = \langle\mu, \hat{\phi}\rangle\\
& = \int_{\mathbb{R}^n}\hat{\phi}(\xi)\,\mu(\mathrm{d}\xi)\\
& = (2\pi)^{-n/2}\int_{\mathbb{R}^n}\int_{\mathbb{R}^n}e^{-i\xi\cdot x}\phi(x)\,\mathrm{d}x\,\mu(\mathrm{d}\xi)\\
& = \int_{\mathbb{R}^n}\bigg((2\pi)^{-n/2}\int_{\mathbb{R}^n}e^{-ix\cdot\xi}\,\mu(\mathrm{d}\xi)\bigg)\phi(x)\,\mathrm{d}x,
\end{aligned}
$$
hence we have,
\begin{equation}
\hat{\mu}(\xi) = (2\pi)^{-n/2}\int_{\mathbb{R}^n}e^{-ix\cdot\xi}\,\mu(\mathrm{d}\xi).\label{FTBBM}
\end{equation}

\begin{theorem}
Let $\mu \in \mathcal{M}_b^+(\mathbb{R}^n)$.  The Fourier transform of $\mu$ is given by \eqref{FTBBM} and it is a uniformly continuous function on $\mathbb{R}^n$.
\end{theorem}

\begin{theorem}
For $\mu, \nu \in \mathcal{M}_b^+(\mathbb{R}^n)$ the convolution theorem holds, i.e.,
\begin{equation}
(\mu\ast\nu)^\wedge(\xi) = (2\pi)^{n/2}(\hat{\mu}\cdot\hat{\nu})(\xi).
\end{equation}
\end{theorem}

To characterise the Fourier transforms of bounded Borel measures we need the notion of positive definite functions.
\begin{definition}
A function $u : \mathbb{R}^n \to \mathbb{C}$ is called \emph{positive definite} if for any choice of $k \in \mathbb{N}$ and vectors $\xi^1, \dots, \xi^k \in \mathbb{R}^n$ the matrix $\big(u(\xi^j - \xi^l)\big)_{j, l = 1, \dots, k}$ is positive Hermitian, i.e. for all $\lambda_1, \dots, \lambda_k \in \mathbb{C}$ we have,
\begin{equation}
\sum_{j, l = 1}^ku(\xi^j - \xi^l)\lambda_j\bar{\lambda}_l \ge 0.
\end{equation}
\end{definition}

The following theorem of Bochner is fundamental in characterising the Fourier transforms of bounded Borel measures.
\begin{theorem}
A function $u : \mathbb{R}^n \to \mathbb{C}$ is the Fourier transform of a bounded Borel measure $\mu \in \mathcal{M}_b^+(\mathbb{R}^n)$ with total mass $\|\mu\| := \mu(\mathbb{R}^n)$, if and only if the following conditions are fulfilled:
\begin{enumerate}
\item $u$ is continuous;

\item $u(0) = \hat{\mu}(0) = (2\pi)^{-n/2}\|\mu\|$;

\item $u$ is positive definite.
\end{enumerate}
\end{theorem}


\section{Convolution Semigroups and Negative\\Definite Functions}\label{Se.CSaNDF}

Having introduced positive definite functions, we now explore negative definite functions and their link to convolution semigroups.
\begin{definition}
Let $G$ be a locally compact space.  Further let $(\mu_\nu)_{\nu \in \mathbb{N}}$ be a sequence in $\mathcal{M}_b(G)$ and $\mu \in \mathcal{M}_b(G)$.  We say that the sequence $(\mu_\nu)_{\nu \in \mathbb{N}}$ \emph{converges weakly} to $\mu$ if for all $u \in C_b(G; \mathbb{R})$ we have,
\begin{equation}
\lim_{\nu \to \infty}\int_Gu(x)\,\mu_\nu(\mathrm{d}x) = \int_Gu(x)\,\mu(\mathrm{d}x).\label{WCoM}
\end{equation}
If \eqref{WCoM} holds only for all $u \in C_0(G; \mathbb{R})$, we say that $(\mu_\nu)_{\nu \in \mathbb{N}}$ \emph{converges vaguely} to $\mu$.
\end{definition}
\begin{remark}
Since $C_0(G; \mathbb{R}^n) \subset C_b(G; \mathbb{R}^n)$ we obviously have that weak convergence implies vague convergence.
\end{remark}

We also need for later,

\begin{theorem}\label{VCimpliesWC}
Suppose that $(\mu_\nu)_{\nu \in \mathbb{N}}$, $\mu_\nu \in \mathcal{M}_b^+(\mathbb{R}^n)$, converges vaguely to $\mu \in \mathcal{M}_b^+(\mathbb{R}^n)$ as $\nu \to \infty$ and $\mu_\nu(\mathbb{R}^n) \to \mu(\mathbb{R}^n)$.  Then $(\mu_\nu)_{\nu \in \mathbb{N}}$ converges weakly to $\mu$.
\end{theorem}
\begin{corollary}
A sequence of probability measures on $G$ converges weakly if and only if it converges vaguely to a probability measure.
\end{corollary}

Now we introduce convolution semigroups.

\begin{definition}
A family $(\mu_t)_{t \ge 0}$ of bounded Borel measures on $\mathbb{R}^n$ is called a \emph{convolution semigroup} on $\mathbb{R}^n$ if and only if the following conditions are fulfilled:
\begin{enumerate}
\item $\mu_t(\mathbb{R}^n) \le 1$ for all $t \ge 0$;

\item $\mu_s \ast \mu_t = \mu_{t + s}$ for $s, t \ge 0$ and $\mu_0 = \varepsilon_0$;

\item $\mu_t \to \varepsilon_0$ vaguely as $t \to 0$,
\end{enumerate}
where for $a \in \mathbb{R}^n$, $\varepsilon_a$ denotes the Dirac measure at $a$.
\end{definition}

\begin{definition}
A function $\psi : \mathbb{R}^n \to \mathbb{C}$ is called negative definite if $\psi(0) \ge 0$ and $\xi \mapsto (2\pi)^{-n/2}e^{-t\psi(\xi)}$ is positive definite for $t \ge 0$.
\end{definition}

The following theorem is of huge importance and establishes a one-to-one correspondence between convolution semigroups on $\mathbb{R}^n$ and continuous negative definite functions.
\begin{theorem}
For any convolution semigroup $(\mu_t)_{t \ge 0}$ on $\mathbb{R}^n$ there exists a uniquely determined continuous negative definite function $\psi : \mathbb{R}^n \to \mathbb{C}$ such that,
\begin{equation}
\hat{\mu}_t(\xi) = (2\pi)^{-n/2}e^{-t\psi(\xi)},\label{CSCNDF}
\end{equation}
holds for all $\xi \in \mathbb{R}^n$ and $t \ge 0$.

Conversely, given a continuous negative definite function $\psi : \mathbb{R}^n \to \mathbb{C}$, then there exists a unique convolution semigroup $(\mu_t)_{t \ge 0}$ on $\mathbb{R}^n$ such that \eqref{CSCNDF} holds.
\end{theorem}

For reference purposes we summarise some properties of negative definite functions.
\begin{theorem}\label{FTaSNDFT1}
Let $\psi : \mathbb{R}^n \to \mathbb{C}$ be a negative definite function.
\begin{enumerate}
\item For any choice of $k \in \mathbb{N}$ and vectors $\xi^1, \dots, \xi^k \in \mathbb{R}^n$ the matrix,
$$
\big(\psi(\xi^j) + \overbar{\psi(\xi^l)} - \psi(\xi^j - \xi^l)\big)_{j, l = 1, \dots, k},
$$
is positive Hermitian;

\item $\psi(\xi) = \overbar{\psi(-\xi)}$;

\item $\sqrt{|\psi(\xi + \eta)|} \le \sqrt{|\psi(\xi)|} + \sqrt{|\psi(\eta)|}$;

\item We have,
$$
\frac{1 + |\psi(\xi)|}{1 + |\psi(\eta)|} \le 2\big(1 + |\psi(\xi - \eta)|\big),
$$
which is known as Peetre's inequality;

\item If $\psi$ is also locally bounded, for example continuous, then there exists a constant $c_\psi > 0$ such that for all $\xi \in \mathbb{R}^n$,
$$
|\psi(\xi)| \le c_\psi\big(1 + |\xi|^2\big);
$$

\item The set of negative definite functions is a convex cone closed under pointwise convergence;

\item The set of continuous negative definite functions is a convex cone which is closed under uniform convergence on compact sets.
\end{enumerate}
\end{theorem}


\section{Some Function Spaces Related to Negative Definite Functions}

We now take a brief moment to define some function spaces related to negative definite functions which are used later on for the purpose of estimating.  These spaces are modelled after classical Bessel potential spaces, see \cite{I.1} or \cite{II.256}, and were introduced in \cite{II.256}, see also \cite{II.87}.  For this section let $\psi : \mathbb{R}^n \to \mathbb{R}$ be a continuous negative definite function.
\begin{definition}
Let $(X, \|\cdot\|_X)$ and $(Y, \|\cdot\|_Y)$ be two normed spaces.  We say that $X$ is \emph{continuously embedded} in $Y$ if $X \subset Y$ and the identity map is continuous, i.e. $\|u\|_Y \le c\|u\|_X$ for some constant $c > 0$ and all $u \in X$.  We write $X \hookrightarrow Y$ when $X$ is continuously embedded in $Y$.
\end{definition}

\begin{definition}
For any $\sigma \in \mathbb{R}$ we define the space,
\begin{equation}
H^{\psi, \sigma}(\mathbb{R}^n) := \left\{u \in \mathcal{S}'(\mathbb{R}^n) : \|u\|_{\psi, s} < \infty\right\},
\end{equation}
where,
\begin{equation}
\|u\|_{\psi, s} = \left\|\big(1 + \psi(\cdot)\big)^{s/2}\hat{u}\right\|_0.
\end{equation}
\end{definition}
\begin{remark}
Clearly $H^{\psi, 0}(\mathbb{R}^n) = \mathcal{L}^2(\mathbb{R}^n)$.
\end{remark}
These spaces are often referred to as anisotropic Sobolev spaces associated to $\psi$ although the name anisotropic Bessel potential spaces would be more appropriate.
For our purposes we need to know only a few results about these spaces.  First we note that $\mathcal{S}(\mathbb{R}^n) \subset H^{\psi, t}(\mathbb{R}^n) \subset H^{\psi, s}(\mathbb{R}^n) \subset \mathcal{L}^2(\mathbb{R}^n)$, $0 < s < t$, and these are dense inclusions.  Most important is that if we impose $\psi(\xi) \ge c_0|\xi|^{r_0}$ for $\xi$ large and some constants $c_0, r_0 > 0$, then for all $\sigma > n/r_0$ we have the continuous embedding $H^{\psi, \sigma}(\mathbb{R}^n) \hookrightarrow C_\infty(\mathbb{R}^n)$ and the following sequence of dense inclusions,
$$
C_0^\infty(\mathbb{R}^n) \subset \mathcal{S}(\mathbb{R}^n) \subset H^{\psi, \sigma}(\mathbb{R}^n) \subset C_\infty(\mathbb{R}^n),
$$
which we use repeatedly later on.


\chapter{One Parameter Semigroups}\label{Ch.oOPS}

\section{One Parameter Operator Semigroups}\label{Se.OPOS}

First we will introduce some results from the theory of one-parameter operator semigroups on a Banach space $(X, \|\cdot\|_X)$.  Then we will focus on Feller semigroups on $C_\infty(\mathbb{R}^n; \mathbb{R})$ and sub-Markovian semigroups on $\mathcal{L}^2(\mathbb{R}^n; \mathbb{R})$.
\begin{definition}
\hspace{0.5cm}
\begin{itemize}
\item[A.] A one-parameter family $(T_t)_{t \ge 0}$ of bounded linear operators $T_t : X \to X$ is called a \emph{(one-parameter) semigroup of operators}, if $T_0 = \mathrm{id}$ and $T_{s + t} = T_s \circ T_t$ hold for all $s, t \ge 0$.

\item[B.] We call $(T_t)_{t \ge 0}$ \emph{strongly continuous} if,
\begin{equation}
\lim_{t \to 0}\|T_tu - u\|_X = 0,
\end{equation}
for all $u \in X$.

\item[C.] The semigroup $(T_t)_{t \ge 0}$ is called a \emph{contraction semigroup} if for all $t \ge 0$,
\begin{equation}
\|T_t\| \le 1,
\end{equation}
holds where $\|\cdot\|$ denotes the operator norm.
\end{itemize}
\end{definition}

\begin{definition}
Let $(T_t)_{t \ge 0}$ be a strongly continuous contraction semigroup on $\big(C_\infty(\mathbb{R}^n; \mathbb{R}), \|\cdot\|_\infty\big)$ which is \emph{positivity preserving}, i.e. $u \ge 0$ implies $T_tu \ge 0$ for all $t \ge 0$.  Then $(T_t)_{t \ge 0}$ is called a \emph{Feller semigroup}.
\end{definition}

\begin{example}\label{FTaSOSE1}
Let $(\mu_t)_{t \ge 0}$ be a convolution semigroup on $\mathbb{R}^n$.  On the Banach space $\big(C_\infty(\mathbb{R}^n; \mathbb{R}), \|\cdot\|_\infty\big)$ we define the operator,
\begin{equation}
T_tu(x) := \int_{\mathbb{R}^n}u(x - y)\,\mu_t(\mathrm{d}y).\label{FSO}
\end{equation}
We claim that $(T_t)_{t \ge 0}$ is a Feller semigroup.  First, since $u \in C_\infty(\mathbb{R}^n)$ is bounded we find,
$$
|T_tu(x)| \le \int_{\mathbb{R}^n}|u(x - y)|\,\mu(\mathrm{d}y) \le \|u\|_\infty\,\mu(\mathbb{R}^n) \le \|u\|_\infty,
$$
which implies,
\begin{equation}
\sup_{x \in \mathbb{R}^n}|T_tu(x)| \le \|u\|_\infty,\label{E1}
\end{equation}
i.e. for all $t \ge 0$ the operator $T_t$ is defined on $C_\infty(\mathbb{R}^n)$ and $T_tu$ is a bounded function.  It is now easy to see that $T_tu \in C_\infty(\mathbb{R}^n)$, in fact for $u \in \mathcal{S}(\mathbb{R}^n)$ we find using the convolution theorem,
\begin{equation}
(T_tu)^\wedge(\xi) = (2\pi)^{n/2}\hat{u}(\xi)\hat{\mu}_t(\xi) = \hat{u}(\xi)e^{-t\psi(\xi)},\label{E2}
\end{equation}
where $\psi : \mathbb{R}^n \to \mathbb{C}$ is a continuous negative definite function.  This implies however, that $(T_tu)^\wedge \in \mathcal{L}^1(\mathbb{R}^n)$ for $u \in \mathcal{S}(\mathbb{R}^n)$ and the Riemann-Lebesgue Lemma \eqref{R-LL} implies $T_tu \in C_\infty(\mathbb{R}^n)$.  Thus, by \eqref{E1} we find using the density of $\mathcal{S}(\mathbb{R}^n)$ in $C_\infty(\mathbb{R}^n)$ that $T_t$ is a contraction on $C_\infty(\mathbb{R}^n)$ for all $t \ge 0$.  Now,
$$
\begin{aligned}
T_s \circ T_tu(x) & = \int_{\mathbb{R}^n}\bigg(\int_{\mathbb{R}^n}u(x - z - y)\,\mu_t(\mathrm{d}y)\bigg)\mu_s(\mathrm{d}z)\\
& = \int_{\mathbb{R}^n}u(x - z)\,\big(\mu_t\ast\mu_s\big)(\mathrm{d}z)\\
& = \int_{\mathbb{R}^n}u(x - z)\,\mu_{t + s}(\mathrm{d}z)\\
& = T_{s + t}u(x).
\end{aligned}
$$
Also, since $\mu_0 = \varepsilon_0$ we see that,
$$
T_0u(x) = \int_{\mathbb{R}^n}u(x - y)\,\varepsilon_0(\mathrm{d}y) = u(x),
$$
i.e. $T_0 = \mathrm{id}$.  Next we show that $(T_t)_{t \ge 0}$ is strongly continuous.  For this, note that any function in $C_\infty(\mathbb{R}^n)$ is uniformly continuous, hence for $\varepsilon > 0$ there exists $\delta > 0$ such that,
$$
|u(x) - u(x - y)| < \varepsilon \,\,\,\, \text{for } \,\, |y| < \delta.
$$
The continuity of $(\mu_t)_{t \ge 0}$ in the Bernoulli topology (the topology referring to the weak convergence of measures) implies that,
$$
\lim_{t \to 0}\mu_t\big(B_\delta(0)\big) = \varepsilon_0\big(B_\delta(0)\big) = 1,
$$
which gives,
$$
\mu_t\big(B_\delta^c(0)\big) < \varepsilon \,\,\,\, \text{and } \,\, 1 - \mu_t(\mathbb{R}^n) < \varepsilon,
$$
for $0 < t \le t_0$.  Now we find,
$$
\begin{aligned}
|T_tu(x) - u(x)| & \le \left|\int_{\mathbb{R}^n}\big(u(x - y) - u(x)\big)\,\mu_t(\mathrm{d}y)\right| + |u(x)|\big(1 - \mu_t(\mathbb{R}^n)\big)\\
& \le \int_{B_\delta(0)}|u(x - y) - u(x)|\,\mu_t(\mathrm{d}y)\\
& \,\,\,\,\,\,\, + \int_{B_\delta^c(0)}|u(x - y) - u(x)|\,\mu_t(\mathrm{d}y) + |u(x)|\big(1 - \mu_t(\mathbb{R}^n)\big)\\
& \le \varepsilon + 2\varepsilon\|u\|_\infty + \varepsilon\|u\|_\infty\\
& = \varepsilon\big(1 + 3\|u\|_\infty\big),
\end{aligned}
$$
implying the strong continuity of $(T_t)_{t \ge 0}$ as $t \to 0$.  Finally, for all $t \ge 0$, if $u \ge 0$ then we find easily from \eqref{FSO} that $T_tu \ge 0$ also.  Thus, $(T_t)_{t \ge 0}$ is a Feller semigroup.
\end{example}

\begin{definition}
Let $(T_t)_{t \ge 0}$ be a strongly continuous contraction semigroup on $\mathcal{L}^p(\mathbb{R}^n; \mathbb{R})$, $1 \le p < \infty$.  We call $(T_t)_{t \ge 0}$ a \emph{sub-Markovian semigroup} on $\mathcal{L}^p(\mathbb{R}^n; \mathbb{R})$, $1 \le p < \infty$ if for $u \in \mathcal{L}^p(\mathbb{R}^n; \mathbb{R})$ such that $0 \le u \le 1$ a.e. it follows that $0 \le Tu \le 1$ a.e.
\end{definition}
\begin{remark}
We find that $(T_t)_{t \ge 0}$ as defined by \eqref{FSO} in Example \ref{FTaSOSE1} is a sub-Markovian semigroup on $\mathcal{L}^2(\mathbb{R}^n; \mathbb{R})$
\end{remark}

We develop some more theory of one-parameter semigroups by introducing the important notion of a generator.
\begin{definition}
Let $(T_t)_{t \ge 0}$ be a strongly continuous contraction semigroup of operators on a Banach space $\big(X, \|\cdot\|_X\big)$.  The \emph{generator} $A$ of $(T_t)_{t \ge 0}$ is defined by,
\begin{equation}
Au := \lim_{t \to 0}\frac{T_tu - u}{t},
\end{equation}
as a strong limit with domain,
\begin{equation}
D(A) := \left\{u \in X : \lim_{t \to 0}\frac{T_tu - u}{t} \,\,\, \text{exists as strong limit}\right\}.
\end{equation}
\end{definition}

We return to Example \ref{FTaSOSE1}.
\begin{example}\label{FTaSOSE2}
Let $\psi : \mathbb{R}^n \to \mathbb{C}$ be a continuous negative definite function with corresponding convolution semigroup $(\mu_t)_{t \ge 0}$, i.e. for all $t \ge 0$, $\hat{\mu}_t(\xi) = (2\pi)^{-n/2}e^{-t\psi(\xi)}$.  Moreover, let $(T_t)_{t \ge 0}$ be the Feller semigroup defined by \eqref{FSO}.  Let $u \in \mathcal{S}(\mathbb{R}^n)$ and note that $\mathcal{S}(\mathbb{R}^n) \subset C_\infty(\mathbb{R}^n)$.  Since $\hat{u} \in \mathcal{S}(\mathbb{R}^n)$ and, by Theorem \ref{FTaSNDFT1}, $|\psi(\xi)| \le c_\psi\big(1 + |\xi|^2\big)$ we can define the operator,
\begin{equation}\label{PDOwNDSeg}
\psi(D)u(x) := (2\pi)^{-n/2}\int_{\mathbb{R}^n}e^{ix\cdot\xi}\psi(\xi)\hat{u}(\xi)\,\mathrm{d}\xi.
\end{equation}
We claim that $\mathcal{S}(\mathbb{R}^n) \subset D\big(-\psi(D)\big)$ and that $-\psi(D)$ extends to the generator of the Feller semigroup $(T_t)_{t \ge 0}$.  First note that,
$$
\frac{T_tu - u}{t} = (2\pi)^{-n/2}\int_{\mathbb{R}^n}e^{ix\cdot\xi}\Big(\frac{e^{-t\psi(\xi)} - 1}{t}\Big)\hat{u}(\xi)\,\mathrm{d}\xi.
$$
By using the estimate,
$$
\left|\frac{e^{-at} - 1 + at}{t}\right| \le \frac{1}{2}a^2t, \,\,\, t \ge 0, \, a \ge 0,
$$
we find that,
$$
\left|\frac{e^{-t\psi(\xi)} - 1 + t\psi(\xi)}{t}\right| \le t|\psi(\xi)|^2 \le tc_\psi\big(1 + |\xi|^2\big)^2.
$$
This gives,
$$
\left\|\frac{T_tu - u}{t} + \psi(D)u\right\|_\infty \le tc_\psi\int_{\mathbb{R}^n}\big(1 + |\xi|^2\big)^2|\hat{u}(\xi)|\,\mathrm{d}\xi,
$$
which implies,
$$
\lim_{t \to 0}\frac{T_tu - u}{t} = -\psi(D)u.
$$
\end{example}
\begin{remark}
In general it is difficult to characterise $D\big(-\psi(D)\big)$ in terms of function spaces.  See \cite{Vol1}.
\end{remark}

We present a theorem from the general theory of semigroups that gives properties of generators and from this we get a useful corollary.  We first however need a definition.
\begin{definition}
Let $\big(A, D(A)\big)$ be a linear operator from $X$ to $Y$ where $X$ and $Y$ are both Banach spaces.
\begin{itemize}
\item[A.] We call $A$ a \emph{closed operator} if $\Gamma(A)$, the graph of the operator, is closed in $X \times Y$.

\item[B.] The operator $A$ is \emph{closable} if it has a closed extension.  The smallest closed extension of $A$ is called its \emph{closure} and denoted by $\big(\bar{A}, D(\bar{A})\big)$.
\end{itemize}
\end{definition}

\begin{theorem}
Let $(T_t)_{t \ge 0}$ be a strongly continuous contraction semigroup on the Banach space $\big(X, \|\cdot\|_X\big)$ and denote by $A$ its generator with domain $D(A) \subset X$.
\begin{itemize}
\item[A.] For any $u \in X$ and $t \ge 0$ it follows that $\int_0^tT_su\,\mathrm{d}s \in D(A)$ and,
\begin{equation}
T_tu - u = A\int_0^tT_su\,\mathrm{d}s.
\end{equation}

\item[B.] For $u \in D(A)$ and $t \ge 0$ we have $T_tu \in D(A)$, i.e. $D(A)$ is invariant under $T_t$, and,
\begin{equation}
\frac{\mathrm{d}}{\mathrm{d}t}T_tu = AT_tu = T_tAu.
\end{equation}

\item[C.] For $u \in D(A)$ and $t \ge 0$ we always get,
\begin{equation}
T_tu - u = \int_0^tAT_su\,\mathrm{d}s = \int_0^tT_sAu\,\mathrm{d}s.
\end{equation}
\end{itemize}
\end{theorem}
\begin{corollary}
Let $A$ be the generator of a strongly continuous contraction semigroup $(T_t)_{t \ge 0}$ on the Banach space $\big(X, \|\cdot\|_X\big)$.  Then $D(A) \subset X$ is a dense subspace and $A$ is a closed operator.  Moreover, $(T_t)_{t \ge 0}$ is a strongly continuous semigroup on $D(A)$ when $D(A)$ is equipped with the graph norm $\|\cdot\|_{A, X} := \|A\cdot\|_X + \|\cdot\|_X$.
\end{corollary}

To finish up this section we want to state a version of the Hille-Yoshida-Ray Theorem, compare \cite{I.88} or \cite{Vol1}.  However we first need,
\begin{definition}
A linear operator $A : D(A) \to X$, $D(A) \subset X$, is called \emph{dissipative} if,
$$
\|\lambda u - Au\|_X \ge \lambda\|u\|_X,
$$
holds for all $\lambda > 0$ and $u \in D(A)$.
\end{definition}

\begin{theorem}
A linear operator $\big(A, D(A)\big)$, on a Banach space $X$ is the generator of a strongly continuous contraction semigroup if and only if the following conditions hold:
\begin{enumerate}
\item $D(A)$ is dense in $X$;

\item $\big(A, D(A)\big)$ is a dissipative operator;

\item $R(\lambda - A) = X$ for some $\lambda > 0$.
\end{enumerate}
\end{theorem}

We also need a variation on this theorem.

\begin{definition}
Let $A : D(A) \to B(\mathbb{R}^n; \mathbb{R})$ be a linear operator with $D(A) \subset B(\mathbb{R}^n; \mathbb{R})$.  We say $\big(A, D(A)\big)$ satisfies the \emph{positive maximum principle} if for any $u \in D(A)$ such that $u(x_0) := \sup_{x \in \mathbb{R}^n}u(x) \ge 0$ we have that $Au(x_0) \le 0$.
\end{definition}

\begin{theorem}
A linear operator $\big(A, D(A)\big)$, $D(A) \subset C_\infty(\mathbb{R}^n; \mathbb{R})$ is closable and its closure is the generator of a Feller semigroup if and only if the following conditions hold:
\begin{enumerate}
\item $D(A)$ is dense in $C_\infty(\mathbb{R}^n; \mathbb{R})$;

\item $\big(A, D(A)\big)$ satisfies the positive maximum principle;

\item $R(\lambda - A)$ is dense in $C_\infty(\mathbb{R}^n; \mathbb{R})$ for some $\lambda > 0$.
\end{enumerate}
\end{theorem}

\begin{example}
Let $\psi : \mathbb{R}^n \to \mathbb{C}$ be a continuous negative definite function.  On $C_0^\infty(\mathbb{R}^n; \mathbb{R})$ we define the operator,
$$
-\psi(D)u(x) = -(2\pi)^{n/2}\int_{\mathbb{R}^n}e^{ix\cdot\xi}\psi(\xi)\hat{u}(\xi)\,\mathrm{d}\xi.
$$
From Example \ref{FTaSOSE1} and Example \ref{FTaSOSE2} we know that $\big(-\psi(D), C_0^\infty(\mathbb{R}^n; \mathbb{R})\big)$ has an extension generating a Feller semigroup, hence the operator $\big(-\psi(D), C_0^\infty(\mathbb{R}^n; \mathbb{R})\big)$ satisfies the positive maximum principle.
\end{example}


\section{Bernstein Functions and Subordination in the Sense of Bochner}\label{BFaSitSoB}

Bernstein functions are a useful tool for generating examples of continuous negative definite functions so we present the theory of subordination which is used to achieve this.
\begin{definition}
A real-valued function $f \in C^\infty\big((0, \infty)\big)$ is said to be \emph{completely monotone} if,
\begin{equation}
(-1)^k\frac{\mathrm{d}^kf}{\mathrm{d}x^k}(x) \ge 0,
\end{equation}
holds for all $k \in \mathbb{N}_0$.
\end{definition}

\begin{definition}
A real-valued function $f \in C^\infty\big((0, \infty)\big)$ is called a \emph{Bernstein function} if,
\begin{equation}
f \ge 0 \,\,\,\,\, \text{and} \,\,\,\,\, (-1)^k\frac{\mathrm{d}^kf}{\mathrm{d}x^k}(x) \le 0,
\end{equation}
holds for all $k \in \mathbb{N}$.
\end{definition}

\begin{definition}
A function $f : (0, \infty) \to \mathbb{R}$ is called a \emph{complete Bernstein function} if there exists a Bernstein function $g \ge 0$ such that,
\begin{equation}
f(x) = x^2\mathcal{L}g(x),
\end{equation}
holds for all $x > 0$ where $\mathcal{L}$ is the Laplace transform of a function defined by,
$$
\mathcal{L}g(x) := \int_0^\infty e^{-xy}g(y)\,\mathrm{d}y.
$$
\end{definition}

A Bernstein function is positive, increasing and concave.  If $f$ is a Bernstein function then $f'$ is completely monotone and conversely if $g'$ is completely monotone then $g$ is a Bernstein function.  The following proposition gives a further link between Bernstein and completely monotone functions.

\begin{proposition}
For a function $f : (0, \infty) \to \mathbb{R}$ the following two assertions are equivalent:
\begin{enumerate}
\item $f$ is a Bernstein function;

\item $f \ge 0$ and for all $t \ge 0$ the function $e^{-tf}$ is completely monotone.
\end{enumerate}
\end{proposition}

\begin{example}
Some examples of Bernstein functions:
\begin{itemize}
\item $f(\lambda) = a$, for $a \ge 0$;

\item $f(\lambda) = \lambda^\alpha$, for $0 \le \alpha \le 1$;

\item $f(\lambda) = \frac{\lambda}{\lambda + a}$, for $a > 0$;

\item $f(\lambda) = \ln\big(1 + \frac{\lambda}{a}\big)$, for $a > 0$.
\end{itemize}
\end{example}

We now wish to relate Bernstein functions to certain convolution semigroups of measures.
\begin{definition}
Let $(\eta_t)_{t \ge 0}$ be a convolution semigroup of measures on $\mathbb{R}$.  It is said to be \emph{supported by $[0, \infty)$} if $\mathrm{supp}\,\eta_t \subset[0, \infty)$ for all $t \ge 0$.
\end{definition}

\begin{theorem}
Let $f : (0, \infty) \to \mathbb{R}$ be a Bernstein function.  Then there exists a unique convolution semigroup $(\eta_t)_{t \ge 0}$ supported by $[0, \infty)$ such that,
\begin{equation}
\mathcal{L}\eta_t(x) = e^{-tf(x)},\label{SCSBF}
\end{equation}
holds for all $x > 0$ and $t > 0$, where $\mathcal{L}$ denotes the Laplace transform of a measure,
$$
\mathcal{L}\eta_t(x) := \int_0^\infty e^{-xy}\,\eta_t(\mathrm{d}y).
$$

Conversely, for any convolution semigroup $(\eta_t)_{t \ge 0}$ supported by $[0, \infty)$ there exists a unique Bernstein function $f$ such that \eqref{SCSBF} holds.
\end{theorem}

We now turn to why Bernstein functions are so useful for generating examples.
\begin{lemma}
For any Bernstein function $f$ and any continuous negative definite function $\psi : \mathbb{R}^n \to \mathbb{C}$, the function $f \circ \psi$ is also continuous negative definite.
\end{lemma}

So we see that if we have a negative definite function $\psi$, it can be composed with a Bernstein function $f$ to produce another negative definite function.  Now since $f \circ \psi$ is continuous negative definite, there exists a convolution semigroup $(\mu_t^f)_{t \ge 0}$ associated with $f \circ \psi$.
\begin{proposition}\label{FTaSBFP2}
Let $\psi : \mathbb{R}^n \to \mathbb{C}$ be a continuous negative definite function with associated convolution semigroup $(\mu_t)_{t \ge 0}$ on $\mathbb{R}^n$.  Further, let $f$ be a Bernstein function with associated semigroup $(\eta_t)_{t \ge 0}$ supported by $[0, \infty)$.  The convolution semigroup $(\mu_t^f)_{t \ge 0}$ on $\mathbb{R}^n$ associated with the continuous negative definite function $f \circ \psi$ is given by,
\begin{equation}
\int_{\mathbb{R}^n}\phi(x)\,\mu_t^f(\mathrm{d}x) = \int_0^\infty\int_{\mathbb{R}^n}\phi(x)\,\mu_s(\mathrm{d}x)\,\eta_t(\mathrm{d}s), \,\,\,\, \phi \in C_0(\mathbb{R}^n).
\end{equation}
\end{proposition}

\begin{definition}
In the situation of Proposition \ref{FTaSBFP2}, we call the convolution semigroup $(\mu_t^f)_{t \ge 0}$ the \emph{semigroup subordinate (in the sense of Bochner)} to $(\mu_t)_{t \ge 0}$ with respect to $(\eta_t)_{t \ge 0}$.
\end{definition}

We now apply this subordination to operator semigroups.  Let $(T_t)_{t \ge 0}$ be a strongly continuous contraction semigroup of operators on a Banach space $\big(X, \|\cdot\|_X)$ with generator $\big(A, D(A)\big)$.  Furthermore, let $f : (0, \infty) \to \mathbb{R}$ be a Bernstein function and $(\eta_t)_{t \ge 0}$ the associated convolution semigroup on $\mathbb{R}$ supported by $[0, \infty)$.
\begin{theorem}\label{FTaSBFT2}
Let $(T_t)_{t \ge 0}$ and $(\eta_t)_{t \ge 0}$ with corresponding Bernstein function $f$ be as above and define $(T_t^f)_{t \ge 0}$ by,
\begin{equation}
T_t^fu(x) := \int_0^\infty T_su(x)\,\eta_t(\mathrm{d}s), \,\,\,\,\, \text{for } u \in X.
\end{equation}
Then the integral is well-defined and $(T_t^f)_{t \ge 0}$ is a strongly continuous contraction semigroup on $X$.
\end{theorem}

\begin{definition}
Let $(T_t)_{t \ge 0}$ and $(\eta_t)_{t \ge 0}$ with corresponding Bernstein function $f$ be as in Theorem \ref{FTaSBFT2}.  The semigroup $(T_t^f)_{t \ge 0}$ is called \emph{subordinate (in the sense of Bochner)} to $(T_t)_{t \ge 0}$ with respect to $(\eta_t)_{t \ge 0}$ or equivalently with respect to $f$.
\end{definition}


\chapter{Stochastic Processes}\label{Ch.oSP}

\section{Kolmogorov's Existence Theorem}\label{Se.KEThm}

To begin we give a brief overview of Kolmogorov's Existence Theorem and constructing canonical processes.  The reference for this section is \cite{Vol3} which follows mainly from \cite{SPRef}.

\begin{definition}
A topological space $(X, T)$ for which there exists at least one metric $d$ on $X$ such that $(X, d)$ is a complete metric space and $d$ induces the topology $T$ is called a \emph{completely metrizable} space.
\end{definition}

\begin{definition}
A topological space $E$ is called a \emph{Polish space} if it is completely metrizable and has a countable base (i.e. has a countable number of sets that generate all other open sets in its topology).
\end{definition}

\begin{definition}
A \emph{stochastic process} with \emph{state space} $(E, \mathcal{B})$ and \emph{time parameter set} $I$ is a quadruple $\big(\Omega, \mathcal{A}, P, (X_t)_{t \in I}\big)$ where $(\Omega, \mathcal{A}, P)$ is a probability space and $X_t : \Omega \to E$ is for each $t \in I$ a random variable, i.e. a measurable mapping.
\end{definition}

\begin{definition}
Let $(\Omega_1, \mathcal{A}_1), \dots, (\Omega_N, \mathcal{A}_N)$ be a collection of measurable spaces.
\begin{itemize}
\item We denote by,
$$
\bigotimes_{j = 1}^N\mathcal{A}_j := \mathcal{A}_1 \otimes \dots \otimes \mathcal{A}_N,
$$
the \emph{product $\sigma$-algebra} in $\Omega_1 \times \dots \Omega_N$.

\item Let $(\Omega, \mathcal{A})$ be a measure space and for $j = 1, \dots, N$ let $X_j : \Omega \to \Omega_j$ be a random variable.  We define the \emph{product mapping},
$$
\begin{aligned}
\bigotimes_{j = 1}^NX_j := X_1 \otimes \dots \otimes X_N : \Omega & \to \Omega_1 \times \dots \times \Omega_N,\\
\omega & \mapsto \big(X_1(\omega), \dots, X_N(\omega)\big).
\end{aligned}
$$
\end{itemize}
\end{definition}
Firstly we let $I \ne \emptyset$ be an index set and unless specified $\emptyset \ne J \subset K \subset I$ throughout for finite $J, K$.  Moreover let $E$ be a Polish space and $\mathcal{B}$ its Borel $\upsigma$-algebra.  We introduce,
$$
E^J := \prod_{j \in J}E_j, \,\,\, E_j := E, \,\,\,\,\, \text{and } \,\,\,\,\, \mathcal{B}^J := \bigotimes_{j \in J}\mathcal{B}_j, \,\,\, \mathcal{B}_j := \mathcal{B},
$$
as well as the projection $\pi_J^K : E^K \to E^J$ with the properties,
\begin{itemize}
\item $\pi_J^K(e_K) := e_K|_J$, where $e_K \in E^K$;

\item $\pi_J := \pi_J^I$;

\item $\pi_{i_0}^K := \pi_{\{i_0\}}^K$;

\item $\pi_{i_0} := \pi_{\{i_0\}}^I$.
\end{itemize}
Now let $\big(\Omega, \mathcal{A}, P, (X_t)_{t\ge 0}\big)$ be a stochastic process with state space $(E, \mathcal{B})$ and set,
$$
X_J := \bigotimes_{t \in J}X_t,
$$
as well as,
\begin{equation}\label{FoFDD}
P_J := X_J(P).
\end{equation}
So $P_J$ is the joint distribution of the family of $(X_t)_{t \in J}$.

\begin{definition}
We call $(P_J)_{J \in \mathcal{H}(I)}$ (as defined in \eqref{FoFDD}) the \emph{family of finite dimensional distributions} of the process $\big(\Omega, \mathcal{A}, P, (X_t)_{t\ge 0}\big)$.  Here $\mathcal{H}(I)$ denotes the system of all finite non-empty subsets of $I$.
\end{definition}
Since $X_J = \pi_J^K\circ X_K$ we find that,
\begin{equation}\label{P_JProperty}
P_J = \pi_J^K\circ P_K.
\end{equation}

\begin{definition}
If \eqref{P_JProperty} holds for all $J, K \in \mathcal{H}(I)$ then we call $(P_J)_{J \in \mathcal{H}(I)}$ a \emph{projective family of probability measures} over $(E, \mathcal{B})$.
\end{definition}
We now state Kolmogorov's Existence Theorem:

\begin{theorem}
Let $E$ be a Polish space and $I \ne \emptyset$.  For every projective family of probability measures over $(E, \mathcal{B})$ there exists a unique probability measure $P_I$ on $(E^I, \mathcal{B}^I)$ such that,
\begin{equation}
\pi_J(P_I) = P_J,
\end{equation}
holds for all $J \in \mathcal{H}(I)$.
\end{theorem}

\begin{corollary}
Let $E$ be a Polish space and $I \ne \emptyset$ an index set.  Then for every projective family of probability measures $(P_J)_{J \in \mathcal{H}(I)}$ over $(E, \mathcal{B})$ there exists a stochastic process with state space $(E, \mathcal{B})$ and parameter set $I$ such that $(P_J)_{J \in \mathcal{H}(I)}$ is the family of its finite dimensional ditributions.
\end{corollary}

Now that we know that given a family of projective probability measures there exists a stochastic process that has that family as its finite dimensional distributions, we go on to give such a process a name.

\begin{definition}\label{FAaSSPD4}
Set $\Omega := E^I$, $\mathcal{A} := \mathcal{B}^I$ and $P := P_I$ where $\pi_J(P_I) = P_J$ for $J \subset I$.  Further, set $X_t : \Omega \to E$, $\omega \mapsto X_t(\omega) := \omega(t) := \pi_t(\omega)$.  The stochastic process $\big(\Omega, \mathcal{A}, P, (X_t)_{t \ge 0}\big)$ is called the \emph{canonical process} corresponding to the projective family of probability measures $(P_J)_{J \in \mathcal{H}(I)}$.
\end{definition}


\section{Negative Definite Functions to L\'evy Processes}\label{Se.NDFtLP}

To move on we would like to show a way of constructing projective families of probability measures.

\begin{definition}
Let $(\Omega, \mathcal{A})$ and $(\Omega', \mathcal{A}')$ be two measurable spaces.  A mapping $K : \Omega \times \mathcal{A}' \to [0, \infty)$ is called a \emph{kernel} from $(\Omega, \mathcal{A})$ to $(\Omega', \mathcal{A}')$ if and only if,
\begin{itemize}
\item[Ki)] $\omega \mapsto K(\omega, A')$ is measurable for each $A' \in \mathcal{A}'$;

\item[Kii)] $A' \mapsto K(\omega, A')$ is for each $\omega \in \Omega$ a measure on $\mathcal{A}'$.
\end{itemize}
If $K(\omega, \Omega') = 1$ for all $\omega \in \Omega$ we call $K$ a \emph{Markovian kernel}.  If $K(\omega, \Omega') \le 1$ for all $\omega \in \Omega$ we call $K$ a \emph{sub-Markovian kernel}.
\end{definition}
This allows us to introduce another central construct.

\begin{definition}
Let $(E, \mathcal{B})$ be a measurable space and let $(p_t)_{t \ge 0}$ be a family of sub-Markovian kernels from $(E, \mathcal{B})$ to itself.  If for the corresponding operators $(P_t)_{t \ge 0}$ defined as,
$$
P_tu(x) = \int u(y)\,p_t(x, \mathrm{d}y),
$$
it holds that,
$$
P_{s + t} = P_s\circ P_t, \,\,\,\,\, s, t \ge 0,
$$
then $(p_t)_{t \ge 0}$ is called a \emph{semigroup of sub-Markovian kernels}.  If all kernels $p_t$ are Markovian then we call $(p_t)_{t \ge 0}$ a \emph{semigroup of Markovian kernels}.  Moreover if the operator $P_0 = \mathrm{id}$, we call $(p_t)_{t \ge 0}$ a \emph{normal semigroup}.
\end{definition}
\begin{remark}
As a consequence of the condition $P_{s + t} = P_s \circ P_t$, $s, t \ge 0$, we must have that the \emph{Chapman-Kolmogorov equations},
\begin{equation}
p_{s + t}(x, B) = \int p_t(y, B)\,p_s(x, \mathrm{d}y),
\end{equation}
hold for $s, t \ge 0$ and $(x, B) \in E \times \mathcal{B}$.  Furthermore, $(p_t)_{t \ge 0}$ is normal if and only if $p_0(x, \{x\}) = 1$ holds for all $x \in E$.
\end{remark}
The next theorem gives a relation between Markovian kernels and Feller semigroups.  Let $(T_t)_{t \ge 0}$ be a Feller semigroup, then we know that $T_t$ is positive, linear and acts on $C_\infty(\mathbb{R}^n; \mathbb{R})$ for all $t \ge 0$.  Thus by the Riesz Representation Theorem \cite[Theorem 2.3.4]{Vol1}, for each $x \in \mathbb{R}^n$ and $t \ge 0$, there exists a unique Borel measure $p_t(x, \cdot)$ on $\mathbb{R}^n$ such that,
\begin{equation}\label{FSKR}
T_tu(x) = \int_{\mathbb{R}^n}u(y)\,p_t(x, \mathrm{d}y),
\end{equation}
holds for all $t \ge 0$ and $x \in \mathbb{R}^n$, where $\int_{\mathbb{R}^n}u(y)\,p_t(x, \mathrm{d}y) < \infty$, for all $u \in C_\infty(\mathbb{R}^n; \mathbb{R})$.  Using the representation \eqref{FSKR} we can extend $(T_t)_{t \ge 0}$ to $B_b(\mathbb{R}^n)$ by monotone convergence of functions (see \cite{Vol1}) and we get $T_t1 \le 1$.  This leads to,

\begin{theorem}
Let $(T_t)_{t \ge 0}$ be a Feller semigroup on $C_\infty(\mathbb{R}^n; \mathbb{R})$ and define $p_t(x, \mathrm{d}y)$ by \eqref{FSKR}.  Then $p_t(x, \mathrm{d}y)$ is a sub-Markovian kernel for all $t \ge 0$.
\end{theorem}
\begin{remark}
If the extension of $(T_t)_{t \ge 0}$ to the bounded Borel functions is conservative, i.e. $T_t1 = 1$, then $p_t(x, \mathrm{d}y)$ is a Markov kernel for all $t \ge 0$.
\end{remark}

We now state some crucial results.

\begin{theorem}\label{FAaSSPT3}
Let $(E, \mathcal{B})$ be a measure space and $(p_t)_{t \ge 0}$ a semigroup of Markovian kernels on $(E, \mathcal{B})$.  Further let $\mu$ be a probability measure on $(E, \mathcal{B})$.  For $J = \{t_1, \dots, t_m\} \in \mathcal{H}(\mathbb{R}_+)$, $t_1 < \dots < t_m$ and $B \in \mathcal{B}^J$ define,
$$
P_J(B) = \int\dots\int\chi_B(x_1, \dots, x_m)p_{t_m - t_{m - 1}}(x_{m - 1}, \mathrm{d}x_m)\dots p_{t_1}(x_0, \mathrm{d}x_1)\,\mu(\mathrm{d}x_0).
$$
Then $(P_J)_{J \in \mathcal{H}(I)}$ is a projective family of probability measures over $(E, \mathcal{B})$.
\end{theorem}

\begin{corollary}
In Theorem \ref{FAaSSPT3}, if $E$ is a Polish space and $\mathcal{B}$ its Borel $\upsigma$-algebra then the family $(P_J)_{J \in \mathcal{H}(\mathbb{R}_+)}$ forms the family of finite dimensional distributions of a canonical process with state space $(E, \mathcal{B})$.
\end{corollary}
So if we have a Polish space $E$, its Borel $\sigma$-algebra $\mathcal{B}$ and a semigroup of Markovian kernels $(p_t)_{t \ge 0}$ on $(E, \mathcal{B})$.  Then the family $(P_J)_{J \in \mathcal{H}(\mathbb{R}_+)}$ can be constructed as in Theorem \ref{FAaSSPT3}.  However, it depends on the initial distribution $\mu$, thus we denote the corresponding canonical process as $\big(\Omega, \mathcal{A}, P^\mu, (X_t)_{t \ge 0}\big)$ where $\Omega$, $\mathcal{A}$ and $(X_t)_{t \ge 0}$ are defined as in Definition \ref{FAaSSPD4} and $P^\mu$ can be shown to satisfy,
$$
P^\mu\{X_t \in B\} = \int p_t(x, B)\,\mu(\mathrm{d}x),
$$
for all $B \in \mathcal{A}, x \in \Omega$.  We call $\mu$ the initial distribution of the process and we define $P^x := P^{\varepsilon_x}$.

Continuing, we introduce Markov processes.

\begin{definition}
Let $(X_t)_{t \ge 0}$ be a stochastic process on $(\Omega, \mathcal{A}, P)$ with state space $(E, \mathcal{B})$ and suppose that $(X_t)_{t \ge 0}$ is adapted to the filtration $(\mathcal{F}_t)_{t \ge 0}$.  We call $(X_t)_{t \ge 0}$ a \emph{Markov process} with respect to $(\mathcal{F}_t)_{t \ge 0}$ if and only if,
\begin{equation}\label{EMP}
P(X_t \in B\,|\,\mathcal{F}_s) = P(X_t \in B\,|\, X_s),
\end{equation}
holds for all $0 \le s < t$ and $B \in \mathcal{B}$.  Here, $P(X_t \in B\,|\, \mathcal{F}_s) = \mathbb{E}(\chi_{\{X_t \in B\}}\,|\, X_s)$ so \eqref{EMP} is equivalent to,
$$
\mathbb{E}(\chi_{\{X_t \in B\}}\,|\, \mathcal{F}_s) = \mathbb{E}(\chi_{\{X_t \in B\}}\,|\, X_s).
$$
We call \eqref{EMP} the \emph{elementary Markov property}.
\end{definition}

\begin{theorem}
Let $\big(\Omega, \mathcal{A}, P^\mu, (X_t)_{t \ge 0}\big)$ be a stochastic process with Polish state space $(E, \mathcal{B})$ and finite dimensional distributions deriving from the normal semigroup of Markovian kernels $(p_t)_{t \ge 0}$ and initial distribution $\mu$.  Then $(X_t)_{t \ge 0}$ is a Markov process with respect to the canonical filtration $(\mathcal{F}_t^0)_{t \ge 0}$.
\end{theorem}
This corollary links Feller semigroups to Markov processes.
\begin{corollary}
Every canonical process constructed by starting with a Feller semigroup $(T_t)_{t \ge 0}$ on $C_\infty(\mathbb{R}^n)$ and a given initial distribution $\mu \in \mathcal{M}_b^1(\mathbb{R}^n)$ is a Markov process with respect to the canonical filtration.
\end{corollary}
Having defined a Markov process, we now define a universal Markov process.

\begin{definition}
Let $(E, \mathcal{B})$ be a measurable space.  We call \\$\big(\Omega, \mathcal{A}, P^x, (X_t)_{t \ge 0}\big)_{x \in E}$ a \emph{universal Markov process} with state space $(E, \mathcal{B})$ if the following conditions hold:
\begin{itemize}

\item[Mi)] $\big(\Omega, \mathcal{A}, P^x, (X_t)_{t \ge 0}\big)$ is for all $x \in E$ a stochastic process with state space $(E, \mathcal{B})$;

\item[Mii)] For all $A \in \mathcal{A}$ the function $x \mapsto P^x(A)$ is $\mathcal{B}$-measurable;

\item[Miii)] $P^x(X_0 = x) = 1$ for all $x \in E$;

\item[Miv)] For all $s, t \ge 0$, $x \in E$ and $B \in \mathcal{B}$, it holds that,
$$
P^x(X_{s + t} \in B\,|\, \mathcal{F}_s^0) = P^{X_s}(X_t \in B), \,\,\,\,\, P^x \text{ - a.s.}
$$

\end{itemize}
Property Miv) is called the \emph{universal Markov property} with respect to the canonical filtration $(\mathcal{F}_t^0)_{t \ge 0}$.
\end{definition}

\begin{theorem}
Let $(p_t)_{t \ge 0}$ be a normal semigroup of Markovian kernels on a Polish state space $(E, \mathcal{B})$.  Then there exists a universal Markov process $\big(\Omega, \mathcal{A}, P^x, (X_t)_{t \ge 0}\big)_{x \in E}$ with state space $(E, \mathcal{B})$ such that for all $t \ge 0$, $x \in E$ and $B \in \mathcal{B}$ we have,
$$
p_t(x, B) = P^x(X_t \in B).
$$
\end{theorem}

\begin{corollary}
Let $(T_t)_{t \ge 0}$ be a conservative Feller semigroup on\\$C_\infty(\mathbb{R}^n)$.  Then the family of canonical processes $\big(\Omega, \mathcal{A}, P^x, (X_t)_{t \ge 0}\big)_{x \in E}$\\associated with $(T_t)_{t \ge 0}$ form a universal Markov process.  It follows that,
$$
T_tu(x) = \mathbb{E}^x(u\circ X_t),
$$
where $\mathbb{E}^x$ is the expectation with respect to $P^x$.
\end{corollary}
In other words, starting with a conservative Feller semigroup we can construct a universal Markov process.

\begin{definition}
A universal Markov process $\big((X_t)_{t \ge 0}, P^x, (\mathcal{F}_t)_{t \ge 0}\big)_{x \in E}$ with locally compact Polish state space $(E, \mathcal{B})$ is called a \emph{Feller process} if $(T_t)_{t \ge 0}$, $T_t := P_t|_{C_\infty(E)}$ is a Feller semigroup.
\end{definition}

We need a brief digression before continuing to introduce c\`adl\`ag processes.  Let $\big(\Omega, \mathcal{A}, P, (X_t)_{t \ge 0}\big)$ be a stochastic process with Polish state space $(E, \mathcal{B})$.  For each $\omega \in \Omega$ we can consider the mapping,
$$
X_\cdot(\omega) : \mathbb{R}_+ \to E, \,\,\, t \mapsto X_t(\omega),
$$
which is called a path or sample path of the process.

\begin{definition}
Let $(X_t)_{t \ge 0}$ be a stochastic process with Polish state space.
\begin{itemize}
\item[A)] The process $(X_t)_{t \ge 0}$ is called \emph{a.s. continuous} if its paths are a.s. continuous.

\item[B)] The process $(X_t)_{t \ge 0}$ is called \emph{a.s. right continuous} if a.s. its paths are right continuous.

\item[C)] We call $(X_t)_{t \ge 0}$ a \emph{c\`adl\`ag process} if a.s. all paths of $(X_t)_{t \ge 0}$ are continuous from the right and have limits from the left.
\end{itemize}
\end{definition}
We now state some results for Feller processes.

\begin{definition}
Let $\big(\Omega, \mathcal{A}, P, (X_t)_{t \ge 0}\big)$ and $\big(\Omega, \mathcal{A}, P, (Y_t)_{t \ge 0}\big)$ be two sto\-chastic processes with state space $(E, \mathcal{B})$.  If for all $t \ge 0$,
$$
P(X_t = Y_t) = 1,
$$
holds then we call $(X_t)_{t \ge 0}$ a \emph{modification} of $(Y_t)_{t \ge 0}$.
\end{definition}

\begin{theorem}
A Feller process $\big((X_t)_{t \ge 0}, P^x\big)_{x \in \mathbb{R}^n}$ has a c\`adl\`ag modification.
\end{theorem}

\begin{corollary}
Let $(X_t)_{t \ge 0}$ be the canonical universal Markov process with respect to the canonical filtration associated with the Feller semigroup $(T_t)_{t \ge 0}$, $T_t : C_\infty(\mathbb{R}^n) \to C_\infty(\mathbb{R}^n)$.  Then it admits a c\`adl\`ag modifcation.
\end{corollary}

\begin{theorem}
Every Feller process $\big((X_t)_{t \ge 0}, P^x\big)_{x \in \mathbb{R}^n}$ is stochastically continuous, i.e.,
$$
\lim_{s \to t}P^x\big(|X_s - X_t| \ge \varepsilon\big) = 0.
$$
\end{theorem}

Finally we are in a position to consider L\'evy processes.

\begin{definition}
Let $\big((X_t)_{t \ge 0}, P\big)$ be a stochastic process with state space $(\mathbb{R}^n, \mathcal{B}^{(n)})$.
\begin{itemize}
\item[A)] We say that $(X_t)_{t \ge 0}$ has \emph{independent increments} if for all $0 \le s < t$ the random variable $X_t - X_s$ is independent of $\mathcal{F}_s$.

\item[B)] The process $(X_t)_{t \ge 0}$ is said to have \emph{stationary increments} if for all $0 \le s < t$ it follows that $P_{X_t - X_s} = P_{X_{t - s}}$.
\end{itemize}
\end{definition}

\begin{theorem}\label{FAaSSPT6}
The process $\big((X_t)_{t \ge 0}, P^\mu\big)$ constructed from the Feller semigroup associated to the continuous negative definite function $\psi$ and the initial distribution $\mu$ has stationary and independent increments.
\end{theorem}

We now define what a L\'evy process is \cite{Sato}.

\begin{definition}
A stochastic process $\big(\Omega, \mathcal{A}, P, (X_t)_{t \ge 0}, (\mathcal{F}_t)_{t \ge 0}\big)$ with\\
state space $(\mathbb{R}^n, \mathcal{B}^{(n)})$ is called a \emph{L\'evy process} if and only if:
\begin{itemize}
\item[Li)] $(X_t)_{t \ge 0}$ is $\mathcal{F}_t$-adapted;

\item[Lii)] $(X_t)_{t \ge 0}$ has independent increments;

\item[Liii)] $(X_t)_{t \ge 0}$ has stationary increments;

\item[Liv)] $(X_t)_{t \ge 0}$ is stochastically continuous;

\item[Lv)] $(X_t)_{t \ge 0}$ is c\`adl\`ag.
\end{itemize}
A stochastic process satisfying Li) - Liv) is called a \emph{L\'evy process in law}.
\end{definition}

We can now state our final result.

\begin{corollary}\label{FAaSSPC6}
The process in Theorem \ref{FAaSSPT6} is a L\'evy process in law.
\end{corollary}

We summarise some of the key points of this chapter with an example.  Let $\psi : \mathbb{R}^n \to \mathbb{C}$ be continuous and negative definite and suppose $\psi(0) = 0$, this entails that we are dealing with probability measures.  Then we have the corresponding unique convolution semigroup $(\mu_t)_{t \ge 0}$ on $\mathbb{R}^n$ defined by,
$$
\hat{\mu}_t(\xi) = (2\pi)^{-n/2}e^{-t\psi(\xi)}.
$$
It can be shown that $(p_t)_{t \ge 0}$ defined as,
$$
p_t(x, B) = \mu_t(B - x), \,\,\, x \in \mathbb{R}^n, \,\,\, B \in \mathcal{B}^{(n)},
$$
is a normal semigroup of Markovian kernels from $\big(\mathbb{R}^n, \mathcal{B}^{(n)}\big)$ to itself with corresponding operators,
$$
P_tu(x) = \int_{\mathbb{R}^n}u(x - y)\,\mu_t(\mathrm{d}y), \,\,\, u \in C_\infty(\mathbb{R}^n).
$$
By Example \ref{FTaSOSE1}, $(P_t)_{t \ge 0}$ forms a Feller semigroup on $C_\infty(\mathbb{R}^n)$.  We can now construct a family of probability measures as in Theorem \ref{FAaSSPT3} given an initial distribution $\varepsilon_x$, from here we can construct the canonical process $\big(\Omega, \mathcal{A}, P^x, (X_t)_{t \ge 0}\big)$.  By Corollary \ref{FAaSSPC6}, the canonical process is a L\'evy process in law.  Our L\'evy process is thus completely determined by a negative definite function $\psi$ which we call the \emph{characteristic exponent} of the L\'evy process.

Assume $e^{-t\psi} \in \mathcal{L}^1\big(\mathbb{R}^n, \lambda^{(n)}\big)$ for all $t > 0$, then if we apply the convolution theorem to our operator $P_t$ we get,
$$
(P_tu)^\wedge(\xi) = (2\pi)^{n/2}\hat{u}(\xi)\hat{\mu}_t(\xi),
$$
or,
$$
P_tu(x) = (2\pi)^{-n/2}\int_{\mathbb{R}^n}e^{ix\cdot\xi}e^{-t\psi(\xi)}\hat{u}(\xi)\,\mathrm{d}\xi,
$$
and by the definition of the Fourier transform,
$$
\begin{aligned}
P_tu(x) & = (2\pi)^{-n/2}\int_{\mathbb{R}^n}e^{ix\cdot\xi}e^{-t\psi(\xi)}(2\pi)^{-n/2}\int_{\mathbb{R}^n}e^{-i\xi\cdot y}u(y)\,\mathrm{d}y\,\mathrm{d}\xi\\
& = \int_{\mathbb{R}^n}u(y)p_t(x - y)\,\mathrm{d}y,
\end{aligned}
$$
where,
\begin{equation}\label{TDfLP}
p_t(x) := (2\pi)^{-n}\int_{\mathbb{R}^n}e^{ix\cdot\xi}e^{-t\psi(\xi)}\,\mathrm{d}\xi.
\end{equation}
So we see that $p_t(x, B) = p_t(x)\lambda^{(n)}(B)$, or in other words our Markovian kernel $p_t(x, \cdot)$ has a density $p_t(x)$ with respect to the Lebesgue measure.  We call the family $(p_t)_{t > 0}$ the \emph{transition densities} (or \emph{probability densities}) of our L\'evy process.

\begin{example}
Some examples of transition densities.
\begin{center}
\begin{tabular}{c | c | c}
L\'evy Process & $\psi(\xi)$ & $p_t(x)$\\
\hline
Cauchy & $|\xi|$ & $\Gamma\big(\frac{n + 1}{2}\big)\frac{t}{(|x|^2 + t^2)^\frac{n + 1}{2}}$\\
Gaussian - $N(0, t)$ & $\frac{1}{2}|\xi|^2$ & $(2\pi t)^{-n/2}e^{-|x|^2/2t}$\\
Symmetric Meixner & $\ln\cosh\xi$ & $\frac{2^{t - 1}}{\pi\Gamma(t)}\big|\Gamma\big(\frac{t + ix}{2}\big)\big|^2$\\
Relativistic Hamiltonian & $\sqrt{m^2 + \xi^2} - m$ & $\frac{mte^{mt}}{\pi}\frac{K_1\big(m\sqrt{t^2 + |x|^2}\big)}{\sqrt{t^2 + |x|^2}}$
\end{tabular}
\end{center}

\end{example}
\begin{remark}
Here $K_1$ is the modified Bessel function of the second kind with index $1$.
\end{remark}

Finally we introduce additive processes \cite{Sato}.
\begin{definition}
A stochastic process $\big(\Omega, \mathcal{A}, P, (X_t)_{t \ge 0}, (\mathcal{F}_t)_{t \ge 0}\big)$ with\\
state space $(\mathbb{R}^n, \mathcal{B}^{(n)})$ is called an \emph{additive process} if and only if:
\begin{itemize}
\item[Ai)] $(X_t)_{t \ge 0}$ is $\mathcal{F}_t$-adapted;

\item[Aii)] $(X_t)_{t \ge 0}$ has independent increments;

\item[Aiii)] $(X_t)_{t \ge 0}$ is stochastically continuous;

\item[Aiv)] $(X_t)_{t \ge 0}$ is c\`adl\`ag.
\end{itemize}
A stochastic process satisfying Ai) - Aiii) is called an \emph{additive process in law}.
\end{definition}
It is noteworthy that a L\'evy process is a special case of an additive process.  We will have a more detailed discussion of additive processes in section \ref{Se.FSoEEaAP}.


\section{Metrics Related to L\'evy Processes}\label{MRtLP}

In this section we present results and ideas from \cite{Paper}, general references for metric measure spaces are \cite{22in8} and \cite{45in8}.
\begin{definition}
A \emph{metric measure space} is a triple $(X, d, \mu)$ where $(X, d)$ is a metric space and $\mu$ is a measure on the Borel sets of the space $X$.
\end{definition}

We are interested in looking at metric measure spaces generated by negative definite functions.  Let $\psi : \mathbb{R}^n \to \mathbb{C}$ be a continuous negative definite function, we know that,
$$
\sqrt{|\psi(\xi + \eta)|} \le \sqrt{|\psi(\xi)|} + \sqrt{|\psi(\eta)|},
$$
and,
$$
|\psi(\xi)| \le c_\psi\big(1 + |\xi|^2\big),
$$
for some constant $c_\psi$.  Since $\psi(-\xi) = \overbar{\psi(\xi)}$, the mapping $\xi \mapsto |\psi(\xi)|$ is even and so it is easy to see that every continuous, non-periodic negative definite function with $\psi(0) = 0$ induces a metric on $\mathbb{R}^n$ by,
$$
d_\psi : \mathbb{R}^n \times \mathbb{R}^n \to [0, \infty), \,\,\,\,\, d_\psi(\xi, \eta) := \sqrt{|\psi(\xi - \eta)|}.
$$
We see that,
$$
d_\psi(\xi + \zeta, \eta + \zeta) = \sqrt{|\psi(\xi + \zeta - \eta - \zeta)|} = \sqrt{|\psi(\xi + \eta)|} = d_\psi(\xi, \eta),
$$
i.e. the metric $d_\psi$ is invariant under translation and so the metric measure space we will consider $\big(\mathbb{R}^n, d_\psi, \lambda^{(n)}\big)$ since the Lebesgue measure $\lambda^{(n)}$ on $\mathbb{R}^n$ is the canonical choice.  However, at the moment the relation of the Borel sets with respect to the Euclidean topology and the Borel sets with respect to the topology induced by $d_\psi$ is not clear, but $\lambda^{(n)}$ is defined on the Borel sets with respect to the Euclidean metric.  To clarify the situation we introduce the notation,
$$
B^{d_\psi}(\xi, r) := \{\eta \in \mathbb{R}^n : d_\psi(\xi, \eta) < r\} = \{\eta \in \mathbb{R}^n : |\psi(\xi - \eta)| < r^2\},
$$
$$
K^{d_\psi}(\xi, r) := \{\eta \in \mathbb{R}^n : d_\psi(\xi, \eta) \le r\} = \{\eta \in \mathbb{R}^n : |\psi(\xi - \eta)| \le r^2\},
$$
which are the open and closed balls (respectively) in our metric measure space with centre $\xi$ and radius $r$.  We now have,
\begin{lemma}
Let $\psi : \mathbb{R}^n \to \mathbb{C}$ be a continuous negative definite function.  Then for $r > 0$, the closed ball $K^{d_\psi}(0, r)$ is bounded in the Euclidean topology if and only if $r^2 < \lim\inf_{|\xi| \to \infty}|\psi(\xi)|$.  Moreover, $d_\psi$ generates on $\mathbb{R}^n$ the Euclidean topology if and only if $\lim\inf_{|\xi| \to \infty}|\psi(\xi)| > 0$.
\end{lemma}

From now on we want to assume $d_\psi$ generates the Euclidean topology and so we introduce the following class of functions.
\begin{definition}
Let $\psi : \mathbb{R}^n \to \mathbb{C}$ be a non-periodic, locally bounded negative definite function with $\psi(0) = 0$.  We call $\psi$ \emph{metric generating} on $\mathbb{R}^n$ if the metric $d_\psi$ generates the Euclidean topology on $\mathbb{R}^n$.  The set of all continuous metric generating negative definite functions on $\mathbb{R}^n$ is denoted by $\mathcal{MCN}(\mathbb{R}^n)$.
\end{definition}
It follows that a continuous negative definite function $\psi : \mathbb{R}^n \to \mathbb{C}$ is in $\mathcal{MCN}(\mathbb{R}^n)$ if and only if $\lim\inf_{|\xi| \to \infty}|\psi(\xi)| > 0$.  We now restrict ourselves to the study of $\big(\mathbb{R}^n, d_\psi, \lambda^{(n)}\big)$ where $\psi \in \mathcal{MCN}(\mathbb{R}^n)$, here the notion of volume doubling is of importance.
\begin{definition}
Let $(X, d, \mu)$ be a metric measure space.  We say that $(X, d, \mu)$ or just $\mu$ has the \emph{volume doubling property} if there exists a constant $c$ such that,
\begin{equation}\label{VDP}
\mu\big(B^d(x, 2r)\big) \le c\mu\big(B^d(x, r)\big),
\end{equation}
holds for all metric balls $B^d(x, r) \subset X$.  If \eqref{VDP} only holds for all balls with radii $r < \rho$ for some fixed $\rho > 0$, we say that $(X, d, \mu)$ or just $\mu$ is \emph{locally volume doubling}.
\end{definition}
\begin{remark}
If $(X, d, \mu)$ has volume doubling with constant $c > 1$, then it follows that for every $R \ge 1$ that,
$$
\mu\big(B^d(x, R)\big) \le c^{\log_2R}\mu\big(B^d(x, 1)\big) = R^{\log_2c}\mu\big(B^d(x, 1)\big),
$$
i.e. volume doubling entails that balls have at most power growth of their volume.
\end{remark}

Let $\psi \in \mathcal{MCN}(\mathbb{R}^n)$, we know that for all negative definite functions we have an asociated L\'evy process with transition densities $(p_t)_{t > 0}$ given by,
$$
p_t(x) = (2\pi)^{-n}\int_{\mathbb{R}^n}e^{ix\cdot\xi}e^{-t\psi(\xi)}\,\mathrm{d}\xi, 
$$
if $e^{-t\psi} \in \mathcal{L}^1\big(\mathbb{R}^n, \lambda^{(n)}\big)$.  For the rest of this section we assume $\psi$ to be real-valued and we see that using the geometry induced by the metric measure space $\big(\mathbb{R}^n, d_\psi, \lambda^{(n)}\big)$, we can express $p_t(0)$ in geometric terms.  We define $d_{\psi, t} = \sqrt{t}d_\psi$.
\begin{theorem}\label{FAaSGRtLPT1}
Let $\psi \in \mathcal{MCN}(\mathbb{R}^n)$ and assume that $e^{-t\psi} \in \mathcal{L}^1\big(\mathbb{R}^n, \lambda^{(n)}\big)$.  Then,
\begin{equation}
p_t(0) = (2\pi)^{-n}\int_0^\infty\lambda^{(n)}\big(B^{d_{\psi, t}}(0, \sqrt{r})\big)e^{-r}\,\mathrm{d}r.
\end{equation}
If the metric measure space $\big(\mathbb{R}^n, d_\psi, \lambda^{(n)}\big)$ has the volume doubling property, then $e^{-t\psi} \in \mathcal{L}^1\big(\mathbb{R}^n, \lambda^{(n)}\big)$ and,
\begin{equation}
p_t(0) \asymp \lambda^{(n)}\big(B^{d_{\psi, t}}(0, 1)\big),
\end{equation}
for all $t > 0$.
\end{theorem}

Having seen that we can understand the diagonal behaviour of $p_t$ in geometric terms, a natural question is to see if the off-diagonal behaviour can be understood in geometric terms also.  Following \cite{Paper} we state,
\begin{conjecture}\label{FAaSGRtLPConj1}
Let $(p_t)_{t > 0}$ be the transition densities of some symmetric L\'evy process with characteristic exponent $\psi \in \mathcal{MCN}(\mathbb{R}^n)$.  There exists a mapping $\delta_\psi : (0, \infty) \times \mathbb{R}^n \times \mathbb{R}^n \to \mathbb{R}$ such that for every $t \in (0, \infty)$ the map $\delta_\psi(t, \cdot, \cdot) : \mathbb{R}^n \times \mathbb{R}^n \to \mathbb{R}$ is a (translation invariant) metric such that,
\begin{equation}\label{PaperConjecture}
p_t(x) = p_t(0)e^{-\delta_\psi^2(t, x, 0)}.
\end{equation}
\end{conjecture}
In short, Conjecture \ref{FAaSGRtLPConj1} states that the transition densities of symmetric L\'evy processes can be fully understood in terms of two one-parameter family of metrics $(d_{\psi, t})_{t > 0}$ and $\big(\delta_\psi(t, \cdot, \cdot)\big)_{t > 0}$.  Reasons for suspecting Conjecture \ref{FAaSGRtLPConj1} is true is given in \cite{Paper} and will not be explored here, however later we will discuss the weaker form of \eqref{PaperConjecture} namely,
\begin{equation}
p_t(x) \simeq p_t(0)e^{-\delta_\psi^2(t, x, 0)}.
\end{equation}
 One piece of evidence we do state however involves subordination using Bernstein functions which we reference later.
\begin{theorem}\label{TDSGMeT3}
Let $(p_t)_{t > 0}$ be the transition densities of a L\'evy process in $\mathbb{R}$ such that the characteristic functions are of the form $p_t(x) = \mathrm{F}_{\xi \mapsto x}^{-1}e^{-tf(|\xi|)}$ with some Bernstein function $f$ satisfying $f(0) = 0$ and $e^{-tf} \in \mathcal{L}^1\big((0, \infty), \lambda\big)$ for all $t > 0$.  Then there exists, for each $t > 0$, a complete Bernstein function $g_t$ such that,
$$
\frac{p_t(x)}{p_t(0)} = e^{-g_t(|x|^2)}.
$$
\end{theorem}

\newpage

\part{Stochastic Processes with Time Dependent Generators and Some Properties of Their Transition Densities}

\newpage\null

\chapter{Time Dependent Generators and Their Fundamental Solutions I: Spatially Homogeneous Symbols}

\section{Some Motivating Examples}\label{Se.SMEg}

We begin by considering a symmetric L\'evy process $(X_t)_{t \ge 0}$ on $\mathbb{R}^n$ with continuous negative definite characteristic exponent $\psi : \mathbb{R}^n \to \mathbb{R}$ which gives the convolution semigroup of measures $(\mu_t)_{t \ge 0}$ defined by $\hat{\mu}_t(\xi) = (2\pi)^{-n/2}e^{-t\psi(\xi)}$.  Assuming $e^{-t\psi} \in \mathcal{L}^1\big(\mathbb{R}^n, \lambda^{(n)}\big)$, for all $t > 0$, then it is well-known that the transition densities $(p_t)_{t > 0}$ of $(X_t)_{t \ge 0}$ can be written as,
\begin{equation}\label{LPTransitionDensities}
p_t(x) = \mathrm{F}^{-1}\hat{\mu}_t(x)= (2\pi)^{-n}\int_{\mathbb{R}^n}e^{ix\cdot\xi}e^{-t\psi(\xi)}\,\mathrm{d}\xi.
\end{equation}
Taking an idea from \cite{Paper} we note that,
$$
p_t(0) = (2\pi)^{-n}\int_{\mathbb{R}^n}e^{-t\psi(\xi)}\,\mathrm{d}\xi \,\,\,\,\, \Leftrightarrow \,\,\,\,\, \int_{\mathbb{R}^n}\frac{e^{-t\psi(\xi)}}{(2\pi)^np_t(0)}\,\mathrm{d}\xi = 1,
$$
or in other words,
\begin{equation}
\rho_t(\mathrm{d}x) := \frac{e^{-t\psi(x)}}{(2\pi)^np_t(0)}\,\lambda^{(n)}(\mathrm{d}x),\label{DPPM}
\end{equation}
is a family of probability measures for all $t > 0$.  An interesting question is when does $(\rho_t)_{t > 0}$ define a process?  If such a process exists then we will call it the ``dual'' process of $(X_t)_{t \ge 0}$, in general we will call $(\rho_t)_{t \ge 0}$ the dual family of probability measures associated with $(\mu_t)_{t \ge 0}$.

\begin{example}\label{DofCPeg}
Consider the Cauchy process in $\mathbb{R}$ which has characteristic exponent $\psi_C(\xi) = |\xi|$ and transition densities $(p_t^C)_{t > 0}$ given by,
$$
p_t^C(x) = \frac{1}{\pi}\frac{t}{x^2 + t^2}.
$$
By direct calculation,
$$
\frac{e^{-t\psi_C(x)}}{(2\pi)p_t^C(0)} = \frac{t}{2}e^{-t|x|},
$$
which is the transition density of the Laplace process at time $t > 0$, i.e. the dual process of the Cauchy process is the Laplace process.
\end{example}

A problem is highlighted when looking to the Gaussian process in $\mathbb{R}^n$ which has characteristic exponent $\psi_G(\xi) = \frac{1}{2}|\xi|^2$ and transition densities $(p_t^G)_{t > 0}$ given by,
$$
p_t^G(x) = \frac{1}{(2\pi t)^{n/2}}e^{-\frac{|x|^2}{2t}}.
$$
If we calculate the dual family of probability measures,
$$
\rho_t^G(\mathrm{d}x) = \frac{e^{-t\psi_G(x)}}{(2\pi)^np_t^G(0)}\,\lambda^{(n)}(\mathrm{d}x) = \bigg(\frac{t}{2\pi}\bigg)^{n/2}e^{-t\frac{|x|^2}{2}}\,\lambda^{(n)}(\mathrm{d}x),
$$
we get something very similar to the Gaussian we started with.  However, we would like $\rho_t^G \to \varepsilon_0$ as $t \to 0$ which does not happen since,
$$
\bigg(\frac{t}{2\pi}\bigg)^{n/2}e^{-t\frac{|x|^2}{2}} \to 0 \,\,\,\,\, \text{as} \,\,\,\,\, t \to 0.
$$
A quick observation however shows that if we replace $t$ with $1/t$ then we get our original Gaussian process which leads to the following result.

\begin{proposition}\label{SharedPropWithNiels}
Let $\psi : \mathbb{R}^n \to \mathbb{R}$ be the characteristic exponent of a symmetric L\'evy process.  Then $\nu_t := \rho_{1/t} \to \varepsilon_0$ weakly as $t \to 0$, where $\rho_t$ is defined as in \eqref{DPPM}.
\end{proposition}
\begin{proof}
Let $\varphi \in C_0^\infty(\mathbb{R}^n)$ then by Plancherel's Theorem,
$$
\begin{aligned}
& \int_{\mathbb{R}^n}\varphi(x - y)\,\nu_t(\mathrm{d}y)\\
= & \int_{\mathbb{R}^n}\frac{e^{-\frac{1}{t}\psi(y)}}{(2\pi)^np_\frac{1}{t}(0)}\varphi(x - y)\,\mathrm{d}y\\
= & \, \frac{1}{(2\pi)^{n/2}p_\frac{1}{t}(0)}\int_{\mathbb{R}^n}\mathrm{F}^{-1}\big((2\pi)^{-n/2}e^{-\frac{1}{t}\psi(\cdot)}\big)(\xi)\,\overbar{\mathrm{F}^{-1}\big(\varphi(x - \cdot)\big)(\xi)}\,\mathrm{d}\xi.
\end{aligned}
$$
Next by substitution,
$$
\begin{aligned}
\overbar{\mathrm{F}^{-1}\big(\varphi(x - \cdot)\big)(\xi)} & = (2\pi)^{-n/2}\overbar{\int_{\mathbb{R}^n}e^{i\xi\cdot\eta}\varphi(x - \eta)\,\mathrm{d}\eta}\\
& = (2\pi)^{-n/2}\int_{\mathbb{R}^n}e^{-i\xi\cdot(x - z)}\overbar{\varphi(z)}\,\mathrm{d}z\\
& = e^{-i\xi\cdot x}\mathrm{F}^{-1}\varphi(\xi),
\end{aligned}
$$
and so by \cite[Theorem 14]{DProofPaper} which states that if $(\pi_t)_{t > 0}$ are the transition densities of a L\'evy process in $\mathbb{R}^n$ then,
$$
\lim_{t \to \infty}\frac{\pi_t(x)}{\pi_t(0)} = 1, \,\,\,\,\, \text{for all} \,\, x \in \mathbb{R}^n,
$$
we get,
$$
\begin{aligned}
\lim_{t \to 0}\int_{\mathbb{R}^n}\varphi(x - y)\,\nu_t(\mathrm{d}y) & = (2\pi)^{-n/2}\int_{\mathbb{R}^n}\lim_{t \to 0}\frac{p_\frac{1}{t}(\xi)}{p_\frac{1}{t}(0)}e^{-ix\cdot\xi}\mathrm{F}^{-1}\varphi(\xi)\,\mathrm{d}\xi\\
& = (2\pi)^{-n/2}\int_{\mathbb{R}^n}e^{-ix\cdot\xi}\mathrm{F}^{-1}\varphi(\xi)\,\mathrm{d}\xi\\
& = \varphi(x).
\end{aligned}
$$
Since $C_0^\infty(\mathbb{R}^n)$ is dense in $C_\infty(\mathbb{R}^n)$ we may extend this result by continuiuty for all $\varphi \in C_\infty(\mathbb{R}^n)$.  However $C_0(\mathbb{R}^n) \subset C_\infty(\mathbb{R}^n)$ and so $\nu_t \to \varepsilon_0$ vaguely as $t \to 0$.  The result now follows from Theorem \ref{VCimpliesWC}.
\end{proof}

Next we define a family of operators $(S_t)_{t \ge 0}$ by,
\begin{equation}
S_tu(x) := \big(u \ast \nu_t\big)(x) = \int_{\mathbb{R}^n}u(x - y)\,\nu_t(\mathrm{d}y),
\end{equation}
which we note is in general not a semigroup.  Using the convolution theorem we may write,
$$
(S_tu)^\wedge(\xi) = (2\pi)^{n/2}\hat{u}(\xi)\hat{\nu}_t(\xi),
$$
or,
$$
S_tu(x) = \int_{\mathbb{R}^n}e^{ix\cdot\xi}\hat{\nu}_t(\xi)\hat{u}(\xi)\,\mathrm{d}\xi.
$$
Since $\psi$ is real-valued and negative definite then $\psi(-x) = \overbar{\psi(x)} = \psi(x)$ and we can write,
$$
\begin{aligned}
\mathrm{F}\big((2\pi)^{-n/2}e^{-t\psi(\cdot)}\big)(\xi) & = (2\pi)^{-n}\int_{\mathbb{R}^n}e^{-i\xi\cdot x}e^{-t\psi(x)}\,\mathrm{d}x\\
& = (2\pi)^{-n}\int_{\mathbb{R}^n}e^{i\xi\cdot y}e^{-t\psi(y)}\,\mathrm{d}y\\
& = p_t(\xi),
\end{aligned}
$$
and so,
$$
\hat{\nu}_t(\xi) = \mathrm{F}\frac{e^{-\frac{1}{t}\psi(\cdot)}}{(2\pi)^np_\frac{1}{t}(0)}(\xi) = (2\pi)^{-n/2}\frac{p_\frac{1}{t}(\xi)}{p_\frac{1}{t}(0)},
$$
thus,
\begin{equation}
S_tu(x) = (2\pi)^{-n/2}\int_{\mathbb{R}^n}e^{ix\cdot\xi}\sigma_t(\xi)\hat{u}\,\mathrm{d}\xi,
\end{equation}
where,
\begin{equation}
\sigma_t(\xi) := \frac{p_\frac{1}{t}(\xi)}{p_\frac{1}{t}(0)}.
\end{equation}
We would like to calculate,
$$
\frac{\partial}{\partial t}S_tu(x) = (2\pi)^{-n/2}\int_{\mathbb{R}^n}e^{ix\cdot\xi}\frac{\partial}{\partial t}\sigma_t(\xi)\hat{u}(\xi)\,\mathrm{d}\xi,
$$
so we compute,
$$
\begin{aligned}
\frac{\partial}{\partial t}\sigma_t(\xi) & = \frac{\partial}{\partial t}\frac{p_\frac{1}{t}(\xi)}{p_\frac{1}{t}(0)}\\
& = \frac{1}{\big(p_\frac{1}{t}(0)\big)^2}\bigg(p_\frac{1}{t}(0)\frac{\partial}{\partial t}p_\frac{1}{t}(\xi) - p_\frac{1}{t}(\xi)\frac{\partial}{\partial t}p_\frac{1}{t}(0)\bigg)\\
& = \frac{p_\frac{1}{t}(\xi)}{p_\frac{1}{t}(0)}\bigg(\frac{\frac{\partial}{\partial t}p_\frac{1}{t}(\xi)}{p_\frac{1}{t}(\xi)} - \frac{\frac{\partial}{\partial t}p_\frac{1}{t}(0)}{p_\frac{1}{t}(0)}\bigg)\\
& = \sigma_t(\xi)\frac{\partial}{\partial t}\ln\frac{p_\frac{1}{t}(\xi)}{p_\frac{1}{t}(0)}.
\end{aligned}
$$
Hence if we define,
\begin{equation}
q(t, \xi) := -\frac{\partial}{\partial t}\ln\frac{p_\frac{1}{t}(\xi)}{p_\frac{1}{t}(0)},
\end{equation}
then,
$$
\frac{\partial}{\partial t}S_tu(x) = -(2\pi)^{-n/2}\int_{\mathbb{R}^n}e^{ix\cdot\xi}q(t, \xi)(S_tu)^\wedge(\xi)\,\mathrm{d}\xi = -q(t, D)S_tu(x),
$$
where $q(t, D)$ is an operator on $\mathcal{S}(\mathbb{R}^n)$ for all $t \ge 0$ given by,
\begin{equation}
q(t, D)u(x) = (2\pi)^{-n/2}\int_{\mathbb{R}^n}e^{ix\cdot\xi}q(t, \xi)\hat{u}(\xi)\,\mathrm{d}\xi, \,\,\, u \in \mathcal{S}(\mathbb{R}^n),
\end{equation}
compare \eqref{PDOwNDSeg} in Example \ref{FTaSOSE2}.  An open question is when $q(t, \xi)$ is negative definite for all $t \ge 0$.

\begin{example}\label{TDSME2}
\hspace{0.5cm}
\begin{itemize}
\item[A.] Consider the Gaussian process in $\mathbb{R}^n$ which has characteristic exponent $\psi_G(\xi) = \frac{1}{2}|\xi|^2$ and transition densities $(p_t^G)_{t > 0}$ given by,
$$
p_t^G(x) = \frac{1}{(2\pi t)^{n/2}}e^{-\frac{|x|^2}{2t}}.
$$
By calculation,
$$
q_G(t, \xi) := -\frac{\partial}{\partial t}\ln\frac{p_\frac{1}{t}^G(\xi)}{p_\frac{1}{t}^G(0)} = \frac{\partial}{\partial t}\frac{t}{2}|\xi|^2 = \frac{1}{2}|\xi|^2 = \psi_G(\xi),
$$
which is continuous negative definite.

\item[B.] Consider the Cauchy process in $\mathbb{R}$ which has characteristic exponent $\psi_C(\xi) = |\xi|$ and transition densities $(p_t^C)_{t > 0}$ given by,
$$
p_t^C(x) = \frac{1}{\pi}\frac{t}{x^2 + t^2}.
$$
By calculation,
$$
q_C(t, \xi) := -\frac{\partial}{\partial t}\ln\frac{p_\frac{1}{t}^C(\xi)}{p_\frac{1}{t}^C(0)} = \frac{\partial}{\partial t}\ln(t^2\xi^2 + 1) = \frac{2}{t}\frac{\xi^2}{\xi^2 + \frac{1}{t^2}},
$$
which is continuous negative definite since $f_a(\lambda) = \frac{\lambda}{\lambda + a}$ is a Bernstein function for all $a > 0$ and $q_C(t, \xi) = \frac{2}{t}f_{1/t^2}(\xi^2)$.

\item[C.] Consider the symmetric Meixner process in $\mathbb{R}$ which has characteristic exponent $\psi_M(\xi) = \ln\cosh\xi$ and transition densities $(p_t^M)_{t > 0}$ given by,
\begin{equation}\label{SMPTD}
p_t^M(x) = \frac{2^{t - 1}}{\pi\Gamma(t)}\bigg|\Gamma\Big(\frac{t + ix}{2}\Big)\bigg|^2.
\end{equation}
By \cite[Example 5.4]{DProofPaper},
$$
\frac{p_t^M(\xi)}{p_t^M(0)} = \bigg|\frac{\Gamma\big(\frac{t + i\xi}{2}\big)}{\Gamma\big(\frac{t}{2}\big)}\bigg|^2 = \prod_{j = 0}^\infty\bigg(1 + \frac{\xi^2}{(t + 2j)^2}\bigg)^{-1},
$$
and since for a sequence $(a_j)_{j \ge 1}$ of positive numbers the convergence of $\prod_{j = 0}^\infty(1 + a_j)$, $\sum_{j = 0}^\infty\ln(1 + a_j)$ and $\sum_{j = 0}^\infty a_j$ is equivalent, we find,
$$
-\ln\frac{p_\frac{1}{t}^M(\xi)}{p_\frac{1}{t}^M(0)} = \sum_{j = 0}^\infty\ln\bigg(1 + \frac{\xi^2}{(\frac{1}{t} + 2j)^2}\bigg).
$$
Since this converges locally uniformly we may differentiate the series term by term to find,
$$
\begin{aligned}
q_M(t, \xi) & := -\frac{\partial}{\partial t}\ln\frac{p_\frac{1}{t}^M(\xi)}{p_\frac{1}{t}^M(0)}\\
& = \sum_{j = 0}^\infty\frac{\partial}{\partial t}\ln\bigg(1 + \frac{\xi^2}{(\frac{1}{t} + 2j)^2}\bigg)\\
& = \sum_{j = 0}^\infty\bigg(1 + \frac{\xi^2}{(\frac{1}{t} + 2j)^2}\bigg)^{-1}\xi^2\frac{\partial}{\partial t}\Big(\frac{1}{t} + 2j\Big)^{-2}\\
& = \sum_{j = 0}^\infty\frac{2}{t^2(\frac{1}{t} + 2j)}\frac{\xi^2}{(\frac{1}{t} + 2j)^2 + \xi^2}.
\end{aligned}
$$
We have that,
$$
\sum_{j = 0}^\infty\frac{2}{t^2(\frac{1}{t} + 2j)}\frac{\xi^2}{(\frac{1}{t} + 2j)^2 + \xi^2} \le \frac{2\xi^2}{t^2}\bigg(t^3 + \frac{1}{8}\sum_{j = 1}\frac{1}{j^3}\bigg),
$$
thus $q_M(t, \xi)$ converges locally uniformly by the comparison test for $t$ and $\xi$ in compact sets.  Moreover, since for all $a > 0$, $\lambda \mapsto \frac{\lambda}{\lambda + a}$ is a Bernstein function, we find that $\frac{x^2}{(\frac{1}{t} + 2j)^2 + x^2}$ is continuous negative definite and since the set of continuous negative definite functions is a convex cone closed under uniform convergence on compact sets, $q_M(t, \xi)$ is continuous negative definite for all $t > 0$.  By direct calculation we can find a closed form for $q_M(t, \xi)$, firstly since $\overbar{\Gamma(z)} = \Gamma(\bar{z})$ for $z \in \mathbb{C}$,
$$
\bigg|\frac{\Gamma\big(\frac{s + i\xi}{2}\big)}{\Gamma\big(\frac{s}{2}\big)}\bigg|^2 = \frac{\Gamma\big(\frac{s + i\xi}{2}\big)}{\Gamma\big(\frac{s}{2}\big)}\cdot\overbar{\frac{\Gamma\big(\frac{s + i\xi}{2}\big)}{\Gamma\big(\frac{s}{2}\big)}} = \Gamma\bigg(\frac{s + i\xi}{2}\bigg)\Gamma\bigg(\frac{s - i\xi}{2}\bigg)\bigg(\Gamma\Big(\frac{s}{2}\Big)\bigg)^{-2},
$$
then,
$$
\frac{\partial}{\partial s}\bigg|\frac{\Gamma\big(\frac{s + i\xi}{2}\big)}{\Gamma\big(\frac{s}{2}\big)}\bigg|^2 = \frac{1}{2}\bigg|\frac{\Gamma\big(\frac{s + i\xi}{2}\big)}{\Gamma\big(\frac{s}{2}\big)}\bigg|^2\bigg(\digamma\bigg(\frac{s + i\xi}{2}\bigg) + \digamma\bigg(\frac{s - i\xi}{2}\bigg) - 2\digamma\bigg(\frac{s}{2}\bigg)\bigg),
$$
where,
$$
\digamma(z) = \frac{\mathrm{d}}{\mathrm{d}z}\ln\Gamma(z) = \frac{\Gamma'(z)}{\Gamma(z)},
$$
is the digamma function.  Finally for $s = 1/t$ we get,
$$
\frac{\partial}{\partial t}\ln\bigg|\frac{\Gamma\big(\frac{1/t + i\xi}{2}\big)}{\Gamma\big(\frac{1/t}{2}\big)}\bigg|^2 = -\frac{1}{t^2}\bigg|\frac{\Gamma\big(\frac{1/t + i\xi}{2}\big)}{\Gamma\big(\frac{1/t}{2}\big)}\bigg|^{-2}\frac{\partial}{\partial s}\bigg|\frac{\Gamma\big(\frac{s + i\xi}{2}\big)}{\Gamma\big(\frac{s}{2}\big)}\bigg|^2,
$$
or,
$$
q_M(t, \xi) = \frac{1}{2t^2}\Bigg(\digamma\bigg(\frac{1/t + i\xi}{2}\bigg) + \digamma\bigg(\frac{1/t - i\xi}{2}\bigg) - 2\digamma\bigg(\frac{1/t}{2}\bigg)\Bigg).
$$
Compare \eqref{SMPTD}.
\end{itemize}
\end{example}
\begin{remark}
We note that the dual of the Meixner process has a symbol that can be thought of as an infinite sum of functions, each of which would be the symbol of the dual of a Cauchy process.
\end{remark}

\begin{example}
Consider the relativistic Hamiltonian process on $\mathbb{R}$ which has characteristic exponent $\psi_H(\xi) = \sqrt{m^2 + \xi^2} - m$ for $m > 0$ and transition densities $(p_t^H)_{t > 0}$ given by,
$$
p_t^H(x) = \frac{mte^{mt}}{\pi}\frac{K_1\big(m\sqrt{t^2 + x^2}\big)}{\sqrt{t^2 + x^2}}.
$$
By calculation,
$$
\frac{p_\frac{1}{t}^H(\xi)}{p_\frac{1}{t}^H(0)} = \frac{K_1\Big(m\sqrt{\frac{1}{t^2} + \xi^2}\Big)}{K_1(\frac{m}{t})}\bigg(\frac{1}{1 + t^2x^2}\bigg)^{1/2},
$$
so,
$$
-\ln\frac{p_\frac{1}{t}^H(\xi)}{p_\frac{1}{t}^H(0)} = \ln K_1\Big(\frac{m}{t}\Big) - \ln K_1\bigg(m\sqrt{\frac{1}{t^2} + \xi^2}\bigg) + \frac{1}{2}\ln(1 + t^2x^2),
$$
thus,
$$
\begin{aligned}
q_H(t, \xi) := & -\frac{\partial}{\partial t}\ln\frac{p_\frac{1}{t}^H(\xi)}{p_\frac{1}{t}^H(0)}\\
= & \frac{m}{t^3\sqrt{\frac{1}{t^2} + \xi^2}}\frac{K_1'\Big(m\sqrt{\frac{1}{t^2} + \xi^2}\Big)}{K_1\Big(m\sqrt{\frac{1}{t^2} + \xi^2}\Big)} - \frac{m}{t^2}\frac{K_1'\big(\frac{m}{t}\big)}{K_1\big(\frac{m}{t}\big)} + \frac{1}{t}\frac{\xi^2}{\frac{1}{t^2} + \xi^2}.
\end{aligned}
$$
It is open whether $\xi \mapsto q_H(t, \xi)$ is a continuous negative definite function or whether $q_H^{1/2}(t, \xi - \eta)$ is a metric on $\mathbb{R}^n$.
\end{example}


\section{Fundamental Solutions of Evolution Equ\-ations and Additive Processes}\label{Se.FSoEEaAP}

Having seen that time dependent negative definite symbols seem to be of importance to the dual processes of L\'evy processes we would like to investigate them further and see what kind of processes they generate, in particular whether they generate processes at all.  Similarly to the time independent case we want to introduce an analogue of the one-parameter semigroup.  For this we introduce the notion of a fundamental solution.  We take the following definition from Tanabe \cite{Tanabe}.

\begin{definition}
Let $Y$ be a Banach space of functions and $X\big([0, T]; Y\big)$ the Banach space of functions on $[0, T]$ giving functions in $Y$.  Let $u \in X\big([0, T]; Y\big)$ be the solution to the initial value problem,
\begin{equation}\label{IVP}
\begin{cases}
\frac{\partial u}{\partial t}(t, x) = A(t)u(t, x) + f(t, x), & 0 \le t \le T,\\
u(0, x) = u_0(x).
\end{cases}
\end{equation}
where $A(t)$ generates a semigroup for each $t \ge 0$.  We call a strongly continuous function $U(t, s)$, that is defined on $0 \le s \le t \le T$ and takes values in $B(Y)$, a \emph{fundamental solution} to \eqref{IVP} if it satisfies the following,
\begin{itemize}

\item[Si)] $U(t, r)U(r, s) = U(t, s)$, for $0 \le s \le r \le t \le T$;

\item[Sii)] $U(s, s) = I$, for $0 \le s \le T$;

\item[Siii)] $\frac{\partial}{\partial t}U(t, s) = A(t)U(t, s)$, for $0 \le s \le t \le T$;

\item[Siv)] $\frac{\partial}{\partial s}U(t, s) = -U(t, s)A(s)$, for $0 \le s \le t \le T$.

\end{itemize}
\end{definition}
\begin{remark}\label{TDSFSR1}
\hspace{0.5cm}
\begin{itemize}
\item[A.] In Siii) and Siv) some care is needed for unbounded operators A(t). In this case it is assumed that the equation holds for a dense subspace which is determined by the concrete operator.

\item[B.] The importance of a fundamental solution to the initial value problem \eqref{IVP} is that when one exists, the solution can be written as,
$$
u(t, x) = U(t, 0)u_0(x) + \int_0^tU(t, s)f(s, x)\,\mathrm{d}s.
$$

\item[C.] When operator $A(t)$ is no longer time dependent, $U(t, s)$ reduces to the semigroup operator $T_{t - s}$.
\end{itemize}
\end{remark}

Let $q : [0, \infty) \times \mathbb{R}^n \to \mathbb{C}$, $(t, \xi) \mapsto q(t, \xi)$, be a continuous function such that for all $t \ge 0$ the function $q(t, \cdot) : \mathbb{R}^n \to \mathbb{C}$ is continuous negative definite.  For all $t \ge 0$ we define on $\mathcal{S}(\mathbb{R}^n)$ the operator,
\begin{equation}
q(t, D)u(x) = (2\pi)^{-n/2}\int_{\mathbb{R}^n}e^{ix\cdot\xi}q(t, \xi)\hat{u}(\xi)\,\mathrm{d}\xi.
\end{equation}
For our purposes we want to consider inital value problems of the form,
\begin{equation}\label{IVPP}
\begin{cases}
\frac{\partial u}{\partial t}(t, x) = -q(t, D)u(t, x) + f(t, x), \,\,\, 0 \le t \le T,\\
u(0, x) = u_0(x),
\end{cases}
\end{equation}
on the Banach space $\mathcal{L}^2\big([0, T]; \mathcal{L}^2(\mathbb{R}^n)\big)$.  We need the following result.
\begin{proposition}\label{TDSFSP1}
The function,
$$
\int_s^tq(\tau, \cdot)\,\mathrm{d}\tau,
$$
is continuous negative definite.
\end{proposition}
\begin{proof}
Since $q(\tau, \cdot)$ for $\tau \ge 0$ is negative definite, for any choice of $k \in \mathbb{N}$ and vectors $\xi^1, \dots, \xi^k \in \mathbb{R}^n$ the matrix $\big(q(\tau, \xi^j) + \overbar{q(\tau, \xi^l)} - q(\tau, \xi^j - \xi^l)\big)_{j, l = 1, \dots, k}$ is positive Hermitian, i.e. for all $\lambda_1, \dots, \lambda_k \in \mathbb{C}$ we have,
$$
\sum_{j, l = 1}^k\big(q(\tau, \xi^j) + \overbar{q(\tau, \xi^l)} - q(\tau, \xi^j - \xi^l)\big)\lambda_l\bar{\lambda}_j \ge 0.
$$
Now,
$$
\int_s^t\sum_{j, l = 1}^k\big(q(\tau, \xi^j) + \overbar{q(\tau, \xi^l)} - q(\tau, \xi^j - \xi^l)\big)\lambda_l\bar{\lambda}_j\,\mathrm{d}\tau \ge 0,
$$
or,
$$
\sum_{j, l = 1}^k\left(\int_s^tq(\tau, \xi^j)\,\mathrm{d}\tau + \overbar{\int_s^tq(\tau, \xi^l)\,\mathrm{d}\tau} - \int_s^tq(\tau, \xi^j - \xi^l)\,\mathrm{d}\tau\right)\lambda_l\bar{\lambda}_j \ge 0,
$$
giving negative definiteness.  The continuity of $\int_s^tq(\tau, \cdot)\,\mathrm{d}\tau$ on compact sets follows easily, hence it is continuous on $\mathbb{R}^n$.
\end{proof}

Since $\int_s^tq(\tau, \xi)\,\mathrm{d}\tau$ is continuous negative definite we know that,
$$
e^{-\int_s^tq(\tau, \xi)\,\mathrm{d}\tau},
$$
is continuous positive definite \cite[Corollary 3.6.17]{Vol1}.  Hence we can define a family $(\mu_{t, s})_{t \ge s \ge 0}$ of bounded Borel measures on $\mathbb{R}^n$ by,
\begin{equation}
\hat{\mu}_{t, s}(\xi) = (2\pi)^{-n/2}e^{-\int_s^tq(\tau, \xi)\,\mathrm{d}\tau}.\label{FBBM}
\end{equation}
\begin{example}\label{TDSFSE1}
\hspace{0.5cm}
\begin{itemize}
\item[A.] Consider the dual of the Guassian process on $\mathbb{R}^n$ which has time independent negative definite symbol $q_G(t, \xi) = \frac{1}{2}|\xi|^2$ by Example \ref{TDSME2}.A.  Since we do not have time dependence we can simply write,
$$
\hat{\mu}_{t, s}^G(\xi) := (2\pi)^{-n/2}e^{-\int_s^tq_G(\tau, \xi)\,\mathrm{d}\tau} = (2\pi)^{-n/2}e^{-\frac{1}{2}(t - s)|\xi|^2}.
$$

\item[B.] Consider the dual of the Cauchy process on $\mathbb{R}$ (see Example \ref{DofCPeg}) which has time dependent negative definite symbol $q_C(t, \xi) = \frac{\partial}{\partial t}\ln(t^2\xi^2 + 1)$ by Example \ref{TDSME2}.B.  By calculation,
$$
\begin{aligned}
\hat{\mu}_{t, s}^C(\xi) := & \, (2\pi)^{-1/2}e^{-\int_s^tq_C(\tau, \xi)\,\mathrm{d}\tau}\\
= & \, (2\pi)^{-1/2}e^{-\ln(t^2\xi^2 + 1) + \ln(s^2\xi^2 + 1)}\\
= & \, (2\pi)^{-1/2}\frac{s^2\xi^2 + 1}{t^2\xi^2 + 1}.
\end{aligned}
$$

\item[C.] Consider the dual family of probability measures associated with the symmetric Meixner semigroup on $\mathbb{R}$ which corresponds to the time dependent negative definite symbol $q_M(t, \xi) = -\frac{\partial}{\partial t}\ln\left|\frac{\Gamma\left(\frac{1/t + i\xi}{2}\right)}{\Gamma\left(\frac{1/t}{2}\right)}\right|^2$ given by Example \ref{TDSME2}.C.  By calculation,
$$
\begin{aligned}
\hat{\mu}_{t, s}^M(\xi) := & \, (2\pi)^{-1/2}e^{-\int_s^tq_M(\tau, \xi)\,\mathrm{d}\tau}\\
= & \, (2\pi)^{-1/2}e^{\ln\left|\frac{\Gamma\left(\frac{1/t + i\xi}{2}\right)}{\Gamma\left(\frac{1/t}{2}\right)}\right|^2 - \ln\left|\frac{\Gamma\left(\frac{1/s + i\xi}{2}\right)}{\Gamma\left(\frac{1/s}{2}\right)}\right|^2}\\
= & \, (2\pi)^{-1/2}\left|\frac{\Gamma\left(\frac{1/t + i\xi}{2}\right)}{\Gamma\left(\frac{1/t}{2}\right)}\right|^2 \left|\frac{\Gamma\left(\frac{1/s}{2}\right)}{\Gamma\left(\frac{1/s + i\xi}{2}\right)}\right|^2.
\end{aligned}
$$
\end{itemize}
\end{example}

We also define the family of operators $\big(V(t, s)\big)_{t \ge s \ge 0}$ on Schwartz space $\mathcal{S}(\mathbb{R}^n)$ by,
\begin{equation}\label{FoO}
V(t, s)u(x) = \int_{\mathbb{R}^n}u(x - y)\,\mu_{t, s}(\mathrm{d}y).
\end{equation}
Taking the Fourier transform in \eqref{FoO} and using the convolution theorem,
\begin{align}
\big(V(t, s)u\big)^\wedge(\xi) & = (u\ast\mu_{t, s})^\wedge(\xi)\nonumber\\
& = (2\pi)^{n/2}\hat{u}(\xi)\hat{\mu}_{t, s}(\xi)\nonumber\\
& = e^{-\int_s^tq(\tau, \xi)\,\mathrm{d}\tau}\hat{u}(\xi)\nonumber,
\end{align}
we see that $V(t, s)$ can be written as,
\begin{equation}\label{Op}
V(t, s)u(x) = (2\pi)^{-n/2}\int_{\mathbb{R}^n}e^{ix\cdot\xi}e^{-\int_s^tq(\tau, \xi)\,\mathrm{d}\tau}\hat{u}(\xi)\,\mathrm{d}\xi,
\end{equation}
which is a useful representation for $V(t, s)$, in particular for the following proof.

\begin{proposition}\label{TDSFSP2}
A fundamental solution of \eqref{IVPP} is given by $V(t, s)$.
\end{proposition}
\begin{proof}
It suffices to show that $V(t, s)$ has all of the properties of a fundamental solution to \eqref{IVPP} in a dense subspace of $\mathcal{L}^2(\mathbb{R}^n)$, namely $\mathcal{S}(\mathbb{R}^n)$.  For $v \in \mathcal{S}(\mathbb{R}^n)$,
$$
\big\|V(t, s)v\big\|_0 = \big\|\mathrm{F}V(t, s)v\big\|_0 \le \|\hat{v}\|_0 = \|v\|_0,
$$
by Plancherel's theorem.  Since $\mathcal{S}(\mathbb{R}^n)$ is dense in $\mathcal{L}^2(\mathbb{R}^n)$, it follows that each $V(t, s)$ has an extension to $\mathcal{L}^2(\mathbb{R}^n)$ (which we again denote by $V(t, s)$) and this extension is bounded and linear.  Thus, $V(t, s)$ takes values in $B\big(\mathcal{L}^2(\mathbb{R}^n)\big)$.  Next we observe,
\begin{align}
\|V(t, s)v - v\|_0 & = \big\|\mathrm{F}\big(V(t, s)v - v\big)\big\|_0\nonumber\\
& = \int_{\mathbb{R}^n}\left|e^{-\int_s^tq(\tau, \xi)\,\mathrm{d}\tau} - 1\right|^2|\hat{u}(\xi)|^2\,\mathrm{d}\xi \to 0 \,\,\, \text{as} \,\,\, s \to t,\nonumber
\end{align}
implying the strong continuity of $V(t, s)$ and for each $t \ge 0$, $q(t, \cdot)$ is negative definite and therefore $-q(t_0, D)$ generates a semigroup $(T_\tau^{(t_0)})_{\tau \ge 0}$ for each $t_0 \ge 0$ fixed.

All that remains is to show that $V(t, s)$ satisfies Si) - Siv) with $A(t) = -q(t, D)$.
\begin{itemize}

\item[Si)] Note that \eqref{Op} can be written,
$$
V(t, s) = \mathrm{F}^{-1}e^{-\int_s^tq(\tau, \xi)\,\mathrm{d}\tau}\mathrm{F}, \,\,\, 0 \le s \le t \le T.
$$
Now for $0 \le s \le r \le t \le T$,
\begin{align}
V(t, r)V(r, s) & = \mathrm{F}^{-1}e^{-\int_r^tq(\tau, \xi)\,\mathrm{d}\tau}\mathrm{F}\mathrm{F}^{-1}e^{-\int_s^rq(\tau, \xi)\,\mathrm{d}\tau}\mathrm{F}\nonumber\\
& = \mathrm{F}^{-1}e^{-\int_r^tq(\tau, \xi)\,\mathrm{d}\tau}e^{-\int_s^rq(\tau, \xi)\,\mathrm{d}\tau}\mathrm{F}\nonumber\\
& = \mathrm{F}^{-1}e^{-\int_s^tq(\tau, \xi)\,\mathrm{d}\tau}\mathrm{F}.\nonumber\\
& = V(t, s)\nonumber
\end{align}

\item[Sii)] For $0 \le s \le T$ and $v \in \mathcal{S}(\mathbb{R}^n)$,
\begin{align}
V(s, s)v(x) & = (2\pi)^{-n/2}\int_{\mathbb{R}^n}e^{ix\cdot\xi}e^{-\int_s^sq(\tau, \xi)\,\mathrm{d}\tau}\hat{v}(\xi)\,\mathrm{d}\xi\nonumber\\
& = (2\pi)^{-n/2}\int_{\mathbb{R}^n}e^{ix\cdot\xi}\hat{v}(\xi)\,\mathrm{d}\xi\nonumber\\
& = v(x).\nonumber
\end{align}
Hence, $V(s, s) = I$.

\item[Siii)] For $0 \le s \le T$ and $v \in \mathcal{S}(\mathbb{R}^n)$,
\begin{align}
\frac{\partial}{\partial t}V(t, s)v(x) & = (2\pi)^{-n/2}\int_{\mathbb{R}^n}e^{ix\cdot\xi}\frac{\partial}{\partial t}\left(e^{-\int_s^tq(\tau, \xi)\,\mathrm{d}\tau}\right)\hat{v}(\xi)\,\mathrm{d}\xi\nonumber\\
& = (2\pi)^{-n/2}\int_{\mathbb{R}^n}e^{ix\cdot\xi}\big(-q(t, \xi)\big)\big(V(t, s)v\big)^\wedge(\xi)\,\mathrm{d}\xi\nonumber\\
& = (2\pi)^{-n/2}\int_{\mathbb{R}^n}e^{ix\cdot\xi}\big(-q(t, D)V(t, s)v\big)^\wedge(\xi)\,\mathrm{d}\xi\nonumber\\
& = -q(t, D)V(t, s)v(x)\nonumber.
\end{align}

\item[Siv)] For $0 \le s \le T$ and $v \in \mathcal{S}(\mathbb{R}^n)$,
\begin{align}
\frac{\partial}{\partial s}V(t, s)v(x) & = (2\pi)^{-n/2}\int_{\mathbb{R}^n}e^{ix\cdot\xi}\frac{\partial}{\partial s}\left(e^{\int_t^sq(\tau, \xi)\,\mathrm{d}\tau}\right)\hat{v}(\xi)\,\mathrm{d}\xi\nonumber\\
& = (2\pi)^{-n/2}\int_{\mathbb{R}^n}e^{ix\cdot\xi}e^{-\int_s^tq(\tau, \xi)\,\mathrm{d}\tau}q(s, \xi)\hat{v}(\xi)\,\mathrm{d}\xi\nonumber\\
& = (2\pi)^{-n/2}\int_{\mathbb{R}^n}e^{ix\cdot\xi}e^{-\int_s^tq(\tau, \xi)\,\mathrm{d}\tau}\big(q(s, D)v\big)^\wedge(\xi)\,\mathrm{d}\xi\nonumber\\
& = -V(t, s)\big(-q(s, D)\big)v(x)\nonumber.
\end{align}

\end{itemize}
\end{proof}

\newpage

\section{Metrics and Transition Densities}\label{Se.MaTD}

Our aim now is to see what kind of processes are generated by our time dependent negative definite symbol and also to see if we can employ some of the geometric methods used in \cite{Paper} to our situation.  We let $q : [0, \infty) \times \mathbb{R}^n \to \mathbb{R}$, $(t, \xi) \mapsto q(t, \xi)$ be a continuous function such that for all $t \ge 0$ the function $q(t, \cdot) : \mathbb{R}^n \to \mathbb{R}$ is continuous negative definite.  Furthermore, we impose $q(t, \xi) = 0$ if and only if $\xi = 0$.  We now have that $(\mu_{t, s})_{t \ge s \ge 0}$ as defined in \eqref{FBBM} is a family of probability measures on $\mathbb{R}^n$ since,
$$
\mu_{t, s}(\mathbb{R}^n) = (2\pi)^{n/2}\hat{\mu}_{t, s}(0) = 1.
$$
We now turn to a theorem discussed in Sato \cite{Sato} that characterises the family of probability measures of additive processes.

\begin{theorem}\label{APfromTDCNDF}
Let $(X_t)_{t \ge 0}$ be an additive process in law on $\mathbb{R}^n$ and for $0 \le s \le t < \infty$, let $\nu_{t, s}$ be the distribution of $X_t - X_s$.  Then $\nu_{t, s}$ is infinitely divisible and,
\begin{itemize}

\item[Ai)] $\nu_{t, r}\ast\nu_{r, s} = \nu_{t, s}$, for $0 \le s \le r \le t < \infty$.

\item[Aii)] $\nu_{s, s} = \varepsilon_0$, for $0 \le s < \infty$.

\item[Aiii)] $\nu_{t, s} \to \varepsilon_0$ as $s \uparrow t$.

\item[Aiv)] $\nu_{t, s} \to \varepsilon_0$ as $t \downarrow s$.

\end{itemize}

Conversely, if $(\nu_{t, s})_{\infty > t \ge s \ge 0}$ is a family of probability measures on $\mathbb{R}^n$ satisfying Ai) - Aiv), then there exists an additive process in law $(X_t)_{t \ge 0}$ such that for $0 \le s \le t < \infty$, $X_t - X_s$ has the distribution $\nu_{t, s}$.
\end{theorem}

\begin{proposition}
Let $(\mu_{t, s})_{t \ge s \ge 0}$ be as in \eqref{FBBM}, then $(\mu_{t, s})_{t \ge s \ge 0}$ is a family of distributions of the increments of an additive process on $\mathbb{R}^n$.
\end{proposition}
\begin{proof}
It suffices to show that $(\mu_{t, s})_{t \ge s \ge 0}$ satisfies Ai) - Aiv), to do this we turn to the Fourier transform of our probability measures.
\begin{itemize}

\item[Ai)] For $0 \le s \le r \le t$,
\begin{align}
\hat{\mu}_{t, r}(\xi)\hat{\mu}_{r, s}(\xi) & = (2\pi)^{-n/2}e^{-\int_r^tq(\tau, \xi)\,\mathrm{d}\tau}(2\pi)^{-n/2}e^{-\int_s^rq(\tau, \xi)\,\mathrm{d}\tau}\nonumber\\
& = (2\pi)^{-n}e^{-\int_s^tq(\tau, \xi)\,\mathrm{d}\tau}\nonumber\\
& = (2\pi)^{-n/2}\hat{\mu}_{t, s}(\xi)\nonumber,
\end{align}
and by the convolution theorem this gives $\mu_{t, r}\ast\mu_{r, s} = \mu_{t, s}$.

\item[Aii)] For $s \ge 0$,
$$
\hat{\mu}_{s, s}(\xi) = (2\pi)^{-n/2}e^{-\int_s^sq(\tau, \xi)\,\mathrm{d}\tau} = (2\pi)^{-n/2}.
$$
This implies $\mu_{s, s} = \varepsilon_0$ since,
$$
\hat{\varepsilon}_0(\xi) = (2\pi)^{-n/2}\int_{\mathbb{R}^n}e^{-ix\cdot\xi}\varepsilon_0(\mathrm{d}x) = (2\pi)^{-n/2}.
$$

\item[Aiii)] For $0 \le s \le t$,
$$
\hat{\mu}_{t, s}(\xi) = (2\pi)^{-n/2}e^{-\int_s^tq(\tau, \xi)\,\mathrm{d}\tau} \to (2\pi)^{-n/2}e^{-\int_t^tq(\tau, \xi)\,\mathrm{d}\tau} = \hat{\varepsilon}_0(\xi),
$$
as $s \uparrow t$.

\item[Aiv)] Shown similarly to Aiii).

\end{itemize}
\end{proof}

If we suppose that $e^{-\int_s^tq(\tau, \cdot)\,\mathrm{d}\tau} \in \mathcal{L}^1(\mathbb{R}^n, \lambda^{(n)})$ we note that our fundamental solution $V(t, s)$ can be re-written as,
\begin{align}
V(t, s)u(x) & = (2\pi)^{-n/2}\int_{\mathbb{R}^n}e^{ix\cdot\xi}e^{-\int_s^tq(\tau, \xi)\,\mathrm{d}\tau}\hat{u}(\xi)\,\mathrm{d}\xi\nonumber\\
& = (2\pi)^{-n/2}\int_{\mathbb{R}^n}e^{ix\cdot\xi}e^{-\int_s^tq(\tau, \xi)\,\mathrm{d}\tau}(2\pi)^{-n/2}\int_{\mathbb{R}^n}e^{-iy\cdot\xi}u(y)\,\mathrm{d}y\,\mathrm{d}\xi\nonumber\\
& = \int_{\mathbb{R}^n}(2\pi)^{-n}\int_{\mathbb{R}^n}e^{i(x - y)\cdot\xi}e^{-\int_s^tq(\tau, \xi)\,\mathrm{d}\tau}\,\mathrm{d}\xi\,u(y)\,\mathrm{d}y,\nonumber
\end{align}
i.e. $V(t, s)$ is a convolution operator with $V(t, s)u = p_{t, s} \ast u$ where,
\begin{equation}
p_{t, s}(x) = (2\pi)^{-n}\int_{\mathbb{R}^n}e^{ix\cdot\xi}e^{-\int_s^tq(\tau, \xi)\,\mathrm{d}\tau}\,\mathrm{d}\xi = \mathrm{F}^{-1}\hat{\mu}_{t, s}(x)\label{TD},
\end{equation}
which defines the transition densities $(p_{t, s})_{t > s \ge 0}$ of the additive process associated to $q(t, \xi)$.
\begin{example}
\hspace{0.5cm}
\begin{itemize}
\item[A.] Consider the dual of the Guassian process on $\mathbb{R}^n$ which has an associated family of probability measures $(\mu_{t, s}^G)_{t \ge s \ge 0}$ which by Example \ref{TDSFSE1}.A is given by,
$$
\hat{\mu}_{t, s}^G(\xi) = (2\pi)^{-n/2}e^{-\frac{1}{2}(t - s)|\xi|^2}.
$$
Since this is just a Gaussian with mean $0$ and standard deviation $t - s$ we know,
$$
p_{t, s}^G(x) := \mathrm{F}^{-1}\hat{\mu}_{t, s}^G(x) = \frac{1}{\big(2\pi(t - s)\big)^{n/2}}e^{-\frac{|x|^2}{2(t - s)}}.
$$

\item[B.] From Example \ref{TDSME2}.B we know,
$$
q_C(t, \xi) = \frac{2}{t}\frac{\xi^2}{\xi^2 + 1/t^2},
$$
is continuous negative definite for all $t \ge 0$ and we also have $q_C(t, \xi) = 0$ if and only if $\xi = 0$, hence by Theorem \ref{APfromTDCNDF} the dual of the Cauchy process on $\mathbb{R}$ exists and is an additive process.  By Example \ref{TDSFSE1}.B its family of probability measures $(\mu_{t, s}^C)_{t \ge s \ge 0}$ is given by,
$$
\hat{\mu}_{t, s}^C(\xi) = (2\pi)^{-1/2}\frac{s^2\xi^2 + 1}{t^2\xi^2 + 1}.
$$
To calculate the transition densities $p_{t, s}^C(x) := \mathrm{F}^{-1}\hat{\mu}_{t, s}^C(x)$ we first note by substitution that,
$$
\begin{aligned}
\mathrm{F}\big((2\pi)^{-1/2}e^{-t|\cdot|}\big)(\xi) & = \frac{1}{2\pi}\int_{\mathbb{R}}e^{-i\xi x}e^{-t|x|}\,\mathrm{d}x\\
& = \frac{1}{2\pi}\bigg(\int_{-\infty}^0 e^{-ix\xi + tx}\,\mathrm{d}x + \int_0^\infty e^{-ix\xi - tx}\,\mathrm{d}x\bigg)\\
& = \frac{1}{2\pi}\bigg(\frac{e^{x(t - i\xi)}}{t - i\xi}\bigg|_{-\infty}^0 - \frac{e^{x(-t - i\xi)}}{t + i\xi}\bigg|_0^\infty\bigg)\\
& = \frac{1}{\pi}\frac{t}{\xi^2 + t^2},
\end{aligned}
$$
or equivalently,
$$
\mathrm{F}^{-1}\bigg((2\pi)^{-1/2}\frac{\frac{1}{t^2}}{|\cdot|^2 + \frac{1}{t^2}}\bigg)(x) = \frac{1}{2t}e^{-\frac{1}{t}|x|}.
$$
Now,
$$
\begin{aligned}
p_{t, s}^C(x) := & \, \mathrm{F}^{-1}\bigg((2\pi)^{-1/2}\frac{s^2|\cdot|^2 + 1}{t^2|\cdot|^2 + 1}\bigg)(x)\\
= & \, \frac{s^2}{t^2}\mathrm{F}^{-1}\bigg((2\pi)^{-1/2}\frac{t^2|\cdot|^2}{t^2|\cdot|^2 + 1}\bigg)(x)\\
& \hspace{0.7cm} + \mathrm{F}^{-1}\bigg((2\pi)^{-1/2}\frac{1}{t^2|\cdot|^2 + 1}\bigg)(x)\\
= & \, \frac{s^2}{t^2}\mathrm{F}^{-1}\big((2\pi)^{-1/2}\big)(x) + \bigg(1 - \frac{s^2}{t^2}\bigg)\mathrm{F}^{-1}\bigg((2\pi)^{-1/2}\frac{\frac{1}{t^2}}{|\cdot|^2 + \frac{1}{t^2}}\bigg)(x)\\
= & \, \frac{s^2}{t^2}\varepsilon_0(x) + \bigg(1 - \frac{s^2}{t^2}\bigg)\frac{1}{2t}e^{-\frac{1}{t}|x|}.
\end{aligned}
$$
Note that this is to be understood in the context of distributions in the sense of L. Schwartz.

\item[C.]  From Example \ref{TDSME2}.C we know,
$$
q_M(t, \xi) = \frac{1}{2t^2}\Bigg(\digamma\bigg(\frac{1/t + i\xi}{2}\bigg) + \digamma\bigg(\frac{1/t - i\xi}{2}\bigg) - 2\digamma\bigg(\frac{1/t}{2}\bigg)\Bigg),
$$
is continuous negative definite for all $t \ge 0$ and we also have $q_M(t, \xi) = 0$ if and only if $\xi = 0$, hence by Theorem \ref{APfromTDCNDF} the dual of the symmetric Meixner process on $\mathbb{R}$ exists and is an additive process.  By Example \ref{TDSFSE1}.C its family of probability measures $(\mu_{t, s}^M)_{t \ge s \ge 0}$ is given by,
$$
\hat{\mu}_{t, s}^M(\xi) = (2\pi)^{-1/2}\left|\frac{\Gamma\left(\frac{1/t + i\xi}{2}\right)}{\Gamma\left(\frac{1/t}{2}\right)}\right|^2 \left|\frac{\Gamma\left(\frac{1/s}{2}\right)}{\Gamma\left(\frac{1/s + i\xi}{2}\right)}\right|^2.
$$
We want to calculate its transition densities $p_{t, s}^M(x) := \mathrm{F}^{-1}\hat{\mu}_{t, s}^M(x)$ however a closed form for $p_{t, s}^M$ is not known so instead we note,
$$
\int_s^tq_M(\tau, \xi)\,\mathrm{d}\tau = -\int_s^t\frac{\partial}{\partial\tau}\ln\frac{p_\frac{1}{\tau}^M(\xi)}{p_\frac{1}{\tau}^M(0)}\,\mathrm{d}\tau = -\ln\frac{p_\frac{1}{t}^M(\xi)}{p_\frac{1}{t}^M(0)} + \ln\frac{p_\frac{1}{s}^M(\xi)}{p_\frac{1}{s}^M(0)},
$$
then by \cite[Theorem 14]{DProofPaper} (see the proof of Proposition \ref{SharedPropWithNiels}),
$$
\lim_{s \to 0}\int_s^tq_M(\tau, \xi)\,\mathrm{d}\tau = -\ln\frac{p_\frac{1}{t}^M(\xi)}{p_\frac{1}{t}^M(0)},
$$
and so,
$$
\begin{aligned}
\hat{\mu}_{t, 0}^M(\xi) & := \lim_{s \to \infty}\hat{\mu}_{t, s}^M(\xi)\\
& = (2\pi)^{-1/2}\lim_{s \to 0}e^{-\int_s^tq(\tau, \xi)\,\mathrm{d}\tau)}\\
& = (2\pi)^{-1/2}\left|\frac{\Gamma\left(\frac{1/t + i\xi}{2}\right)}{\Gamma\left(\frac{1/t}{2}\right)}\right|^2.
\end{aligned}
$$
Using our knowledge of the Meixner process we calculate,
$$
\begin{aligned}
p_{t, 0}^M(x) & = \mathrm{F}^{-1}\hat{\mu}_{t, 0}^M(x)\\
& = \frac{1}{2\pi}\int_{\mathbb{R}}e^{ix\cdot\xi}\left|\frac{\Gamma\left(\frac{1/t + i\xi}{2}\right)}{\Gamma\left(\frac{1/t}{2}\right)}\right|^2\,\mathrm{d}\xi\\
& = \frac{1}{\sqrt{2\pi}}\frac{\pi\Gamma(1/t)}{2^{\frac{1}{t} - 1}\big|\Gamma\big(\frac{1/t}{2}\big)\big|^2}\mathrm{F}^{-1}\Bigg(\frac{2^{\frac{1}{t} - 1}}{\pi\Gamma(1/t)}\bigg|\Gamma\left(\frac{1/t + i\xi}{2}\right)\bigg|^2\Bigg)(x)\\
& = \frac{1}{2\pi}\frac{\pi\Gamma(1/t)}{2^{\frac{1}{t} - 1}\big|\Gamma\big(\frac{1/t}{2}\big)\big|^2}\mathrm{F}^{-2}\left(e^{-\frac{1}{t}\ln\cosh(\cdot)}\right)(x).
\end{aligned}
$$
However, $x \mapsto e^{-\frac{1}{t}\ln\cosh x}$ is even, thus $\mathrm{F}^{-2}\big(e^{-\frac{1}{t}\ln\cosh(\cdot)}\big)(x) = e^{-\frac{1}{t}\ln\cosh x}$ and we get,
$$
p_{t, 0}^M(x) = \frac{\Gamma(1/t)}{\big|\Gamma\big(\frac{1/t}{2}\big)\big|^2}\bigg(\frac{1}{2\cosh x}\bigg)^{1/t},
$$
as expected since this is of the form,
$$
p_{t, 0}^M(x) = \frac{e^{-\frac{1}{t}\psi_M(x)}}{(2\pi)p_\frac{1}{t}^M(0)},
$$
where $\psi_M$ and $p_t^M$ are given in Example \ref{TDSME2}.C.  Compare \eqref{DPPM}.
\end{itemize}
\end{example}
For ease of notation we define,
\begin{equation}
Q_{t, s}(\xi) := \int_s^t q(\tau, \xi)\,\mathrm{d}\tau,\label{Q}
\end{equation}
which by Proposition \ref{TDSFSP1} is continuous negative definite, it is also easy to see that it is non-periodic (since $q(t, \xi) = 0$ if and only if $\xi = 0$ and $q(t, \xi) \ge 0$) and so for $t > s \ge 0$ we can define the metrics \cite{Paper},
\begin{equation}
d_{Q_{t, s}}(\xi, \eta) := \sqrt{Q_{t, s}(\xi - \eta)}\label{M}.
\end{equation}
We would also like $Q_{t, s}$ to be metric generating and we note that it is enough to assume $q(\tau, \cdot) \in \mathcal{MCN}(\mathbb{R}^n)$ for all $\tau > 0$ since by Fatou's lemma \cite[Theorem 9.11]{FubiniBook},
$$
\liminf_{|\xi| \to \infty} Q_{t, s}(\xi) = \liminf_{|\xi| \to \infty}\int_s^tq(\tau, \xi)\,\mathrm{d}\tau \ge \int_s^t\liminf_{|\xi| \to \infty} q(\tau, \xi)\,\mathrm{d}\tau > 0,
$$
i.e. if $q(\tau, \cdot) \in \mathcal{MCN}(\mathbb{R}^n)$ for all $\tau > 0$ then $Q_{t, s} \in \mathcal{MCN}(\mathbb{R}^n)$ for all $t > s \ge 0$.

\begin{theorem}\label{ThesisMainResult}
Assume that $q(\tau, \cdot) \in \mathcal{MCN}(\mathbb{R}^n)$ for all $\tau > 0$ and that $e^{-\int_s^t q(\tau, \cdot)\,\mathrm{d}\tau} \in \mathcal{L}^1(\mathbb{R}^n, \lambda^{(n)})$.  Then for all $t > s \ge 0$,
\begin{equation}
p_{t, s}(0) = (2\pi)^{-n}\int_0^\infty\lambda^{(n)}\big(B^{d_{Q_{t, s}}}(0, \sqrt{r})\big)e^{-r}\,\mathrm{d}r\label{DT}.
\end{equation}
If we have for all $t \ge 0$ that,
\begin{equation}\label{BetaBound}
\beta_0q(t_0, \xi) \le q(t, \xi) \le \beta_1q(t_0, \xi),
\end{equation}
for some $t_0 \ge 0$ and $\beta_1 \ge \beta_1 > 0$ and that the metric measure space $(\mathbb{R}^n, d_{q(t_0, \cdot)}, \lambda^{(n)})$ has volume doubling, then $e^{-\int_s^tq(\tau, \xi)\,\mathrm{d}\tau} \in \mathcal{L}^1(\mathbb{R}^n, \lambda^{(n)})$ for all $t > s \ge 0$ and,
\begin{equation}
p_{t, s}(0) \asymp \lambda^{(n)}\Big(B^{d_{Q_{t, s}}}\big(0, \sqrt{\beta_1/\beta_0}\big)\Big),
\end{equation}
for all $t > s \ge 0$.
\end{theorem}
\begin{proof}
First by Fubini's Theorem \cite[Theorem 13.11]{FubiniBook},
\begin{align}
(2\pi)^np_{t, s}(0) & = \int_{\mathbb{R}^n}e^{-Q_{t, s}(\xi)}\,\mathrm{d}\xi\nonumber\\
& = \int_0^\infty\lambda^{(n)}\big\{\xi \in \mathbb{R}^n : e^{-Q_{t, s}(\xi)} \ge\rho\big\}\,\mathrm{d}\rho\nonumber\\
& = \int_0^1\lambda^{(n)}\big\{\xi \in \mathbb{R}^n : Q_{t, s}(\xi) \le -\ln\rho\big\}\,\mathrm{d}\rho\nonumber.
\end{align}
Now using the substitution $r = -\ln\rho$ we get,
\begin{align}
(2\pi)^np_{t, s}(0) & = -\int_\infty^0\lambda^{(n)}\big\{\xi \in \mathbb{R}^n : Q_{t, s}(\xi) \le r\big\}e^{-r}\,\mathrm{d}r\nonumber\\
& = \int_0^\infty\lambda^{(n)}\big(B^{d_{Q_{t, s}}}(0, \sqrt{r})\big)e^{-r}\,\mathrm{d}r.\nonumber
\end{align}

Next, since $(\mathbb{R}^n, d_{q(t_0, \cdot)}, \lambda^{(n)})$ has the volume doubling property, by \cite[Corollary 3.10]{Paper} and ``d) $\Rightarrow$ a)'' of \cite[Proposition 5]{DProofPaper} we get that $e^{-uq(t_0, \cdot)} \in \mathcal{L}^1(\mathbb{R}^n, \lambda^{(n)})$ for all $u > 0$.  Since $e^{-uq(t_0, \cdot)} \in \mathcal{L}^1(\mathbb{R}^n, \lambda^{(n)})$ for all $u > 0$ then $e^{-\beta_0(t - s)q(t_0, \cdot)} \in \mathcal{L}^1(\mathbb{R}^n, \lambda^{(n)})$ for all $t > s \ge 0$.  Now for all $\xi \in \mathbb{R}^n$,
\begin{equation}\label{beta0inequalityproof}
\beta_0(t - s)q(t_0, \xi) \le \int_s^tq(\tau, \xi)\,\mathrm{d}\tau,
\end{equation}
or,
$$
e^{-\beta_0(t - s)q(t_0, \xi)} \ge e^{-\int_s^tq(\tau, \xi)\,\mathrm{d}\tau},
$$
and thus $e^{-\int_s^tq(\tau, \cdot)\,\mathrm{d}\tau} \in \mathcal{L}^1(\mathbb{R}^n, \lambda^{(n)})$ for all $t > s \ge 0$.  Moving on we use the monotonicity of $r \mapsto \lambda^{(n)}\big(B^{d_{Q_{t, s}}}(0, \sqrt{r})\big)$,
\begin{align}
(2\pi)^np_{t, s}(0) & \ge \int_{\beta_1/\beta_0}^\infty\lambda^{(n)}\big(B^{d_{Q_{t, s}}}(0, \sqrt{r})\big)e^{-r}\,\mathrm{d}r\nonumber\\
& \ge \lambda^{(n)}\Big(B^{d_{Q_{t, s}}}\big(0, \sqrt{\beta_1/\beta_0}\big)\Big)\int_{\beta_1/\beta_0}^\infty e^{-r}\,\mathrm{d}r\nonumber\\
& = \frac{1}{e^{\beta_1/\beta_0}}\lambda^{(n)}\Big(B^{d_{Q_{t, s}}}\big(0, \sqrt{\beta_1/\beta_0}\big)\Big).\nonumber
\end{align}
For the upper estimate we split the integral according to,
$$
\begin{aligned}
(2\pi)^np_{t, s}(0) = & \int_0^{\beta_1/\beta_0}\lambda^{(n)}\big(B^{d_{Q_{t, s}}}(0, \sqrt{r})\big)e^{-r}\,\mathrm{d}r\\
& \hspace{1cm} + \int_{\beta_1/\beta_0}^\infty\lambda^{(n)}\big(B^{d_{Q_{t, s}}}(0, \sqrt{r})\big)e^{-r}\,\mathrm{d}r.
\end{aligned}
$$
Firstly we note,
$$
\begin{aligned}
& \int_0^{\beta_1/\beta_0}\lambda^{(n)}\big(B^{d_{Q_{t, s}}}(0, \sqrt{r})\big)e^{-r}\,\mathrm{d}r\\
\le & \, \lambda^{(n)}\Big(B^{d_{Q_{t, s}}}\big(0, \sqrt{\beta_1/\beta_0}\big)\Big)\int_0^{\beta_1/\beta_0}e^{-r}\,\mathrm{d}r\\
= & \, \left(1 - \frac{1}{e^{\beta_1/\beta_0}}\right)\lambda^{(n)}\Big(B^{d_{Q_{t, s}}}\big(0, \sqrt{\beta_1/\beta_0}\big)\Big).
\end{aligned}
$$
Secondly, we see using \eqref{beta0inequalityproof} that,
$$
\begin{aligned}
& \int_{\beta_1/\beta_0}^\infty\lambda^{(n)}\big(B^{d_{Q_{t, s}}}(0, \sqrt{r})\big)e^{-r}\,\mathrm{d}r\\
\le & \int_{\beta_1/\beta_0}^\infty\lambda^{(n)}\left\{\xi \in \mathbb{R}^n : \bigg(\int_s^t\beta_0q(t_0, \xi)\,\mathrm{d}\tau\bigg)^{1/2} \le \sqrt{r}\right\}e^{-r}\,\mathrm{d}r\\
= & \int_{\beta_1/\beta_0}^\infty\lambda^{(n)}\left\{\xi \in \mathbb{R}^n : \sqrt{\beta_0(t - s)q(t_0, \xi)} \le \sqrt{r}\right\}e^{-r}\,\mathrm{d}r.
\end{aligned}
$$
Now, if $d_{q(t_0, \cdot)}$ has volume doubling then,
$$
\lambda^{(n)}\big(B^{d_{q(t_0, \cdot)}}(0, kr)\big) \le c(t_0, k)\lambda^{(n)}\big(B^{d_{q(t_0, \cdot)}}(0, r)\big),
$$
where $c(t_0, k) \le k^{\alpha(t_0)}c(t_0, 1)$ for all $k \ge 1$ and some $\alpha(t_0) \ge 0$.  Thus,
$$
\begin{aligned}
& \int_{\beta_1/\beta_0}^\infty\lambda^{(n)}\big(B^{d_{Q_{t, s}}}(0, \sqrt{r})\big)e^{-r}\,\mathrm{d}r\\
\le & \int_{\beta_1/\beta_0}^\infty c(t_0, \sqrt{r})\lambda^{(n)}\left\{\xi \in \mathbb{R}^n : \sqrt{\beta_0(t - s)q(t_0, \xi)} \le 1\right\}e^{-r}\,\mathrm{d}r\\
\le & \, c(t_0, 1)\int_{\beta_1/\beta_0}^\infty\lambda^{(n)}\left\{\xi \in \mathbb{R}^n : \sqrt{(t - s)q(t_0, \xi)} \le \sqrt{1/\beta_0}\right\}r^{\alpha(t_0)/2}e^{-r}\,\mathrm{d}r\\
\le & \, c(t_0, 1)\int_{\beta_1/\beta_0}^\infty\lambda^{(n)}\left\{\xi \in \mathbb{R}^n : \bigg(\int_s^tq(\tau, \xi)\,\mathrm{d}\tau\bigg)^{1/2} \le \sqrt{\beta_1/\beta_0}\right\}r^{\alpha(t_0)/2}e^{-r}\,\mathrm{d}r\\
= & \, d(t_0)\lambda^{(n)}\Big(B^{d_{Q_{t, s}}}\big(0, \sqrt{\beta_1/\beta_0}\big)\Big),
\end{aligned}
$$
where $d(t_0) := c(t_0, 1)\int_{\beta_1/\beta_0}^\infty r^{\alpha(t_0)/2}e^{-r}\,\mathrm{d}r < \infty$.  Hence finally we obtain,
$$
(2\pi)^np_{t, s}(0) \le \left(1 - \frac{1}{e^{\beta_1/\beta_0}} + d(t_0)\right)\lambda^{(n)}\Big(B^{d_{Q_{t, s}}}\big(0, \sqrt{\beta_1/\beta_0}\big)\Big).
$$
\end{proof}

\begin{example}\label{TDSGMetcE2}
Consider the dual of the Gaussian process on $\mathbb{R}^n$ which has time independent negative definite symbol $q_G(t, \xi) = \frac{1}{2}|\xi|^2$.  We have,
$$
Q_{t, s}^G(\xi) := \int_s^tq_G(\tau, \xi)\,\mathrm{d}\tau = \frac{t -s}{2}|\xi|^2,
$$
and metric,
$$
d_{Q_{t, s}^G}(\xi, \eta) := \sqrt{Q_{t, s}(\xi - \eta)} = \frac{t - s}{2}|\xi - \eta|.
$$
For this situation, in \eqref{BetaBound} we have $\beta_0 = \beta_1 = 1$ and then,
$$
\begin{aligned}
B^{d_{Q_{t, s}^G}}(0, 1) & = \left\{\xi \in \mathbb{R}^n : d_{Q_{t, s}^G}(\xi, 0) < 1\right\}\\
& = \left\{\xi \in \mathbb{R}^n : |\xi| < \sqrt{\frac{2}{t - s}}\right\}\\
& = B\bigg(0, \sqrt{\frac{2}{t - s}}\bigg),
\end{aligned}
$$
which is the ball of radius $\sqrt{2/(t - s)}$ centred at $0$ in $\mathbb{R}^n$, hence,
$$
\lambda^{(n)}\Big(B^{d_{Q_{t, s}^G}}(0, 1)\Big) = \frac{(2\pi)^{n/2}}{\Gamma(\frac{n}{2} + 1)}(t - s)^{-n/2}.
$$
Note that we can write,
$$
p_{t, s}^G(x) = k_n\lambda^{(n)}\Big(B^{d_{Q_{t, s}^G}}(0, 1)\Big)e^{-\frac{1}{t - s}d_{\psi_G}^2(x, 0)},
$$
where $d_{\psi_G}(\xi, \eta) = \frac{1}{\sqrt{2}}|\xi - \eta|$ and $k_n = (2\pi)^{-n}\Gamma(\frac{n}{2} + 1)$.
\end{example}

\begin{example}
Let $a : [0, \infty) \to (0, \infty)$ such that $a_0 \le a(t) \le a_1$ for all $t \ge 0$ with $a_1, a_0 > 0$.  Define $q : [0, \infty) \times \mathbb{R}^n \to (0, \infty)$ by $q(t, \xi) = a(t)f\big(|\xi|^2\big)$ where $f$ is a Bernstein function such that $f(0) = 0$.  Since $a(t)$ is positive for all $t \ge 0$ and $f\big(|\xi|^2\big)$ is clearly continuous negative definite we have by Theorem \ref{FTaSNDFT1} that $q(t, \xi)$ is continuous negative definite for all $t \ge 0$.  We also have that $q(t, \xi)$ is non-periodic, $q(t, \xi) = 0$ if and only if $\xi = 0$ and $\lim\inf_{|\xi| \to \infty}q(t, \xi) = a(t)\lim\inf_{\xi \to \infty}f\big(|\xi|^2\big) > 0$ since $f$ is strictly monotone increasing, hence $q(t, \cdot) \in \mathcal{MCN}(\mathbb{R}^n)$ for all $t \ge 0$.  We now apply Theorem \ref{ThesisMainResult} and get,
$$
p_{t, s}(0) \asymp \lambda^{(n)}\Big(B^{d_{Q_{t, s}}}\big(0, \sqrt{a_1/a_0}\big)\Big), \,\,\,\,\, t > s \ge 0,
$$
where $Q_{t, s}(\xi) = \int_s^tq(\tau, \xi)\,\mathrm{d}\tau$ and $p_{t, s}(x) = (2\pi)^{-n/2}\mathrm{F}^{-1}\big(e^{-Q_{t, s}(\cdot)}\big)(x)$ since clearly $a_0\psi(\xi) \le q(t, \xi) \le a_1\psi(\xi)$ for all $t \ge 0$.  Since $f$ is strictly monotone increasing it has an inverse and we note that,
$$
\begin{aligned}
B^{d_{Q_{t, s}}}\big(0, \sqrt{a_1/a_0}\big) & = \bigg\{\xi \in \mathbb{R}^n : \sqrt{Q_{t, s}(\xi)} < \sqrt{\frac{a_1}{a_0}}\bigg\}\\
& = \bigg\{\xi \in \mathbb{R}^n : f\big(|\xi|^2\big) < \frac{a_1}{a_0A(t, s)}\bigg\}\\
& = \bigg\{\xi \in \mathbb{R}^n : |\xi| < f^{-1}\Big(\frac{a_1}{a_0A(t, s)}\Big)\bigg\}\\
& = B\bigg(0, f^{-1}\Big(\frac{a_1}{a_0A(t, s)}\Big)\bigg),
\end{aligned}
$$
where $A(t, s) := \int_s^ta(\tau)\,\mathrm{d}\tau$.  Thus,
$$
p_{t, s}(0) \asymp \frac{\pi^{n/2}}{\Gamma(\frac{n}{2} + 1)}\bigg(f^{-1}\Big(\frac{a_1}{a_0A(t, s)}\Big)\bigg).
$$
\end{example}

We take further inspiration from \cite{Paper} and find that we can obtain a result similar to Theorem \ref{TDSGMeT3} for our situation.  Let $f : [0, \infty) \times (0, \infty) \to [0, \infty)$, $(t, x) \mapsto f(t, x)$ be a function such that for all $t \ge 0$ the function $f(t, \cdot) : (0, \infty) \to [0, \infty)$ is a Bernstein function with $f(t, 0) = 0$ and,
$$
\int_0^\infty e^{-\int_s^t f(\tau, r)\,\mathrm{d}\tau}\,\mathrm{d}r < \infty.
$$
We define,
\begin{equation}
F_{t, s}(x) := \int_s^t f(\tau, x)\,\mathrm{d}\tau\label{BF}.
\end{equation}

\begin{lemma}\label{TDSGMeL1}
$F_{t, s}$ is a Bernstein function for all $t \ge s \ge 0$.
\end{lemma}
\begin{proof}
Since $f(\tau, \cdot)$ is a Bernstein function for all $\tau \ge 0$, then $f(\tau, x)$ is a smooth function on $[s, t] \times (0, \infty)$ and therefore also on $[s, t] \times [a, b]$ where $[a, b] \subset (0, \infty)$.  Hence we may take derivatives under the integral sign and $F_{t, s} \in C^\infty\big((0, \infty)\big)$ for $t \ge s \ge 0$.  Also it is easy to see that $F_{t, s}(x) \ge 0$ for all $x \ge 0$ and $t \ge s \ge 0$.  Now for $k \in \mathbb{N}$ and $t \ge s \ge 0$,
$$
(-1)^k\frac{\partial^kF_{t, s}}{\partial x^k}(x) = \int_s^t(-1)^k\frac{\partial^kf}{\partial x^k}(\tau, x)\,\mathrm{d}\tau \le 0.
$$
\end{proof}

From this and by using Theorem \ref{TDSGMeT3} we get a simple corollary.
\begin{corollary}\label{TDSGMetcC1}
If for $t > s \ge 0$ we define,
\begin{equation}
p_{t, s}^f(x) := (2\pi)^{-n}\int_{\mathbb{R}^n}e^{ix\xi}e^{-\int_s^tf(\tau, |\xi|)\,\mathrm{d}\tau}\,\mathrm{d}\xi,\label{SubEx}
\end{equation}
where for all $\tau \ge 0$, $f(\tau, \cdot)$ is a Bernstein function with $f(\tau, 0) = 0$ and $e^{-\int_s^tf(\tau, \cdot)\,\mathrm{d}\tau} \in \mathcal{L}^1\big((0, \infty), \lambda\big)$.  Then there exists for all $t > s \ge 0$ a complete Bernstein function $g_{t, s}$ such that,
$$
\frac{p_{t, s}^f(x)}{p_{t, s}^f(0)} = e^{-g_{t, s}(|x|^2)}.
$$
\end{corollary}
\begin{proof}
We fix $t > s \ge 0$ and note that for $F_{t, s}$ as in \eqref{BF} we can write \eqref{SubEx} as,
$$
p_{t, s}^f(x) := (2\pi)^{-n}\int_{\mathbb{R}^n}e^{ix\xi}e^{-F_{t, s}(|\xi|)}\,\mathrm{d}\xi,
$$
since $e^{-F_{t, s}} \in \mathcal{L}^1\big((0, \infty), \lambda\big)$.  Now, we have that $F_{t, s}(0) = 0$ and by Lemma \ref{TDSGMeL1}, $F_{t, s}$ is a Bernstein function, thus $p_{t, s}^f$ is the transition density for a L\'evy process in $\mathbb{R}$ and is of the form $\mathrm{F}^{-1}e^{-F_{t, s}(|\cdot|)}(x)$.  Hence by Theorem \ref{TDSGMeT3}, there exists a complete Bernstein function $g_{t, s}$ such that,
$$
\frac{p_{t, s}^f(x)}{p_{t, s}^f(0)} = e^{-g_{t, s}(|x|^2)}.
$$
We note that also by Theorem \ref{TDSGMeT3}, $g_{t, s}$ is a complete Bernstein function.
\end{proof}

Corollary \ref{TDSGMetcC1} and Example \ref{TDSGMetcE2} seem to hint at the structure of transition densities of additive processes with time dependent symbols, and so we spend the rest of this section returning to the case studied in section \ref{Se.SMEg}.  Let $(X_t)_{t \ge 0}$ be a symmetric L\'evy process with continuous negative definite characteristic exponent $\psi : \mathbb{R}^n \to \mathbb{R}$.  We know its transition densities $(p_t)_{t \ge 0}$ are given by \eqref{LPTransitionDensities}.  Following section \ref{Se.SMEg} we derived a time dependent function,
$$
q(\tau, \xi) = -\frac{\partial}{\partial\tau}\ln\frac{p_\frac{1}{\tau}(\xi)}{p_\frac{1}{\tau}(0)},
$$
which is not known to be negative definite.  With this in mind we want to derive some interesting insights into dual processes (when they exist) using the framework we have developed throughout.

Firstly,
$$
Q_{t, s}(\xi) := \int_s^tq(\tau, \xi)\,\mathrm{d}\xi = -\ln\frac{p_\frac{1}{t}(\xi)}{p_\frac{1}{t}(0)} + \ln\frac{p_\frac{1}{s}(\xi)}{p_\frac{1}{s}(0)},
$$
and since $\ln$ is continuous we may use \cite[Theorem 14]{DProofPaper} again (see the proof of Proposition \ref{SharedPropWithNiels}) to calculate the limit,
$$
Q_t(\xi) := \lim_{s \to 0}Q_{t, s}(\xi) = -\ln\frac{p_\frac{1}{t}(\xi)}{p_\frac{1}{t}(0)},
$$
which is equivalent to,
$$
p_t(x) = p_t(0)e^{-d_{Q_{1/t}^2}(x, 0)},
$$
where $d_{Q_t}(\xi, \eta) := \sqrt{Q_t(\xi - \eta)}$.  In other words we have a representation for the proposed metric in Conjecture \ref{FAaSGRtLPConj1} which we note is not known to be a metric since we do not know if $q(\tau, \xi)$ is negative definite for all $\tau \ge 0$.  Noteworthy is that what governs the off-diagonal behaviour of the original L\'evy process is also what governs the diagonal behaviour of the dual process.

Next, combining Corollary \ref{TDSGMetcC1} and Example \ref{TDSGMetcE2} it is tempting to conjecture something similar to Conjecture \ref{FAaSGRtLPConj1} such as,
$$
p_{t, s}(x) = p_{t, s}(0)e^{-\frac{1}{t - s}d_\psi^2(x, 0)}.
$$
Essentially this is the opposite of what we have just discussed, namely what governs the off-diagonal behaviour of the dual process is what governs the diagonal behaviour of the original L\'evy process.  We consider,
$$
p_{t, s}(x) = (2\pi)^{-n}\int_{\mathbb{R}^n}e^{ix\cdot\xi}e^{-Q_{t, s}(\xi)}\,\mathrm{d}\xi,
$$
if we let $s \to 0$, then since $\psi : \mathbb{R}^n \to \mathbb{R}$ is even we know $\mathrm{F}^{-2}\big(e^{-\frac{1}{t}\psi(\cdot)}\big)(x) = e^{-\frac{1}{t}\psi(x)}$ and thus,
$$
\begin{aligned}
p_{t, 0}(x) & = (2\pi)^{-n}\int_{\mathbb{R}^n}e^{ix\cdot\xi}\frac{p_\frac{1}{t}(\xi)}{p_\frac{1}{t}(0)}\,\mathrm{d}\xi\\
& = \frac{1}{(2\pi)^{n/2}p_\frac{1}{t}(0)}\mathrm{F}^{-2}\big((2\pi)^{-n/2}e^{-\frac{1}{t}\psi(\cdot)}\big)(x)\\
& = \frac{e^{-\frac{1}{t}\psi(x)}}{(2\pi)^np_\frac{1}{t}(0)},
\end{aligned}
$$
as expected,  Thus we can write,
$$
p_{t, 0}(x) = p_{t, 0}(0)e^{-\frac{1}{t}d_\psi^2(x, 0)},
$$
since $p_{t, 0}(0) = (2\pi)^{-n}\frac{1}{p_\frac{1}{t}(0)}$.  Note that by Theorem \ref{FAaSGRtLPT1} we get,
$$
p_{t, 0}(0) \asymp \Big(\lambda^{(n)}\big(B^{d_{\psi, 1/t}}(0, 1)\big)\Big)^{-1},
$$
or in some sense the characteristic exponent $\psi$ of our original L\'evy process completely determines the behaviour of the transition densities of the dual process.

All of this hints at a larger idea, that being that the dual of the dual of a L\'evy process should be the original L\'evy process.  We can again follow the same idea as in \cite{Paper} and note that for $e^{-Q_t} \in \mathcal{L}^1(\mathbb{R}^n)$ we get,
$$
p_{t, 0}(0) = (2\pi)^{-n}\int_{\mathbb{R}^n}e^{-Q_t(\xi)}\,\mathrm{d}\xi \,\,\,\,\, \Leftrightarrow \,\,\,\,\, \int_{\mathbb{R}^n}\frac{e^{-Q_t(\xi)}}{(2\pi)^np_{t, 0}(0)}\,\mathrm{d}\xi = 1,
$$
or $\pi_t(x) = \frac{e^{-Q_t(x)}}{(2\pi)^np_{t, 0}(0)}$ is a transition density defining the dual of our dual.  Indeed we find,
$$
\pi_t(x) = \frac{e^{-Q_t(x)}}{(2\pi)^np_{t, 0}(0)} = p_\frac{1}{t}(0)\frac{p_\frac{1}{t}(x)}{p_\frac{1}{t}(0)} = p_\frac{1}{t}(x),
$$
which, as expected, is our original transition density with $t$ replaced by $1/t$. Again, by Theorem \ref{ThesisMainResult} we get,
$$
p_t(0) \asymp \bigg(\lambda^{(n)}\Big(B^{d_{Q_{1/t}}}\big(0, \sqrt{\beta_1/\beta_0}\big)\Big)\bigg)^{-1},
$$
or in some sense, $q(t, \xi)$ completely determines the behaviour of the transition densities of our original L\'evy process.


\chapter{Time Dependent Generators and Their Fundamental Solutions II: Spatially Inhomogeneous Symbols}

\section{Estimating Fundamental Solutions}

For the remainder of this thesis we would now like to take our time dependent symbols and make them also depend on the state space and use techniques used in L\'evy-type processes to gain some insight.  We now consider operators of the form,
\begin{equation}
q(t, x, D)u(t, x) = (2\pi)^{-n/2}\int_{\mathbb{R}^n}e^{ix\cdot\xi}q(t, x, \xi)\hat{u}(t, \xi)\,\mathrm{d}\xi,
\end{equation}
where initially $u(t, \cdot) \in \mathcal{S}(\mathbb{R}^n)$ for all $t \ge 0$.  We assume that $q : [0, \infty) \times \mathbb{R}^n \times \mathbb{R}^n \to \mathbb{C}$ is a locally bounded function such that for all $t \ge 0$ and $x \in \mathbb{R}^n$ the function $q(t, x, \cdot) : \mathbb{R}^n \to \mathbb{C}$ is negative definite and continuous.

\begin{definition}
We call a function $q : [0, \infty) \times \mathbb{R}^n \times \mathbb{R}^n \to \mathbb{C}$ a time dependent continuous negative definite symbol if $q$ is continuous and for all $t \ge 0$ and $x \in \mathbb{R}^n$, the function $q(t, x, \cdot) : \mathbb{R}^n \to \mathbb{C}$ is negative definite and continuous.
\end{definition}

For a time dependent negative definite symbol $q$ for every compact set $K \subset \mathbb{R}^n$, there exists a constant $C_K(t)$ such that,
$$
|q(t, x, \xi)| \le C_K(t)(1 + |\xi|^2),
$$
holds for all $x \in K$ and $\xi \in \mathbb{R}^n$.  We will add the additional assumption that we can find a bound for $C_K(t)$ independent of $t$, i.e. we will require for all $\xi \in \mathbb{R}^n$,
$$
|q(t, x, \xi)| \le C_K(1 + |\xi|^2),
$$
with $C_K$ independent of $t \ge 0$ and $x \in K$.  Furthermore we consider time dependent negative definite symbols that have the decomposition,
\begin{equation}
q(t, x, \xi) = q_1(t, \xi) + q_2(t, x, \xi).
\end{equation}
Note that this decomposition can be achieved when freezing the coefficients with respect to $x$, i.e.,
$$
q(t, x, \xi) = q(t, x_0, \xi) + \big(q(t, x, \xi) - q(t, x_0, \xi)\big).
$$

\begin{assumptions}\label{TDSTDSA1}
We assume that the function $q : [0, \infty) \times \mathbb{R}^n \times \mathbb{R}^n \to \mathbb{C}$ is a time dependent continuous negative definite symbol having the decomposition $q(t, x, \xi) = q_1(t, \xi) + q_2(t, x, \xi)$ into a continuous function $q_1 : [0, \infty) \times \mathbb{R}^n \to \mathbb{C}$ such that $q_1(t, \cdot) : \mathbb{R}^n \to \mathbb{C}$ is negative definite and a continuous function $q_2 : [0, \infty) \times \mathbb{R}^n \times \mathbb{R}^n \to \mathbb{C}$.  Further let $\psi : \mathbb{R}^n \to \mathbb{R}$ be a fixed continuous negative definite function that satisfies
\begin{equation}\label{GCfNDRF}
\psi(\xi) \ge c_0|\xi|^{r_0},
\end{equation}
for some $c_0, r_0 > 0$ and all $|\xi| \ge R > 0$.
\begin{itemize}
\item[A.1] For fixed $t \ge 0$, the symbol $q_1$ satisfies, for $\gamma_0 > 0$ and $\gamma_1, \gamma_2 \ge 0$ all independent of $t$, the estimates,
\begin{equation}
\gamma_0\psi(\xi) \le \text{Re}\,q_1(t, \xi) \le \gamma_1\psi(\xi), \,\,\, \text{for all } |\xi| \ge 1,\label{A11}
\end{equation}
and,
\begin{equation}
|\text{Im}\,q_1(t, \xi)| \le \gamma_2\text{Re}\,q_1(t, \xi), \,\,\, \text{for all } \xi \in \mathbb{R}^n.\label{A12}
\end{equation}
Note that \eqref{A11} and \eqref{A12} imply that for all $\xi \in \mathbb{R}^n$ and $t \ge 0$,
\begin{equation}
1 + \text{Re}\,q_1(t, \xi) \le 1 + |q_1(t, \xi)| \le \tau_1\big(1 + \psi(\xi)\big),\label{C11}
\end{equation}
for some $\tau_1 > 0$ independent of $t$ and,
\begin{equation}
1 + |\text{Im}\,q_1(t, \xi)| \le \tau_2\big(1 + \text{Re}\,q_1(t, \xi)\big),\label{C12}
\end{equation}
for some $\tau_2 > 0$ independent of $t$.

\item[A.2] For $m \in \mathbb{N}_0$ and $t \ge 0$, the function $x \mapsto q_2(t, x, \xi)$ belongs to $C^m(\mathbb{R}^n)$ and we have the estimate,
\begin{equation}
|\partial_x^\alpha q_2(t, x, \xi)| \le \varphi_\alpha(x)\big(1 + \psi(\xi)\big),\label{A21}
\end{equation}
for all $\alpha \in \mathbb{N}_0^n$, $|\alpha| \le m$, with functions $\varphi_\alpha \in \mathcal{L}^1(\mathbb{R}^n)$.
\end{itemize}
\end{assumptions}

Under Assumptions \ref{TDSTDSA1} with $m \ge n + \lfloor a \rfloor + 1$ for $a \ge 1$, it can be shown that $\big(-q(t_0, x, D), \mathcal{S}(\mathbb{R}^n)\big)$ is the generator of a sub-Markovian semigroup $\big(S_\sigma^{(t_0)}\big)_{\sigma \ge 0}$ on $\mathcal{L}^2(\mathbb{R}^n; \mathbb{R})$ and extends to the generator of a Feller semigroup $\big(T_\tau^{(t_0)}\big)_{\tau \ge 0}$ on $C_\infty(\mathbb{R}^n; \mathbb{R})$ for each fixed $t_0 \ge 0$ such that they coincide on $\mathcal{L}^2(\mathbb{R}^n; \mathbb{R}) \cap C_\infty(\mathbb{R}^n; \mathbb{R})$, i.e. $T_t^{(t_0)}\big|_{\mathcal{L}^2(\mathbb{R}^n; \mathbb{R}) \cap C_\infty(\mathbb{R}^n; \mathbb{R})} = S_t^{(t_0)}\big|_{\mathcal{L}^2(\mathbb{R}^n; \mathbb{R}) \cap C_\infty(\mathbb{R}^n; \mathbb{R})}$, $t \ge 0$.  We also get that a fundamental solution to,
\begin{equation}
\begin{cases}
\frac{\partial u}{\partial t}(t, x) + q(t, x, D)u(t, x) = 0, & 0 \le t \le T,\\
u(0, x) = u_0(x), & x \in \mathbb{R}^n,\label{IVP2}
\end{cases}
\end{equation}
exists \cite[Theorem 8.18]{RZ}.

By $U(t, s)$, $0 \le s \le t \le T$, we denote a fundamental solution to \eqref{IVP2}.  We also denote by $V(t, s)$, $0 \le s \le t \le T$, a fundamental solution to the initial value problem,
\begin{equation}
\begin{cases}
\frac{\partial u}{\partial t}(t, x) + q_1(t, D)u(t, x) = 0, & 0 \le t \le T,\\
u(0, x) = u_0(x), & x \in \mathbb{R}^n,\label{IVP3}
\end{cases}
\end{equation}
which by Proposition \ref{TDSFSP2} can be written as,
$$
V(t, s)v(x) = (2\pi)^{-n/2}\int_{\mathbb{R}^n}e^{ix\cdot\xi}e^{-\int_s^tq_1(\tau, \xi)\,\mathrm{d}\tau}\hat{v}(\xi)\,\mathrm{d}\xi, \,\,\, \text{for } v \in \mathcal{S}(\mathbb{R}^n).
$$
We prove a small lemma.

\begin{lemma}\label{TDSTDSL1}
Let $X$ be a Banach space and for each $t \ge 0$ let $A(t)$, $B(t)$ be operators with equal domains residing in $X$ that both generate a strongly continuous contraction semigroup.  Furthermore, let $U(t, s)$, $0 \le r \le s \le t \le T$, be a fundamental solution to the initial value problem,
$$
\begin{cases}
\frac{\partial u}{\partial t}(t) = A(t)u(t), & 0 \le r \le t \le T,\\
u(r) = u_0,
\end{cases}
$$
and $V(t, s)$, $0 \le r \le s \le t \le T$, be a fundamental solution to,
$$
\begin{cases}
\frac{\partial u}{\partial t}(t) = B(t)u(t), & 0 \le r \le t \le T,\\
u(r) = u_0.
\end{cases}
$$
Then,
$$
U(t, r)u_0 - V(t, r)u_0 = \int_r^tU(t, s)\big(A(s) - B(s)\big)V(s, r)u_0\,\mathrm{d}s.
$$
\end{lemma}
\begin{proof}
For $u_0 \in \mathcal{S}(\mathbb{R}^n)$ we have by the product rule,
$$
\begin{aligned}
\frac{\partial}{\partial s}\big(U(t, s)V(s, r)\big)u_0 & = \left(\frac{\partial}{\partial s}U(t, s)\right)V(s, r)u_0 + U(t, s)\left(\frac{\partial}{\partial s}V(s, r)\right)u_0\\
& = -U(t, s)A(s)V(s, r) + U(t, s)B(s)V(s, r)u_0\\
& = U(t, s)\big(B(s) - A(s)\big)V(s, r)u_0.
\end{aligned}
$$
Thus,
$$
\begin{aligned}
\int_r^tU(t, s)\big(B(s) - A(s)\big)V(s, r)u_0\,\mathrm{d}s & = U(t, s)V(s, r)u_0\,\bigg|_r^t\\
& = IV(t, r)u_0 - U(t, r)Iu_0,
\end{aligned}
$$
where $I$ is the identity operator.  Hence,
$$
U(t, r)u_0 - V(t, r)u_0 = \int_r^tU(t, s)\big(A(s) - B(s)\big)V(s, r)u_0\,\mathrm{d}s.
$$
\end{proof}

We now wish to compare $U(t, s)$ with $V(t, s)$ for $0 \le s \le t \le T$.  Under Assumptions \ref{TDSTDSA1}, $q_1(t, D)$ is a closed operator on $\mathcal{L}^2(\mathbb{R}^n)$ with domain $H^{\psi, 2}(\mathbb{R}^n)$ and we have for all $u \in H^{\psi, \tau}(\mathbb{R}^n)$,
\begin{equation}
\kappa_0^{\tau/2}\|u\|_{\psi, \tau} \le \|(1 + q_1(t, D))^{\tau/2} u\|_0 \le \kappa_1^{\tau/2}\|u\|_{\psi, \tau},
\end{equation}
for some $\kappa_0, \kappa_1 > 0$.  For the next proof we follow \cite[Theorem 3.1]{EstimatesPaper}.

\begin{theorem}
For $0 \le s \le t \le T$ let $U(t, s)$ and $V(t, s)$ be fundamental solutions to \eqref{IVP2} and \eqref{IVP3}, respectively.  Moreover, assume A.2 for some $m > n + 1$.  Then we have for $0 < t \le \frac{1}{\gamma_0}$ and $0 \le \rho < 1$,
\begin{equation}
\|U(t, s)u - V(t, s)u\|_0 \le c\frac{\sqrt{1 + \tau_2^2}}{\gamma_0^\rho}\frac{\tau_1}{\kappa_0}\frac{(t - s)^{1 - \rho}}{1 - \rho}\sum_{|\alpha| \le n + 1}\|\varphi_\alpha\|_{\mathcal{L}^1}\|u\|_{\psi, 2(1 - \rho)}.
\end{equation}
If $\rho = 0$, we have for all $t \ge 0$,
\begin{equation}
\|U(t, s)u - V(t, s)u\|_0 \le c(t - s)\sqrt{1 + \tau_2^2}\frac{\tau_1}{\kappa_0}\sum_{|\alpha| \le n + 1}\|\varphi_\alpha\|_{\mathcal{L}^1}\|u\|_{\psi, 2}.
\end{equation}
\end{theorem}
\begin{proof}
Since for each $t \ge 0$, $q(t, \cdot)$ is negative definite, then $-q(t, D)$ generates a sub-Markovian semigroup on $\mathcal{L}^2(\mathbb{R}^n; \mathbb{R})$ for all $t \ge 0$ and so by Lemma \ref{TDSTDSL1} we see that for $u \in H^{\psi, 2}(\mathbb{R}^n)$,
$$
\begin{aligned}
\left\|U(t, s)u - V(t, s)u\right\|_0 & = \left\|\int_s^tU(t, r)q_2(r, x, D)V(r, s)u\,\mathrm{d}r\right\|_0\\
& \le \int_s^t\left\|U(t, r)\right\|\left\|q_2(r, x, D)V(r, s)u\right\|_0\,\mathrm{d}r\\
& \le c\sum_{|\alpha| \le n + 1}\|\varphi_{\alpha}\|_{\mathcal{L}^1}\int_s^t\left\|V(r, s)u\right\|_{\psi, 2}\,\mathrm{d}r,
\end{aligned}
$$
where we have used \cite[Theorem 6.11]{RZ} i.e.,
$$
\|q_2(t, x, D)u(t, \cdot)\|_{\psi, s} \le c'\sum_{|\alpha| \le m}\|\varphi\|_\alpha\|u(t, \cdot)\|_{\psi, s + 2}, \,\,\, m \ge n + \lfloor s\rfloor + 1,
$$
which holds for all $s \ge 0$ and $u(t, \cdot) \in H^{\psi, s + 2}(\mathbb{R}^n)$ for each $t \ge 0$.  To estimate $\|V(r, s)u\|_{\psi, 2}$ we assume first that $0 < t \le \frac{1}{\gamma_0}$ and $0 \le \rho < 1$.  Then for $0 \le s < r \le t$,
$$
\begin{aligned}
& \,\,\,\,\,\,\,\, \|V(r, s)u\|_{\psi, 2}^2\\
& \le \frac{1}{\kappa_0^2}\big\|\big((1 + q_1(r, D)\big)V(r, s)u\big\|_0^2\\
& \le \frac{1}{\kappa_0^2}\int_{\mathbb{R}^n}\big|1 + q_1(r, \xi)\big|^2\Big|e^{-\int_s^rq_1(\sigma, \xi)\,\mathrm{d}\sigma}\Big|^2|\hat{u}(\xi)|^2\,\mathrm{d}\xi\\
& = \frac{1}{\kappa_0^2}\int_{\mathbb{R}^n}\Big(\big(1 + \text{Re}\,q_1(r, \xi)\big)^2 + \big(\text{Im}\,q_1(r, \xi)\big)^2\Big)\times\\
& \hspace{2.5cm} \times\Big(e^{-\int_s^r\text{Re}\,q_1(\sigma, \xi)\,\mathrm{d}\sigma}\Big)^2\Big|e^{-i\int_s^r\text{Im}\,q_1(\sigma, \xi)\,\mathrm{d}\sigma}\Big|^2|\hat{u}(\xi)|^2\,\mathrm{d}\xi\\
& \le \frac{1 + \tau_2^2}{\kappa_0^2}\int_{\mathbb{R}^n}\big(1 + \text{Re}\,q_1(r, \xi)\big)^2\frac{1}{\left(1 + \int_s^r\text{Re}\,q_1(\sigma, \xi)\,\mathrm{d}\sigma\right)^2}|\hat{u}(\xi)|^2\,\mathrm{d}\xi\\
& \le \frac{1 + \tau_2^2}{\kappa_0^2}\int_{\mathbb{R}^n}\big(1 + \text{Re}\,q_1(r, \xi)\big)^2\frac{1}{\left(1 + \gamma_0\int_s^r\psi(\xi)\,\mathrm{d}\sigma\right)^2}|\hat{u}(\xi)|^2\,\mathrm{d}\xi\\
& \le \frac{1 + \tau_2^2}{\kappa_0^2}\tau_1^2\int_{\mathbb{R}^n}\big(1 + \psi(\xi)\big)^2\frac{1}{\big(1 + \gamma_0(r - s)\psi(\xi)\big)^2}|\hat{u}(\xi)|^2\,\mathrm{d}\xi\\
& \le \frac{1}{(r - s)^{2\rho}}\frac{1 + \tau_2^2}{\gamma_0^{2\rho}}\frac{\tau_1^2}{\kappa_0^2}\int_{\mathbb{R}^n}\big(1 + \psi(\xi)\big)^{2(1 - \rho)}|\hat{u}(\xi)|^2\,\mathrm{d}\xi\\
& = \frac{1}{(r - s)^{2\rho}}\frac{1 + \tau_2^2}{\gamma_0^{2\rho}}\frac{\tau_1^2}{\kappa_0^2}\|u\|_{\psi, 2(1 - \rho)}^2.
\end{aligned}
$$
This gives,
$$
\begin{aligned}
\left\|U(t, s)u - V(t, s)u\right\|_0 & \le c\frac{\sqrt{1 + \tau_2^2}}{\gamma_0^\rho}\frac{\tau_1}{\kappa_0}\sum_{|\alpha| \le n + 1}\|\varphi_\alpha\|_{\mathcal{L}^1}\|u\|_{\psi, 2(1 - \rho)}\int_s^t\frac{\mathrm{d}r}{(r - s)^\rho}\\
& = c\frac{\sqrt{1 + \tau_2^2}}{\gamma_0^\rho}\frac{\tau_1}{\kappa_0}\frac{(t - s)^{1 - \rho}}{1 - \rho}\sum_{|\alpha| \le n + 1}\|\varphi_\alpha\|_{\mathcal{L}^1}\|u\|_{\psi, 2(1 - \rho)}.
\end{aligned}
$$
Similarly we calculate for $t \ge 0$,
$$
\begin{aligned}
\|V(r, s)u\|_{\psi, 2}^2 & \le \frac{1 + \tau_2^2}{\kappa_0^2}\int_{\mathbb{R}^n}\big(1 + \text{Re}\,q_1(r, \xi)\big)^2\Big(e^{-\int_s^r\text{Re}\,q_1(\sigma, \xi)\,\mathrm{d}\sigma}\Big)^2|\hat{u}(\xi)|^2\,\mathrm{d}\xi\\
& \le \frac{1 + \tau_2^2}{\kappa_0^2}\int_{\mathbb{R}^n}\big(1 + \text{Re}\,q_1(r, \xi)\big)^2|\hat{u}(\xi)|^2\,\mathrm{d}\xi\\
& \le \frac{1 + \tau_2^2}{\kappa_0^2}\tau_1^2\int_{\mathbb{R}^n}\big(1 + \psi(\xi)\big)^2|\hat{u}(\xi)|^2\,\mathrm{d}\xi\\
& = \frac{1 + \tau_2^2}{\kappa_0^2}\tau_1^2\|u\|_{\psi, 2}^2.
\end{aligned}
$$
Which yields,
$$
\left\|U(t, s)u - V(t, s)u\right\|_0 \le c(t - s)\sqrt{1 + \tau_2^2}\frac{\tau_1}{\kappa_0}\sum_{|\alpha| \le n + 1}\|\varphi_\alpha\|_{\mathcal{L}^1}\|u\|_{\psi, 2}.
$$
\end{proof}
\begin{remark}
From section \ref{Se.MaTD} we know that $V(t, s)u = p_{t, s} \ast u$, i.e. $V(t, s)$ is a convolution operator with,
$$
p_{t, s}(x) := (2\pi)^{-n}\int_{\mathbb{R}^n}e^{ix\cdot\xi}e^{-\int_s^tq_1(t, \xi)\,\mathrm{d}\tau}\,\mathrm{d}\xi.
$$
Also from our investigations in section \ref{Se.MaTD} we know that we can estimate $p_{t, s}$ in terms of metrics and combining this knowledge with the above theorem allows us to get a control over additive-type processes using the geometry previously developed in the chapter.
\end{remark}



\section{The Symbol of an Additive-type Process}

In this final section we want to construct the symbol for $U(t, s)$, a fundamental solution to \eqref{IVP2}.  Firstly we want a kernel representation for $U(t, s)$ by showing that $U(t, s)$ satisfies the conditions of the Riesz Representation Theorem and for this purpose we follow \cite{KernelPaper}, modifying arguments to fit our case.  Once we have a kernel representation we can follow proofs given in \cite{SymbolPaper} and \cite{Conservative} to construct the symbol of $U(t, s)$.  Throughout this section we assume that our time dependent symbol $q$ has the property $q(t, x, 0) = 0$ for all $x \in \mathbb{R}^n$ and $t \ge 0$.  Since $\mathcal{S}(\mathbb{R}^n)$ is dense in $C_\infty(\mathbb{R}^n)$ we can extend the domain of $U(t, s)$ by continuity to include $C_\infty(\mathbb{R}^n)$, this extension of $U(t, s)$ we again denote by $U(t, s)$.  We start with some preparation.
\begin{definition}
Let $A_1, A_2, \dots, A_N, B$ be topological spaces.  A mapping $f : A_1 \times A_2 \times \dots \times A_N \to B$ is called \emph{jointly continuous} if $f$ is a continuous map from $A_1 \times A_2 \times \dots \times A_N$ equipped with the product topology.
\end{definition}

\begin{lemma}\label{TDSSoAPL1}
Let $u$ be a solution to \eqref{IVP2} with a positive initial condition $u_0$.  If $u$ is jointly continuous and,
$$
\lim_{|x| \to \infty}\sup_{t \in [s, T]}|u(t, x)| = 0, \,\,\, \text{for} \,\,\, s \in [0, T),
$$
then $u$ is real-valued (i.e. finite), unique and $u(t, x) \ge 0$ for all $t \in [s, T]$ and $x \in \mathbb{R}^n$.
\end{lemma}
\begin{proof}
First suppose that $u$ is real and fix $s \in [0, T)$.  Select $\chi \in C_0^\infty(\mathbb{R}^n)$ such that $0 \le \chi \le 1$, $\chi(x) = 1$ for $|x| < 1$ and $\chi(x) = 0$ for $|x| > 2$.  Now assume that,
$$
\alpha := \inf_{(t, x) \in [s, T] \times \mathbb{R}^n}u(t, x) < 0.
$$
By $\lim_{|x| \to \infty}\sup_{t \in [s, T]}u(t, x) = 0$, it is obvious that there exists an $R > 0$ such that $|u(t, x)| < |\alpha|/2$ for all $t \in [s, T]$ and $|x| > R$.  Thus the infimum is actually a minimum attained at some point $(t_0, x_0) \in [s, T] \times [-R, R]^n$ since $u$ is jointly continuous and $u(0, x) = u_0(x) \ge 0$ for all $x \in \mathbb{R}^n$.

Define for $k > R$,
$$
v_k(t, x) := u(t, x) + \frac{T - t}{T - s}\frac{|\alpha|}{2}\chi\left(\frac{x}{k}\right).
$$
Now,
$$
\inf_{(t, x) \in [s, T] \times [-R, R]^n}v_k(t, x) \le \alpha + \frac{T - t_0}{T - s}\frac{|\alpha|}{2}\chi\left(\frac{x_0}{k}\right) \le -\frac{|\alpha|}{2},
$$
holds and this infimum is again a minimum at some point $(t_1, x_1) \in [s, T) \times [-R, R]^n$ since $v_k(t, x) > -|\alpha|/2$ for $t \in [s, T]$ and $|x| > R$.  Moreover, $\frac{\partial}{\partial t}u(t, x)$ exists since $u$ is a solution to \eqref{IVP2}, thus the property of the minimum and $t \in [s, T)$ imply that,
$$
\frac{\partial}{\partial t}v_k(t, x)\bigg|_{(t, x) = (t_1, x_1)} \ge 0.
$$
Therefore, on the one hand,
$$
\begin{aligned}
\left(\frac{\partial}{\partial t} + q(t, x, D)\right)v_k(t, x)\bigg|_{(t, x) = (t_1, x_1)} \ge q(t_1, x_1, D)v_k(t_1, x_1) \ge 0,
\end{aligned}
$$
holds since $-q(t, x, D)$ satisfies the positive maximum principle.  On the other hand for $k$ large,
$$
\begin{aligned}
& \left(\frac{\partial}{\partial t} + q(t, x, D)\right)v_k(t, x)\bigg|_{(t, x) = (t_1, x_1)}\\
\le & \, -\frac{|\alpha|}{2T}\chi\left(\frac{x_1}{k}\right) + \frac{T - t}{T - s}\frac{|\alpha|}{2}q(t, x, D)\chi\left(\frac{x}{k}\right)\bigg|_{(t, x) = (t_1, x_1)} < 0,
\end{aligned}
$$
holds.  The negativity follows by dominated convergence as,
$$
\left|q(t, x, D)\chi\left(\frac{x}{k}\right)\right| = \left|(2\pi)^{-n/2}\int_{\mathbb{R}^n}e^{ix\cdot\eta/k}q\left(t, x, \frac{\eta}{k}\right)\hat{\chi}(\eta)\,\mathrm{d}\eta\right| \to 0
$$
as $k \to \infty$ since for each compact $K \subset \mathbb{R}^n$, $q(t, x, \xi) \le C_K\big(1 + |\xi|^2\big)$ for all $x \in K$ and $t \ge 0$.  By the above inequalities we get a contradiction and thus $u(t, x) \ge 0$.

To show the uniqueness, suppose that $v$ is another real solution such that,
$$
\lim_{|x| \to \infty}\sup_{t \in [s, T]}|v(t, x)| = 0.
$$
Then,
$$
\lim_{|x| \to \infty}\big(v(t, x) - u(t, x)\big) = 0 \,\,\,\,\, \text{and} \,\,\,\,\, v(0, x) - u(0, x) = 0,
$$
and thus $v - u$ and similarly $u - v$ are both positive real solutions.  Hovever, this implies that $u = v$.  Finally, suppose that $u$ is a complex-valued solution to \eqref{IVP2} with real initial value $u_0(x)$, such that $\lim_{|x| \to \infty}\sup_{t \in [s, T]}|u(t, x)| = 0$.  Then since $q(t, x, D)$ maps real-valued functions to real-valued functions and by linearity it follows that $\text{Im}\,u$ is the solution to \eqref{IVP2} with $f \equiv 0$ and initial value $0$.  It is also easy to see that $\lim_{|x| \to \infty}\sup_{t \in [s, T]}|\text{Im}\,u(t, x)| = 0$.  Thus, $\text{Im}\,u(t, x) \ge 0$ for all $t \in [s, T]$ and $x \in \mathbb{R}^n$ and furthermore by Remark \ref{TDSFSR1} we may write $\text{Im}\,u$ as,
$$
\text{Im}\,u(t, x) = U(t, s)0,
$$
but $U(t, s)$ is a bounded operator and so,
$$
0 \le |\text{Im}\,u(t, x)| = |U(t, s)0| \le c\|0\|_\infty = 0,
$$
i.e. $\text{Im}\,u(t, x) \equiv 0$.
\end{proof}

This next result allows us to give our fundamental solution to \eqref{IVP2} a kernel representation which we use throughout this section.
\begin{theorem}\label{TDSSoAPT1}
Let $U(t, s)$ be a fundamental solution to \eqref{IVP2}, then under Assumptions \ref{TDSTDSA1} we have that $U(t, s)u_0 \ge 0$ for $0 \le s \le t$ and all a.e.-positive $u_0 \in H^{\psi, \sigma + 2}(\mathbb{R}^n)$, $\sigma \ge 0$.
\end{theorem}
\begin{proof}
Firstly for a.e.-positive $u_0 \in H^{\psi, \sigma + 2}(\mathbb{R}^n)$, clearly $U(t, s)u_0$ solves \eqref{IVP2}.  Next for each $t, \tau \in [0, T]$ and $s \in [0, t]$ we get,
$$
\begin{aligned}
\left\|U(t, s)u_0 - U(\tau, s)u_0\right\|_\infty & = \left\|\int_\tau ^t\frac{\partial}{\partial r}U(r, s)u_0\,\mathrm{d}r\right\|_\infty\\
& = \left\|\int_\tau ^tq(r, \cdot, D)U(r, s)u_0\,\mathrm{d}r\right\|_\infty\\
& \le \int_\tau^t\left\|q(r, \cdot, D)U(r, s)u_0\right\|_\infty\,\mathrm{d}r,
\end{aligned}
$$
then by \cite[Section 7]{RZ} for $\sigma > n/r_0$ (compare \eqref{GCfNDRF}),
$$
\left\|U(t, s)u_0 - U(\tau, s)u_0\right\|_\infty \le c_{\sigma, r_0, n}\int_\tau^t\left\|q(r, \cdot, D)U(r, s)u_0\right\|_{\psi, \sigma}\,\mathrm{d}r\\
$$
and \cite[Proposition 6.6 + Theorem 6.11]{RZ},
$$
\begin{aligned}
\left\|U(t, s)u_0 - U(\tau, s)u_0\right\|_\infty & \le c_{\sigma, r_0, n}\tilde{c}_{\gamma_1, n, m, \sigma, \psi}\int_\tau^t\left\|U(r, s)u_0\right\|_{\psi, \sigma + 2}\,\mathrm{d}r\\
& \le c_{\sigma, r_0, n}\tilde{c}_{\gamma_1, n, m, \sigma, \psi}c'\int_\tau ^t\left\|u_0\right\|_{\psi, \sigma + 2}\,\mathrm{d}r\\
& = c''|t - \tau|\left\|u_0\right\|_{\psi, \sigma + 2},
\end{aligned}
$$
holds where $c'$ is the operator norm of $U(t, s)$ and $c'' := c_{\sigma, r_0, n}\tilde{c}_{\gamma_1, n, m, \sigma, \psi}c'$.  Furthermore, for $ r \in [s, t]$, $U(r, s)$ maps $H^{\psi, \sigma + 2}(\mathbb{R}^n)$ into itself, and this space is, for $\sigma + 2 > n/r_0$, a subset of $C_\infty(\mathbb{R}^n)$ \cite[Section 2.6]{Vol2}, i.e. $U(t, s)u_0 \in C_\infty(\mathbb{R}^n)$.  It follows that $(t, x) \mapsto U(t, s)u_0(x)$ is jointly continuous.

Secondly, for $\varepsilon > 0$ set,
$$
M := \left\{s + k\frac{\varepsilon}{2\tilde{c}'\|u_0\|_{\psi, 2}} : k \in \mathbb{N}_0\right\} \cap [s, T].
$$
Then for each $s \in [0, t]$ there exists an $R > 0$ such that $|U(t, s)u_0(x)| < \varepsilon/2$ for all $t \in M$ and $|x| > R$.  Thus for $|y| > R$ and $r \in [s, T]$,
$$
|U(r, s)u_0(y)| \le |U(r, s)u_0(y) - U(r', s)u_0(y)| + |U(r', s)u_0(y)| < \frac{\varepsilon}{2} + \frac{\varepsilon}{2} = \varepsilon,
$$
for $r' \in M$ with $|r - r'| < \varepsilon/2\tilde{c}'\|u_0\|_{\psi, \sigma + 2}$.  Thus,
$$
\lim_{|x| \to \infty}\sup_{t \in [s, T]}|U(t, s)u_0(x)| = 0,
$$
and so the positivity property follows from Lemma \ref{TDSSoAPL1}.
\end{proof}

In Theorem \ref{TDSSoAPT1} we see that $U(t, s)$ maps $H^{\psi, \sigma + 2}(\mathbb{R}^n)$ into itself and for $\sigma + 2 > n/r_0$ the space $H^{\psi, \sigma + 2}(\mathbb{R}^n)$ is linearly embedded in $C_\infty(\mathbb{R}^n)$ hence $U(t, s)$ is a linear operator on $C_\infty(\mathbb{R}^n)$.  Also by Theorem \ref{TDSSoAPT1} it is positivity preserving, thus by the Riesz Representation Theorem \cite[Theorem 2.3.4.C]{Vol1}, there exists a unique Borel measure $p_{t, s}(x, \cdot)$ on $\mathbb{R}^n$ such that for $u \in C_\infty(\mathbb{R}^n)$,
\begin{equation}
U(t, s)u(x) = \int_{\mathbb{R}^n}u(y)\,p_{t, s}(x, \mathrm{d}y),
\end{equation}
where,
\begin{equation}\label{BOoVaI}
\int_{\mathbb{R}^n}u(y)p_{t, s}(x, \mathrm{d}y) < \infty.
\end{equation}

We can now extend the domain of $U(t, s)$ again to include the bounded Borel measurable functions by taking a sequence of bounded functions in $C_\infty(\mathbb{R}^n)$ that converge to a function in $B_b(\mathbb{R}^n)$ via pointwise monotone convergence.  Again we denote this extension of $U(t, s)$ as $U(t, s)$.

We also need \eqref{BOoVaI} to hold for all $u \in B_b(\mathbb{R}^n)$ and so to continue we suppose that,
\begin{equation}
\big|\sigma\big(U(t, s)\big)\big| \le \kappa,\label{SymbolClassEstimate}
\end{equation}
holds where $\sigma\big(U(t, s)\big)$ denotes the symbol of $U(t, s)$ and $\kappa > 0$ is a constant independent of $t$ and $s$.  Note that in \cite{KernelPaper} the author proves this estimate for similar symbols with the additional condition of being elliptic uniformly in $t$, from this we believe the estimate \eqref{SymbolClassEstimate} should hold in a more general case.
\begin{lemma}\label{TDSSoAPL1.5}
Under Assumptions \ref{TDSTDSA1} with $m \ge n + \lfloor a \rfloor + 1$ for $a \ge 1$, we have that,
\begin{equation}
\|U(t, s)u\|_\infty \le c\|u\|_\infty,
\end{equation}
for some constant $c > 0$ and all real-valued $u \in H^{\psi, \sigma + 2}(\mathbb{R}^n)$ for $\sigma > n/r_0$.
\end{lemma}
\begin{proof}
For $k \in \mathbb{N}$ let $\phi_k(x) := e^{-|x|^2/k^2}$ then $\phi_k \in \mathcal{S}(\mathbb{R}^n)$.  We find that $\hat{\phi}_k = k^ne^{-(k\xi)^2/2}$ and so,
$$
\begin{aligned}
0 \le U(t, s)\phi_k(x) & = (2\pi)^{-n/2}\bigg|\int_{\mathbb{R}^n}e^{ix\cdot\xi}\sigma\big(U(t, s)\big)\hat{\phi}_k(\xi)\,\mathrm{d}\xi\bigg|\\
& \le \kappa(2\pi)^{-n/2}\int_{\mathbb{R}^n}k^ne^{-(k\xi)^2/2}\,\mathrm{d}\xi = \kappa
\end{aligned}
$$
Next let $g \in C_0^\infty(\mathbb{R}^n)$, then there exists $k \in \mathbb{N}$ such that $g/\|g\|_\infty \le 2\phi_k$ and since $C_0^\infty(\mathbb{R}^n) \subset H^{\psi, \sigma}(\mathbb{R}^n)$ we see that $2\phi_k - g/\|g\|_\infty \in H^{\psi, \sigma + 2}(\mathbb{R}^n)$.  By Theorem \ref{TDSSoAPT1}, $U(t, s)$ is positivity preserving for a.e.-positive functions in $H^{\psi, \sigma + 2}(\mathbb{R}^n)$ and so,
$$
0 \le U(t, s)\bigg(2\phi_k(x) - \frac{g(x)}{\|g\|_\infty}\bigg),
$$
then by linearity,
$$
U(t, s)\frac{g(x)}{\|g\|_\infty} \le 2U(t, s)\phi_k(x) \le 2\kappa,
$$
or,
$$
\|U(t, s)g\|_\infty \le c\|g\|_\infty.
$$

All that remains is to extend the inequality from $C_0^\infty(\mathbb{R}^n)$ to $H^{\psi, \sigma + 2}(\mathbb{R}^n)$.  Since $C_0^\infty(\mathbb{R}^n)$ is dense in $H^{\psi, \sigma + 2}(\mathbb{R}^n)$, for $u \in H^{\psi, \sigma + 2}(\mathbb{R}^n)$ there exists a sequence $(g_k)_{k \in \mathbb{N}}$ in $C_0^\infty(\mathbb{R}^n)$ such that $g_k \to u$ uniformly as $k \to \infty$.  Then by \cite[Section 7]{RZ},
$$
\begin{aligned}
\|U(t, s)u\|_\infty & \le \|U(t, s)(u - g_k)\|_\infty + \|U(t, s)g_k\|_\infty\\
& \le c'\|u - g_k\|_{\psi, \sigma + 2} + c\|g_k\|_\infty\\
& \le 2\tilde{c}\|u - g_k\|_{\psi, \sigma + 2} + c\|u\|_\infty,
\end{aligned}
$$
and taking the limit as $k \to \infty$ gives the result.
\end{proof}

We now show that the constant in Lemma \ref{TDSSoAPL1.5} is 1.
\begin{theorem}
Under Assumptions \ref{TDSTDSA1} with $m \ge n + \lfloor a \rfloor + 1$ for $a \ge 1$, we have that,
\begin{equation}
U(t, s)1 := \sup_{k \in \mathbb{N}}U(t, s)u_k = 1,
\end{equation}
for real-valued $u_k \in H^{\psi, \sigma}(\mathbb{R}^n)$ where $u_k \uparrow 1$, that is $u_k$ is point-wise monotone increasing to $1$.  Furthermore,
\begin{equation}\label{UBiSN}
\|U(t, s)u\|_\infty \le \|u\|_\infty,
\end{equation}
holds for all real-valued $u \in H^{\psi, \sigma + 2}(\mathbb{R}^n)$.
\end{theorem}
\begin{proof}
Define $\chi \in C_0^\infty(\mathbb{R}^n)$ such that $\chi(x) = 1$ for $|x| < 1$, $0 \le \chi(x) \le 1$ for $1 \le |x| \le 2$ and $\chi(x) = 0$ for $|x| > 2$.  If we consider the initial value problem,
$$
\begin{cases}
\frac{\partial u}{\partial t}(t, x) + q(t, x, D)u(t, x) = q(t, x, D)\chi\big(\frac{x}{k}\big), & 0 \le s \le t \le T,\\
\hspace{3.4cm} u(s, x) = \chi\big(\frac{x}{k}\big), & x \in \mathbb{R}^n,
\end{cases}
$$
then clearly this has the solution $u(t, x) = \chi\big(\frac{x}{k}\big)$ which we can write as,
$$
\chi\Big(\frac{x}{k}\Big) = U(t, s)\chi\Big(\frac{x}{k}\Big) + \int_s^tU(t, r)q(r, x, D)\chi\Big(\frac{x}{k}\Big)\,\mathrm{d}r.
$$
Using the fact that $U(t, s)$ is a bounded operator we see that,
$$
\begin{aligned}
\bigg|\int_s^tU(t, r)q(r, x, D)\chi\Big(\frac{x}{k}\Big)\,\mathrm{d}r\bigg| & \le (t - s)\sup_{r \in [s, t]}\Big|U(t, r)q(r, x, D)\chi\Big(\frac{x}{k}\Big)\Big|\\
& \le \kappa(t - s)\sup_{r \in [s, t]}\Big|q(r, x, D)\chi\Big(\frac{x}{k}\Big)\Big| \to 0
\end{aligned}
$$
as $k \to \infty$ by the same argument as in Lemma \ref{TDSSoAPL1}.  Thus,
$$
\lim_{k \to \infty}U(t, s)\chi\Big(\frac{x}{k}\Big) = \lim_{k \to \infty}\chi\Big(\frac{x}{k}\Big) = 1.
$$
For $u_k \in H^{\psi, \sigma + 2}(\mathbb{R}^n)$ with $u_k \uparrow 1$ it holds that $u_k \cdot \chi\big(\frac{\cdot}{l}\big) \in H^{\psi, \sigma + 2}(\mathbb{R}^n)$ also and so,
$$
\sup_{k \in \mathbb{N}}U(t, s)u_k(x) = \sup_{k, l \in \mathbb{N}}U(t, s)\bigg(u_k(x)\chi\Big(\frac{x}{l}\Big)\bigg) = \sup_{l \in \mathbb{N}}U(t, s)\chi\Big(\frac{x}{l}\Big) = 1.
$$
Finally, for each $g \in C_0^\infty(\mathbb{R}^n)$ there exists a $k \in \mathbb{N}$ such that $g/\|g\|_\infty \le \chi\big(\frac{x}{k}\big)$, then by a similar argument to Lemma \ref{TDSSoAPL1.5},
$$
U(t, s)\frac{g(x)}{\|g\|_\infty} \le U(t, s)\chi\Big(\frac{x}{k}\Big) \le \sup_{k \in \mathbb{N}}U(t, s)\chi\Big(\frac{x}{k}\Big) = 1,
$$
or,
$$
\|U(t, s)g\|_\infty \le \|g\|_\infty,
$$
which can be extended to $H^{\psi, \sigma + 2}(\mathbb{R}^n)$ in the same way as in Lemma \ref{TDSSoAPL1.5}.
\end{proof}
\begin{remark}
From $U(t, s)1 = 1$ we can conclude that our positive Borel measures $p_{t, s}(x, \cdot)$ are probability measures and due to the properties,
$$
U(t, s) = U(t, r)U(r, s), \text{ } 0 \le s \le r \le t \text{ and } U(t, t) = I,
$$
they define a projective limit.  Thus by Kolmogorov's existence theorem, a corresponding canonical process exists.  We note that this process is a Markov process that is in general time and space inhomogeneous.
\end{remark}

Now that \eqref{UBiSN} holds for all bounded Borel functions and $e_\xi$ is in the domain of $U(t, s)$ where $e_\xi(x) := e^{ix\cdot\xi}$ for all $x, \xi \in \mathbb{R}^n$, for $t \ge s \ge 0$ we can define the function $\lambda_{t, s} : \mathbb{R}^n \times \mathbb{R}^n \to \mathbb{C}$ given by,
\begin{equation}
\lambda_{t, s}(x, \xi) := e_{-\xi}(x)U(t, s)e_\xi(x).
\end{equation}
The proof of the next result follows a similar proof given in \cite{SymbolPaper}, see also \cite{Conservative}.

\begin{theorem}
For any $u \in \mathcal{S}(\mathbb{R}^n)$ we have,
\begin{equation}
U(t, s)u(x) = (2\pi)^{-n/2}\int_{\mathbb{R}^n}e^{ix\cdot\xi}\lambda_{t, s}(x, \xi)\hat{u}(\xi)\,\mathrm{d}\xi,
\end{equation}
i.e. on $\mathcal{S}(\mathbb{R}^n)$ the operator $U(t, s)$ is a pseudo-differential operator with symbol $\lambda_{t, s}(x, \xi)$.
\end{theorem}
\begin{proof}
For $\nu \in \mathbb{N}$ let $(j_\nu)_{\nu > 0}$ be a family of functions such that $\hat{j}_\nu \to \varepsilon_0$ and $j_\nu \to (2\pi)^{-n/2}$ pointwise as $\nu \to 0$.  In addition, suppose that $j_\nu \in \mathcal{S}(\mathbb{R}^n)$, which implies that $\hat{j}_\nu \in \mathcal{S}(\mathbb{R}^n)$.  Since,
$$
|\lambda_{t, s}(x, \xi)| \le \int_{\mathbb{R}^n}p_{t, s}(x, \mathrm{d}y) < \infty,
$$
it follows that,
$$
\begin{aligned}
& \int_{\mathbb{R}^n}e^{ix\cdot\xi}\lambda_{t, s}(x, \xi)j_\nu(\xi)\hat{u}(\xi)\,\mathrm{d}\xi\\
= & \, (2\pi)^{-n/2}\int_{\mathbb{R}^n}\int_{\mathbb{R}^n}e^{ix\cdot\xi}\lambda_{t, s}(x, \xi)j_\nu(\xi)e^{-iy\cdot\xi}u(y)\,\mathrm{d}y\,\mathrm{d}\xi\\
= & \, (2\pi)^{-n/2}\int_{\mathbb{R}^n}\int_{\mathbb{R}^n}e^{-iy\cdot\xi}U(t, s)e_\xi(x)j_\nu(\xi)\,\mathrm{d}\xi\,u(y)\,\mathrm{d}y\\
= & \, (2\pi)^{-n/2}\int_{\mathbb{R}^n}\int_{\mathbb{R}^n}e^{-iy\cdot\xi}\int_{\mathbb{R}^n}e^{iz\cdot\xi}\,p_{t, s}(x, \mathrm{d}z)\,j_\nu(\xi)\,\mathrm{d}\xi\,u(y)\,\mathrm{d}y\\
= & \, (2\pi)^{-n/2}\int_{\mathbb{R}^n}\int_{\mathbb{R}^n}\int_{\mathbb{R}^n}e^{-i(y - z)\cdot\xi}j_\nu(\xi)\,\mathrm{d}\xi\,p_{t, s}(x, \mathrm{d}z)\,u(y)\,\mathrm{d}y\\
= & \int_{\mathbb{R}^n}\int_{\mathbb{R}^n}\hat{j}(y - z)\,p_{t, s}(x, \mathrm{d}z)\,u(y)\,\mathrm{d}y.
\end{aligned}
$$
Passing the limit as $\nu \to 0$ we find that,
$$
\begin{aligned}
(2\pi)^{-n/2}\int_{\mathbb{R}^n}e^{ix\cdot\xi}\lambda_{t, s}(x, \xi)\hat{u}(\xi)\,\mathrm{d}\xi & = \lim_{\nu \to 0}\int_{\mathbb{R}^n}e^{ix\cdot\xi}\lambda_{t, s}j_\nu(\xi)\hat{u}(\xi)\,\mathrm{d}\xi\\
& = \lim_{\nu \to 0}\int_{\mathbb{R}^n}\int_{\mathbb{R}^n}\hat{j}_\nu(y - z)\,p_{t, s}(x, \mathrm{d}z)\,u(y)\,\mathrm{d}y\\
& = \int_{\mathbb{R}^n}u(z)\,p_{t, s}(x, \mathrm{d}z)\\
& = U(t, s)u(x).
\end{aligned}
$$
\end{proof}

Next we would like to show that under some assumptions we get,
\begin{equation}
\frac{\partial}{\partial s}\lambda_{t, s}(x, \xi)\bigg|_{s = t} = q(t, x, \xi),
\end{equation}
as one might expect from the definition of a fundamental solution, however we first need some lemmas.  For the rest of the section we follow \cite{Conservative}.  Consider a function $\chi_1 \in C_0^\infty(\mathbb{R}^n)$ such that $\chi_{B_1(0)} \le \chi_1 \le \chi_{B_2(0)}$, where $B_r(0)$ is the standard Euclidean ball of radius $r > 0$ in $\mathbb{R}^n$.  We set $\chi_k(x) := \chi_1(x/k)$, then clearly $\chi_k \to 1$ as $k \to \infty$ and we note that,
$$
\begin{aligned}
\hat{\chi}_k(\xi) & = (2\pi)^{-n/2}\int_{\mathbb{R}^n}e^{i\xi\cdot x}\chi_1(x/k)\,\mathrm{d}x\\
& = (2\pi)^{-n/2}\int_{\mathbb{R}^n}e^{ik\xi\cdot y}\chi_1(y)k^n\,\mathrm{d}y\\
& = k^n\hat{\chi}_1(k\xi).
\end{aligned}
$$

\begin{lemma}\label{TDSSoAPL3}
Under Assumptions \ref{TDSTDSA1} we get that,
$$
\lim_{k \to \infty}U(t, s)q_2(s, x, D)\big(e_\xi\chi_k\big)(x) = U(t, s)\big(q_2(s, \cdot, \xi)e_\xi\big)(x),
$$
holds true and for every $\xi \in \mathbb{R}^n$,
$$
\big|U(t, s)q_2(s, x, D)\big(e_\xi\chi_k\big)(x)\big| \le c_\xi,
$$
uniformly in $x \in \mathbb{R}^n$, $t \ge s \ge 0$ and $k \in \mathbb{N}$.
\end{lemma}
\begin{proof}
Firstly,
$$
\big(e_\xi\chi_k\big)^\wedge(y) = (2\pi)^{-n/2}\int_{\mathbb{R}^n}e^{-i(y - \xi)\cdot z}\chi_k(x)\,\mathrm{d}z = \hat{\chi}_k(y - \xi) = k^n\hat{\chi}_1\big(k(y - \xi)\big)
$$
thus,
$$
\begin{aligned}
q_2(s, x, D)\big(e_\xi\chi_k\big)(x) & = (2\pi)^{-n/2}\int_{\mathbb{R}^n}e^{ix\cdot y}q_2(s, x, y)\hat{\chi}_1\big(k(y - \xi)\big)k^n\,\mathrm{d}y\\
& = (2\pi)^{-n/2}\int_{\mathbb{R}^n}e^{ix\cdot(\xi + \rho/k)}q_2\Big(s, x, \xi + \frac{\rho}{k}\Big)\hat{\chi}_1(\rho)\,\mathrm{d}\rho,
\end{aligned}
$$
which yields,
$$
\begin{aligned}
& \, \mathrm{F}_{x \mapsto \eta}\big(q_2(s, \cdot, D)(e_\xi\chi_k)\big)(\eta)\\
= & \, (2\pi)^{-n/2}\int_{\mathbb{R}^n}(2\pi)^{-n/2}\int_{\mathbb{R}^n}e^{-i(\eta - \xi - \rho/k)\cdot x}q_2\Big(s, x, \xi + \frac{\rho}{k}\Big)\hat{\chi}_1(\rho)\,\mathrm{d}\rho\,\mathrm{d}x\\
= & \, (2\pi)^{-n/2}\int_{\mathbb{R}^n}\tilde{q}_2\Big(s, \eta - \xi - \frac{\rho}{k}, \xi + \frac{\rho}{k}\Big)\hat{\chi}_1(\rho)\,\mathrm{d}\rho
\end{aligned}
$$
where,
$$
\tilde{q}_2(s, x, \xi) := (2\pi)^{-n/2}\int_{\mathbb{R}^n}e^{-ix\cdot y}q_2(s, y, \xi)\,\mathrm{d}y.
$$
Changing the order of integration in the above calculation is justified since $\hat{\chi}_1$ is rapidly decreasing and,
$$
\begin{aligned}
\Big|q_2\Big(s, x, \xi + \frac{\rho}{k}\Big)\Big| & \le \varphi_0(x)\bigg(1 + \phi\Big(\xi + \frac{\rho}{k}\Big)\bigg)\\
& \le c_\psi\varphi_0(x)\bigg(1 + \Big|\xi + \frac{\rho}{k}\Big|^2\bigg)\\
& \le 2c_\psi\varphi_0(x)\big(1 + |\xi|^2\big)\bigg(1 + \Big|\frac{\rho}{k}\Big|^2\bigg).
\end{aligned}
$$
Since $U(t, s)$ is a pseudo-differential operator with symbol $\lambda_{t, s}(x, \xi)$ we get,
$$
\begin{aligned}
& \, U(t, s)q_2(s, x, D)\big(e_\xi\chi_k\big)(x)\\
= & \, (2\pi)^{-n/2}\int_{\mathbb{R}^n}(2\pi)^{-n/2}\int_{\mathbb{R}^n}e^{ix\cdot\eta}\lambda_{t, s}(x, \eta)\tilde{q}_2\Big(s, \eta - \xi - \frac{\rho}{k}, \xi + \frac{\rho}{k}\Big)\hat{\chi}_1(\rho)\,\mathrm{d}\rho\,\mathrm{d}\eta.
\end{aligned}
$$
For the integrand, by \cite[Lemma 6.2.B]{RZ} and Peetre's inequality the following estimate holds,
$$
\begin{aligned}
& \, \bigg|e^{ix\cdot\eta}\lambda_{t, s}(x, \eta)\tilde{q}_2\Big(s, \eta - \xi - \frac{\rho}{k}, \xi + \frac{\rho}{k}\Big)\hat{\chi}_1(\rho)\bigg|\\
\le & \, \bigg|\tilde{q}_2\Big(s, \eta - \xi - \frac{\rho}{k}, \xi + \frac{\rho}{k}\Big)\bigg|\big|\hat{\chi}_1(\rho)\big|\\
\le & \, 2Cc_\psi\|\varphi_0\|_\infty\bigg(1 + \Big|\eta - \Big(\xi + \frac{\rho}{k}\Big)\Big|^2\bigg)\bigg(1 + \psi\Big(\xi + \frac{\rho}{k}\Big)\bigg)\big|\hat{\chi}_1(\rho)\big|\\
\le & \, 2^{n + 2}C'\big(1 + |\eta|^2\big)^{-(n + 1)/2}\big(1 + |\xi|^2\big)^{(n + 1)/2}\bigg(1 + \Big|\frac{\rho}{k}\Big|^2\bigg)^{(n + 1)/2}\\
& \,\,\, \times \big(1 + \psi(\xi)\big)\bigg(1 + \psi\Big(\frac{\rho}{k}\Big)\bigg)\big|\hat{\chi}_1(\rho)\big|\\
\le & \, C''\big(1 + |\xi|^2\big)^{(n + 3)/2}\big(1 + |\eta|^2\big)^{-(n + 1)/2}\big(1 + |\rho|^2\big)^{(n + 3)/2}\big|\hat{\chi}_1(\rho)\big|.
\end{aligned}
$$
The right-hand side is integrable with respect to $\eta$ and $\rho$ and the estimate is uniform in $x \in \mathbb{R}^n$, $t \ge s \ge 0$ and $k \in \mathbb{N}$.  Therefore we can use dominated convergence and arrive at,
$$
\begin{aligned}
& \lim_{k \to \infty}U(t, s)q_2(s, x, D)\big(e_\xi\chi_k\big)(x)\\
= & \, (2\pi)^{-n}\int_{\mathbb{R}^n}\int_{\mathbb{R}^n}e^{ix\cdot\eta}\lambda_{t, s}(x, \eta)\lim_{k \to \infty}\tilde{q}_2\Big(s, \eta - \xi - \frac{\rho}{k}, \xi + \frac{\rho}{k}\Big)\hat{\chi}_1(\rho)\,\mathrm{d}\rho\,\mathrm{d}\eta\\
= & \, (2\pi)^{-n}\int_{\mathbb{R}^n}\int_{\mathbb{R}^n}e^{ix\cdot\eta}\lambda_{t, s}(x, \eta)\tilde{q}_2(s, \eta - \xi, \xi)\hat{\chi}_1(\rho)\,\mathrm{d}\rho\,\mathrm{d}\eta\\
= & \, \bigg((2\pi)^{-n/2}\int_{\mathbb{R}^n}e^{i0\cdot\rho}\hat{\chi}_1(\rho)\,\mathrm{d}\rho\bigg)\bigg((2\pi)^{-n/2}\int_{\mathbb{R}^n}e^{ix\cdot\eta}\lambda_{t, s}(x, \eta)\tilde{q}_2(s, \eta - \xi, \xi)\,\mathrm{d}\eta\bigg)\\
= & \, \chi_1(0)(2\pi)^{-n/2}\int_{\mathbb{R}^n}e^{ix\cdot\eta}\lambda_{t, s}(x, \eta)(2\pi)^{-n/2}\int_{\mathbb{R}^n}e^{-i(\eta - \xi)\cdot \zeta}q_2(s, x, \xi)\,\mathrm{d}\zeta\,\mathrm{d}\eta\\
= & \, (2\pi)^{-n/2}\int_{\mathbb{R}^n}e^{ix\cdot\eta}\lambda_{t, s}(x, \eta)\big(q_2(s, \cdot, \xi)e_\xi\big)^\wedge(\eta)\,\mathrm{d}\eta\\
= & \, U(t, s)\big(q_2(s, \cdot, \xi)e_\xi\big)(x),
\end{aligned}
$$
and,
$$
\sup_{x \in \mathbb{R}^n}\sup_{t \ge s \ge 0}\sup_{k \in \mathbb{N}}\big|U(t, s)q_2(s, x, D)\big(e_\xi\chi_k\big)(x)\big| \le c_{n, \psi, \chi_1, \xi}.
$$
\end{proof}

The following lemma has essentially the same proof.
\begin{lemma}\label{TDSSoAPL4}
Assume Assumptions \ref{TDSTDSA1} we get that,
$$
\lim_{k \to \infty}U(t, s)q_1(s, D)\big(e_\xi\chi_k\big)(x) = U(t, s)\big(q_1(s, \xi)e_\xi(x)\big),
$$
holds true and for every $\xi \in \mathbb{R}^n$,
$$
\big|U(t, s)q_1(s, D)\big(e_\xi\chi_k\big)(x)\big| \le c_\xi,
$$
uniformly in $x \in \mathbb{R}^n$, $t \ge s \ge 0$ and $k \in \mathbb{N}$.
\end{lemma}
\begin{proof}
Due to translation invariance,
$$
\mathrm{F}_{y \mapsto \eta}q_1(s, D)(e_\xi\chi_k)(\eta) = q_1(s, \eta)(e_\xi\chi_k)^\wedge(\eta),
$$
then,
$$
\begin{aligned}
& \, U(t, s)q_1(s, D)(e_\xi\chi_k)(x)\\
= & \, (2\pi)^{-n/2}\int_{\mathbb{R}^n}e^{ix\cdot\eta}\lambda_{t, s}(x, \eta)q_1(s, \eta)k^n\hat{\chi}_1\big(k(y - \xi)\big)\,\mathrm{d}\eta\\
= & \, (2\pi)^{-n/2}\int_{\mathbb{R}^n}e^{ix\cdot(\xi + \rho/k)}\lambda_{t, s}\Big(x, \xi + \frac{\rho}{k}\Big)q_1\Big(s, \xi + \frac{\rho}{k}\Big)\hat{\chi}_1(\rho)\,\mathrm{d}\rho.
\end{aligned}
$$
By Peetre's inequality the following estimate holds,
$$
\begin{aligned}
& \, \bigg|e^{ix\cdot(\xi + \rho/k)}\lambda_{t, s}\Big(x, \xi + \frac{\rho}{k}\Big)q_1\Big(s, \xi + \frac{\rho}{k}\Big)\hat{\chi}_1(\rho)\bigg|\\
\le & \, \bigg|q_1\Big(s, \xi + \frac{\rho}{k}\Big)\bigg|\big|\hat{\chi}_1(\rho)\big|\\
\le & \, \tau_1\bigg(1 + \psi\Big(\xi + \frac{\rho}{k}\Big)\bigg)\big|\hat{\chi}_1(\rho)\big|\\
\le & \, \tau_1'\big(1 + \psi(\xi)\big)\bigg(1 + \psi\Big(\frac{\rho}{k}\Big)\bigg)\big|\hat{\chi}_1(\rho)\big|\\
\le & \, \tau_1'c_\psi^2\big(1 + |\xi|^2\big)\big(1 + |\rho|^2\big).
\end{aligned}
$$
The right-hand side is integrable with respect to $\rho$ and is uniform in $t \ge s \ge 0$ and $k \in \mathbb{N}$.  Therefore we can use dominated convergence and arrive at,
$$
\begin{aligned}
& \, \lim_{k \to \infty}U(t, s)q_1(s, D)(e_\xi\chi_k)(x)\\
= & \, (2\pi)^{-n/2}\int_{\mathbb{R}^n}\lim_{k \to \infty}e^{ix\cdot(\xi + \rho/k)}\lambda_{t, s}\Big(x, \xi + \frac{\rho}{k}\Big)q_1\Big(s, \xi + \frac{\rho}{k}\Big)\hat{\chi}_1(\rho)\,\mathrm{d}\rho\\
= & \, e_\xi(x)\lambda_{t, s}(x, \xi)q_1(s, \xi)\,(2\pi)^{-n/2}\int_{\mathbb{R}^n}\hat{\chi}_1(\rho)\,\mathrm{d}\rho\\
= & \, e_\xi(x)e_{-\xi}(x)U(t, s)e_\xi(x)q_1(s, \xi)\\
= & \, U(t, s)q_1(s, \xi)e_\xi(x),
\end{aligned}
$$
and,
$$
\sup_{x \in \mathbb{R}^n}\sup_{t \ge s \ge 0}\sup_{k \in \mathbb{N}}\big|U(t, s)q_1(s, \xi)e_\xi(x)\big| \le c_{\psi, \chi_1, \xi}.
$$
\end{proof}

\begin{theorem}\label{TDSSoAPT3}
Under Assumptions \ref{TDSTDSA1} with $m \ge n + \lfloor a \rfloor + 1$ for $a \ge 1$ and $\varphi_0 \in \mathcal{L}^1(\mathbb{R}^n) \cap \mathcal{L}^\infty(\mathbb{R}^n)$, we have that,
\begin{equation}
\frac{\partial}{\partial s}\lambda_{t, s}(x, \xi)\bigg|_{s = t} = q(t, x, \xi), \,\,\,\, t \ge s,
\end{equation}
holds for all $t \ge 0$ and $x, \xi \in \mathbb{R}^n$.
\end{theorem}
\begin{proof}
From Lemma \ref{TDSSoAPL3} and \ref{TDSSoAPL4} we know that,
$$
\big|U(t, s)q(s, x, D)\big(e_\xi\chi_k\big)(x)\big| \le c_\xi, \,\,\,\, \xi \in \mathbb{R}^n,
$$
uniformly in $x \in \mathbb{R}^n$, $s \ge 0$ and $k \in \mathbb{N}$.  Since $e_\xi\chi_k \in D\big(-q(t, x, D)\big)$, by properties of our fundamental solution $U(t, s)$ we find, for all $t \ge 0$,
$$
\begin{aligned}
U(t, s)e_\xi(x) - e_\xi(x) & = \lim_{k \to \infty}\big(U(t, s)\big(e_\xi\chi_k\big)(x) - e_\xi(x)\chi_k(x)\big)\\
& = -\lim_{k \to \infty}\int_s^tU(t, r)q(r, x, D)\big(e_\xi\chi_k\big)(x)\,\mathrm{d}r\\
& = -\int_s^t\lim_{k \to \infty}U(t, r)q(r, x, D)\big(e_\xi\chi_k\big)(x)\,\mathrm{d}r\\
& = \int_t^sU(t, r)\big(q(r, \cdot, \xi)e_\xi\big)(x)\,\mathrm{d}r,
\end{aligned}
$$
by combining Lemma \ref{TDSSoAPL3} and \ref{TDSSoAPL4}.  Now we note that,
$$
U(t, t)u(x) = u(x), \,\,\,\, \text{for all } t \ge 0,
$$
and,
$$
U(t, t)u(x) = (2\pi)^{-n/2}\int_{\mathbb{R}^n}e^{ix\cdot\xi}\lambda_{t, t}(x, \xi)\hat{u}(\xi)\,\mathrm{d}\xi
$$
implies $\lambda_{t, t}(x, \xi) = 1$ for all $t \ge 0$ and $x, \xi \in \mathbb{R}^n$.  Now,
$$
\begin{aligned}
\frac{\lambda_{t, s}(x, \xi) - 1}{s - t} & = e_{-\xi}(x)\frac{U(t, s)e_\xi(x) - e_\xi(x)}{s - t}\\
& = \frac{e_{-\xi}(x)}{s - t}\int_t^sU(t, r)\big(q(r, \cdot, \xi)e_\xi\big)(x)\,\mathrm{d}r.
\end{aligned}
$$
Since $q(t, \cdot, \xi)e_\xi$ is bounded for all $t \ge 0$ and fixed $\xi \in \mathbb{R}^n$, 
$$
\begin{aligned}
& \, \lim_{s \to t}U(t, s)\big(q(s, \cdot, \xi)e_\xi\big)(x)\\
= & \, (2\pi)^{-n/2}\int_{\mathbb{R}^n}e^{ix\cdot\eta}\Big(\lim_{s \to t}\lambda_{t, s}(x, \eta)\Big)\Big(\lim_{s \to t}\big(q(s, \cdot, \xi)e_\xi\big)^\wedge(\eta)\Big)\,\mathrm{d}\eta\\
= & \, (2\pi)^{-n/2}\int_{\mathbb{R}^n}e^{ix\cdot\eta}\big(q(t, \cdot, \xi)e_\xi\big)^\wedge(\eta)\,\mathrm{d}\eta\\
= & \, q(t, x, \xi)e_\xi(x),
\end{aligned}
$$
and therefore,
$$
\begin{aligned}
& \, \left|\frac{1}{s - t}\int_t^sU(t, r)\big(q(r, \cdot, \xi)e_\xi\big)(x)\,\mathrm{d}r - q(t, x, \xi)e_\xi(x)\right|\\
= & \, \left|\frac{1}{s - t}\int_t^s\Big(U(t, r)\big(q(r, \cdot, \xi)e_\xi\big)(x) - q(t, x, \xi)e_\xi(x)\Big)\,\mathrm{d}r\right|\\
= & \, \left|\frac{1}{t - s}\int_s^t\Big(U(t, r)\big(q(r, \cdot, \xi)e_\xi\big)(x) - q(t, x, \xi)e_\xi(x)\Big)\,\mathrm{d}r\right|\\
\le & \, \frac{1}{t - s}\int_s^t\sup_{s \le r \le t}\Big|U(t, r)\big(q(r, \cdot, \xi)e_\xi\big)(x) - q(t, x, \xi)e_\xi(x)\Big|\,\mathrm{d}r\\
= & \, \sup_{s \le r \le t}\Big|U(t, r)\big(q(r, \cdot, \xi)e_\xi\big)(x) - q(t, x, \xi)e_\xi(x)\Big|
\end{aligned}
$$
which tends to $0$ as $s \to t$.  Thus,
$$
\begin{aligned}
\frac{\partial}{\partial s}\lambda_{t, s}(x, \xi)\bigg|_{s = t} & = \lim_{s \to t}\frac{\lambda_{t, s}(x, \xi) - 1}{s - t}\\
& = e_{-\xi}(x)q(t, x, \xi)e_\xi(x)\\
& = q(t, x, \xi).
\end{aligned}
$$
\end{proof}
\begin{corollary}
In the case of Theorem \ref{TDSSoAPT3} we have,
\begin{equation}
\frac{\partial}{\partial t}\lambda_{t, s}(x, \xi)\bigg|_{t = s} = -q(s, x, \xi).
\end{equation}
\end{corollary}
\begin{proof}
By the strong continuity of $U(t, s)$,
$$
\begin{aligned}
\frac{\partial}{\partial t}\lambda_{t, s}(x, \xi)\bigg|_{t = s} & = \lim_{t \to s}\frac{\lambda_{t, s}(x, \xi) - 1}{t - s}\\
& = -\lim_{s \to t}\frac{\lambda_{t, s}(x, \xi) - 1}{s - t}\bigg|_{t = s}\\
& = -q(s, x, \xi).
\end{aligned}
$$
\end{proof}

\newpage\null

\chapter*{Index of Notation}
\addcontentsline{toc}{chapter}{Index of Notation}

$\mathbb{N}$ - natural numbers

\noindent$\mathbb{N}_0 = \mathbb{N} \cup \{0\}$

\noindent$\mathbb{N}_0^n$ - set of all multiindices

\noindent$\mathbb{R}$ - real numbers

\noindent$\mathbb{R}^n$ - Euclidean vector space

\noindent$\mathbb{C}$ - complex numbers

\noindent$a \wedge b = \min\{a, b\}$

\noindent$a \vee b = \max\{a, b\}$

\noindent$|\alpha| = \alpha_1 + \dots +\alpha_n$ for $\alpha \in \mathbb{N}_0^n$

\noindent$\partial^\alpha u = \frac{\partial^{|\alpha|}u}{\partial^{\alpha_1}x_1\dots\partial^{\alpha_n}x_n}$ for $\alpha \in \mathbb{N}_0^n$

\bigbreak

\noindent$\chi_A$ - characteristic function of the set $A$

\noindent$f|_A$ - restriction of $f$ to $A$

\noindent$\text{Re}\,f$ - real part of $f$

\noindent$\text{Im}\,f$ - imaginary part of $f$

\noindent$(f_\nu)_{\nu \in \mathbb{N}}$ - sequence of functions

\noindent$f \circ g$ - composition of functions

\noindent$f \ast g$ - convolution of functions

\noindent$K_\lambda$ - modified Bessel function of the second kind with index $\lambda$

\noindent$\mathcal{L}u$ - Laplace transform of $u$

\noindent$\hat{u}, \mathrm{F}u$ - Fourier transform of $u$

\noindent$\mathrm{F}^{-1}u$ - inverse Fourier transform of $u$

\noindent$\text{supp}\,u$ - support of a function or distribution

\bigbreak

\noindent$\mathcal{B}^{(n)}$ - Borel sets in $\mathbb{R}^n$

\noindent$\lambda^{(n)}$ - Lebesgue measure in $\mathbb{R}^n$

\noindent$\varepsilon_a$ - Dirac measure at $a \in \mathbb{R}^n$

\noindent$\mu_1 \ast \mu_2$ - convolution of the measures $\mu_1$ and $\mu_2$

\noindent$\|\mu\|$ - total mass of a measure $\mu$

\noindent$\text{supp}\,\mu$ - support of a measure $\mu$

\noindent$(\mu_t)_{t \ge 0}$ - convolution semigroup of measures

\noindent$(\mu_t^f)_{t \ge 0}$ - subordinate convolution semigroup

\bigbreak

\noindent For this section $G \subseteq \mathbb{R}^n$ is open, $K \subset \mathbb{R}^n$ is compact.  If $X(\Omega)$ is any of the spaces in the following list, $X(\Omega; \mathbb{R})$ stands for the space of real-valued elements of $X(\Omega)$.

\noindent$B(\Omega)$ - Borel measurable functions

\noindent$B_b(\Omega)$ - bounded Borel measurable functions

\noindent$C(G)$ - continuous functions

\noindent$C_0(G)$ - continuous functions with compact support

\noindent$C_\infty(G)$ - continuous functions vanishing at infinity

\noindent$C_b(G)$ - bounded continuous functions

\noindent$C^m(G)$ - m-times continuously differentiable functions

\noindent$C_0^m(G) = C^m(G) \cap C_0(G)$

\noindent$C^\infty(G) = \bigcap_{m \in \mathbb{N}}C^m(G)$

\noindent$C_0^\infty(G) = \bigcap_{m \in \mathbb{N}}C_0^m(G)$

\noindent$C_0^\infty(K)$ - arbitrarily often differentiable functions which have compact support in $K$

\noindent$\mathcal{D}'(G)$ - space of all distributions

\noindent$H^{\psi, s}(\mathbb{R}^n) = \{u \in \mathcal{S}'(\mathbb{R}^n) : \|u\|_{\psi, s} < \infty\}$

\noindent$\mathcal{L}^p(\Omega, \mu)$ - usual Lebesgue space over $(\Omega, \mathcal{A}, \mu)$

\noindent$\mathcal{L}^p(\mathbb{R}^n) = \mathcal{L}^p\big(\mathbb{R}^n, \lambda^{(n)}\big)$

\noindent$\mathcal{M}^+(\Omega)$ - measures on $\Omega$

\noindent$\mathcal{M}_b^+(\Omega)$ - bounded measures on $\Omega$

\noindent$\mathcal{M}_b^1(\Omega)$ - probability measures on $\Omega$

\noindent$\mathcal{S}(\mathbb{R}^n)$ - Schwartz space of tempered functions

\noindent$\mathcal{S}'(\mathbb{R}^n)$ - tempered distributions

\bigbreak

\noindent$|x|$ - Euclidean distance of $x$ in $\mathbb{R}^n$

\noindent$\|u\|_X$ - norm of $u$ in the space of $X$

\noindent$\|u\|_{A, X} = \|u\|_X + \|Au\|_X$ - graph norm of $u$ with respect to the operator $A : X \to X$

\noindent$\|A\| = \|A\|_{X, Y}$ - operator norm of the operator $A : X \to Y$

\noindent$\|u\|_0$, $\langle u, u\rangle_0$ - norm and scalar product in $\mathcal{L}^2(\Omega, \mu)$

\noindent$\|u\|_\infty = \sup|u(x)|$

\noindent$\|u\|_{\psi, s}$ - norm in the space of $H^{\psi, s}(\mathbb{R}^n)$

\bigbreak

\noindent$(X, \|\cdot\|_X)$ - Banach space with norm $\|\cdot\|_X$

\noindent$(X, d)$ - metric space with metric $d(\cdot, \cdot)$ on $X \times X$

\noindent$(X, d, \mu)$ - metric measure space with metric $d$ and measure $\mu$

\noindent$B^d(x, r)$ - open ball of radius $r$ and centre $x$ with respect to metric $d$

\noindent$X^\ast$ - topological dual space of a topological space $X$

\noindent$X \hookrightarrow Y$ - continuous embedding of $X$ into $Y$

\noindent$B(X)$ - bounded linear operators from $X$ into itself

\noindent$D(A)$ - domain of the operator $A$

\noindent$\big(A, D(A)\big)$ - linear operator with domain $D(A)$

\noindent$\psi(D)$ - pseudo-differential operator with symbol $\psi(\xi)$

\noindent$q(x, D)$ - pseudo-differential operator with symbol $q(x, \xi)$

\noindent$q(t, x, D)$ - pseudo-differential operator with symbol $q(t, x, \xi)$

\noindent$(T_t)_{t \ge 0}$ - one-parameter semigroup of operators

\noindent$(T_t^f)_{t \ge 0}$ - subordinated semigroup





\begin{thebibliography}{99}

\bibitem{I.1} Adams D R, Hedberg L I.  \emph{Function spaces and potential theory}.  Grundlehren der mathematischen Wissenschaften, Vol. 314, Springer Verlag, Berlin 1996

\bibitem{SPRef} Bauer H.  \emph{Probability theory}. de Gruyter Studies in Mathematics, Vol 23, Berlin 1996

\bibitem{I.25} Berg C, Forst G.  \emph{Potential theory on locally compact Abelian groups}.  Ergebnisse der Mathematik und ihrer Grenzgebiete (Ser.II), Vol. 87, Springer Verlag, Berlin 1975

\bibitem{KernelPaper} B\"ottcher B.  \emph{Construction of time-inhomogeneous Markov processes via evolution equations using pseudo-differential operators}.  J. London Math. Soc., 2008, (2) 78: 605-621

\bibitem{I.88} Ethier S N, Kurtz T G.  \emph{Markov Processes - characterization and convergence}.  Wiley Series in Probability and Mathematical Statistics, John Wiley \& Sons, New York 1986

\bibitem{II.87} Farkas W, Jacob N, Schilling R L.  \emph{Function spaces related to continuous negative definite functions: $\Psi$-Bessel potential spaces}.  Dissertationes Mathematicae CCCXCIII (2001), 1-62

\bibitem{m-stableEstimate} Grigor'yan A, Liu L.  \emph{Heat kernel and Lipschitz-Besov spaces}.  Forum Mathematicum, 2014, ISSN (Online) 1435-5337, ISSN (Print) 0933-7741, doi: 10.1515/forum-2014-0034

\bibitem{22in8} Heinonen J.  \emph{Lectures on Analysis on Metric Spaces}.  Springer Verlag, New York 2001

\bibitem{SymbolPaper} Jacob N.  \emph{Characteristic functions and symbols in the theory of Feller processes}.  Potential Analysis, 1998, 8: 61-68

\bibitem{II.149} Jacob N.  \emph{Dirichlet forms and pseudo differential operators}.  Exop. Math. 6 (1988), 363-371

\bibitem{Vol1} Jacob N.  \emph{Pseudo-Differential Operators and Markov Processes, vol. 1: Fourier Analysis and Semigroups}.  Imperial College Press, London 2001

\bibitem{Vol2} Jacob N.  \emph{Pseudo-Differential Operators and Markov Processes, vol. 2: Generators and Their Potential Theory}.  Imperial College Press, London 2002

\bibitem{Vol3} Jacob N.  \emph{Pseudo-Differential Operators and Markov Processes, vol. 3: Markov Processes and Applications}.  Imperial College Press, London 2005

\bibitem{Paper} Jacob N, Knopova V, Landwehr S, Schilling R L.  \emph{A geometric intepretation of the transition density of a symmetric L\'evy process}.  Sci China Math, 2012, 55(6): 1099-1126, doi: 10.1007/s11425-012-4368-0

\bibitem{EstimatesPaper} Jacob N, Schilling R L.  \emph{Estimates for Feller semigroups generated by pseudodifferential operators}.  In: Rakosnik, J. (ed.), Function Spaces, Differential Operators and Nonlinear Analysis.  Prometheus Publishing House, Praha 1996, 27-49

\bibitem{DProofPaper} Knopova V, Schilling R L.  \emph{A note on the existence of transition probability densities for L\'evy processes}.  Forum Math, 2013, 25: 125-149

\bibitem{I.235} Pazy A.  \emph{Semigroups of linear operators and applications to partial differential equations}.  Applied Mathematical Sciences, Vol. 44, Springer Verlag, New York 1983

\bibitem{Sato} Sato K.  \emph{L\'evy Processes and Infinitely Divisible Distributions}.  Cambridge University Press, Cambridge 1999

\bibitem{Conservative} Schilling R L.  \emph{Conservativeness of semigroups generated by pseudo-differential operators}.  Potential Analysis, 1998, 9: 91-104

\bibitem{FubiniBook} Schilling R L.  \emph{Measures, Integrals and Martingales}.  Cambridge University Press, Cambridge 2006

\bibitem{Bernstein} Schilling R L, Song R, Vondracek Z.  \emph{Bernstein Functions.  Theory and Application}.  Walter de Gruyter, Berlin 2010

\bibitem{II.256} Stein E M.  \emph{Singular integrals and differentiability properties of functions}.  Princeton Mathematical Series, Vol. 30, Princeton University Press, Princeton NJ 1970

\bibitem{45in8} Sturm K T.  \emph{Diffusion processes and heat kernels on metric spaces}.  Ann Probab, 1998, 26: 1-55

\bibitem{Tanabe} Tanabe H.  \emph{Equations of Evolution}.  Pitman Publishing Ltd., Bath 1979

\bibitem{RZ} Zhang R. \emph{Fundamental solutions of a class of pseudo-differential operators with time-dependent negative definite symbols}.  PhD Thesis. Swansea University, Swansea 2011

\end{thebibliography}

\end{document}